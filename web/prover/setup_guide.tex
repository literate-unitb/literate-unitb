\documentclass{article}
\usepackage{hyperref}
\usepackage[utf8]{inputenc}
\usepackage{listings}
\usepackage{titling}
\usepackage[dvipsnames]{xcolor}

\definecolor{code-fg}{HTML}{C7254E}
\definecolor{code-bg}{HTML}{F9F2F4}
\definecolor{url-fg}{HTML}{099099}
\colorlet{link-fg}{violet!85!black}


\lstset{
  basicstyle=\ttfamily\color{code-fg}
}

\lstdefinestyle{inline}{
  columns=fullflexible,
  breaklines=false,
}

\newcommand{\code}[1]
{\colorbox{code-bg}{\lstinline[style=inline]!#1!}}

\hypersetup{colorlinks=true, % Don't draw a box around links, instead color every
                             % piece of text, which is a link
  urlcolor=url-fg, % Color them red
  linkcolor=link-fg, % Color them red
  hypertexnames=false, % Use a default object counter to count links (chapters,
                       % sections, etc. appear as 1,2,3 instead of 1, 1.1, etc.)
  pdfhighlight=/N % When clicking a link and holding the mouse button, don't use
                  % any graphical effects - i.e. the alternative would be to behave
                  % like a "button" which is pressed within the PDF
}

\title{Starting the Web Prover}
\author{
  Amin Bandali
  \\
  \href{mailto:amin9@my.yorku.ca}
  {\normalsize\texttt{amin9@my.yorku.ca}}
}
\date{\today}

\begin{document}

\setlength{\droptitle}{-2cm}

\maketitle

\section{Setup}

To start the web prover, the project needs to be set up first. Please follow the
instructions in \verb#README.md# to do that. Also, note that the web prover
uses the logic module of Literate Unit-B, which in turn uses the Z3
prover\footnote{\url{https://github.com/Z3Prover/z3}}. Make
sure Z3 is installed and its binary is in the \code{\$PATH} environment variable.

\section{Starting}

One of the possible options to set the web server up so that it keeps running
when the terminal is closed is to use GNU Screen. To do so, assuming that the
project is set up on the \verb#unitb# account, follow these steps:

\begin{enumerate}
\item \verb#ssh# into \code{red.cse.yorku.ca},

\item become the \verb#unitb# user by issuing \code{bu unitb},

\item run \code{screen -r || screen}, which re-attaches to an existing session
  if it exists, and creates a new one otherwise,

\item \code{cd src/haskell/literate-unitb/web/prover}, which changes directory
  into where the project resides (as of writing this guide),

\item start the web server by running \code{stack exec yesod devel}.
\end{enumerate}

Upon successful start of the server, the output should end similar to this:

\begin{verbatim}
...
Starting development server...
Starting devel application
Devel application launched: http://localhost:3000
\end{verbatim}

You can now navigate to \url{http://red.cse.yorku.ca:3000} in your browser to
access the web prover.\newline

At this stage you can close the terminal window if you wish, as you would close
any other window, and the web server should remain running.\newline

To stop the web server, follow steps 1-3 above, to re-attach to the screen
session, then type \textbf{quit} and press enter twice, and close the terminal.

\end{document}
