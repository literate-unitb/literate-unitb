
\documentclass[12pt]{amsart}
\usepackage{geometry} % see geometry.pdf on how to lay out the page. There's lots.
\usepackage{bsymb}
\usepackage{unitb}
\usepackage{calculation}
\usepackage{ulem}
\usepackage{hyperref}
\normalem
\geometry{a4paper} % or letter or a5paper or ... etc
% \geometry{landscape} % rotated page geometry

% See the ``Article customise'' template for some common customisations

\title{}
\author{}
\date{} % delete this line to display the current date

%%% BEGIN DOCUMENT
\begin{document}

\maketitle
\tableofcontents

%\section{}
%\subsection{}

\begin{machine}{m0}

%\hide{
	\begin{align*}
\variable{		n,a : \Int}
	\end{align*}
%}

\begin{align*}
\invariant{inv0}
{	a = n^3	}
\end{align*}
%
%%\begin{proof}{INIT/INV/inv0}
%%\begin{calculation}
%%	
%%\end{calculation}
%%\end{proof}

\newevent{evt}

%
\begin{align*}
\initialization{init0}
{	n = 0 \land a = 0	}
\end{align*}
%%
%%
\begin{align*}
\evassignment{evt}{a0}
{	n' = n + 1	}
\end{align*}
%
We use the proof obligation of \ref{inv0} to deduce a proper assignment to $a$:
%
\begin{proof}{evt/INV/inv0}
	\begin{calculation}
		(n')^3
	\hint{=}{ \ref{a0} }
		(n+1)^3
	\hint{=}{ arithmetic }
		n^3 + 3 \cdot n^2 + 3 \cdot n + 1
	\hint{=}{ \ref{inv0} }
		a + 3 \cdot n^2 + 3 \cdot n + 1
	\hint{=}{ we add a variable $b$: \ref{inv1}, see below }
		a + b
	\hint{=}{ \ref{a1} }
%	\hint{=}{ }
		a'
	\end{calculation}
\end{proof}
%
\begin{align*}
\variable{	b : \Int}
\end{align*}
%
\begin{align*}
\invariant{inv1}
{	b = 3 \cdot n^2 + 3 \cdot n + 1	}
\end{align*}
%%%%%\begin{invariant}{inv2}
%%%%%	b ~=~ 3 \cdot n^2 + 3 \cdot n + 1
%%%%%\end{invariant}
%%%%
\begin{align*}
\evassignment{evt}{a1}
{	a' = a + b	}
\end{align*}
%
We now have a new invariant to preserve. It is easy to see how to establish it initially:
%%
\begin{align*}
\initialization{in1}
{	b = 1	}
\end{align*}
%
%\begin{itemize}
%\item label initialization predicates
%\item spacing commands
%\item \sout{dummy section}
%\item \sout{test case: train system}
%\item refinement
%
%	add refinement environments for: changing schedules, transforming progress properties
%\item po labels: check that proofs match a po
%\item translate the tags in proof obligations
%	
%	create toc entries with the proof environment, in unitb.sty
%\end{itemize}
%
%\section{Todo:}
%\begin{itemize}
%\item grammar: make operator grammar generic
%\item grammar: include prefix operators and quantifiers
%\item 
%\item in Z3.Z3.merge, remove the exceptions to integrate it to error handling
%\item spacing in the middle of event and set declarations
%\item testing the input validation, error messages
%\item proof structures (proof by cases, etc)
%\item types
%\item invariant theorems
%\item error checking
%\item better LaTeX formatting
%\item lazy proof checking
%\item \sout{continuous checking}
%\item \sout{\emph{bug}: last of empty list of calculation steps}
%\item \sout{error report: report line number instead of step number when proof is incorrect}
%\item generate documentation
%\item handle the ``\textbackslash input'' commands in latex and use it to produce a project hierarchy
%\end{itemize}
%
%I'm now describing how I came up with the proof. It came to me in a dream and I forgot it in another dream.
%%
\begin{proof}{evt/INV/inv1}
	\begin{calculation}
		3 \cdot (n')^2 + 3 \cdot n' + 1
	\hint{=}{ \ref{a0} }
		3 \cdot (n+1)^2 + 3 \cdot (n+1) + 1
	\hint{=}{ arithmetic }
		3 \cdot (n^2+2\cdot n+1) + 3 \cdot (n+1) + 1
	\hint{=}{ arithmetic }
		3 \cdot n^2+6\cdot n+3 + 3 \cdot n + 3 + 1
	\hint{=}{ arithmetic }
		3 \cdot n^2 + 3 \cdot n + 1 +6\cdot n+3 + 3
	\hint{=}{ \ref{inv1} }
		b+6\cdot n+3+3
	\hint{=}{ \ref{inv2}, see below }
		b+c
	\hint{=}{ \ref{a2}, see below }
%	\hint{=}{ }
		b'
	\end{calculation}
\end{proof}
%
\begin{align*}
\variable{	c : \Int}
\end{align*}
%
\begin{align*}
\invariant{inv2}
{	c = 6 \cdot n + 6	}
\end{align*} 
%
\begin{align*}
\evassignment{evt}{a2}
{	b' = b + c	}
\end{align*}

Invariant \ref{inv2} is easy to satisfy initially:

\begin{align*}
\initialization{in2}
{	c = 6	}
\end{align*}
%
\begin{proof}{evt/INV/inv2}

	If we increase $c$ by 6, we can easily see that \ref{inv2} is preserved:
\begin{align*}
\evassignment{evt}{a3}
{	c' = c + 6	}
\end{align*}

	\easy
%	\begin{calculation}
%		6 \cdot (n') + 6
%	\hint{=}{ \ref{a0} }
%		6 \cdot (n+1) + 6
%	\hint{=}{ arithmetic }
%		6 \cdot n + 6 + 6
%	\hint{=}{ \ref{inv2} }
%		c + 6
%	\hint{=}{ \ref{a3} }
%		c'
%	\end{calculation}
\end{proof}
%%

\begin{align*}
\invariant{inv3}
{	f = \qfun{i}{ 0 \le i \land i < n }{ i^3 } }
\end{align*}

\begin{align*}
\variable{	f : \Int \pfun \Int}
\end{align*}

\begin{align*}
\dummy{	i : \Int}
\end{align*}

\with{functions} \with{sets}

\begin{proof}{INIT/INV/inv3}
	\begin{calculation}
		\qfun{i}{ 0 \le i \land i < n }{ i^3 }
	\hint{=}{ \ref{init0} }
		\qfun{i}{ 0 \le i \1\land i < 0 }{ i^3 }
	\hint{=}{ arithmetic }
%		\qfun{i}{ \false }{ i^3 }
%	\hint{=}{ arithmetic }
		\oftype{ \emptyfun }{ \pfun [ \Int, \Int ] }
	\hint{=}{ \ref{init3} }
		f
	\end{calculation}
\end{proof}

\begin{align*}
\initialization{init3}
{	f = \emptyfun	}
\end{align*}

\begin{proof}{evt/INV/inv3}
	\begin{calculation}
		\qfun{i}{ 0 \le i \land i < n' }{ i^3 }
	\hint{=}{ \ref{a0} }
		\qfun{i}{ 0 \le i \1\land i < n+1 }{ i^3 }
	\hint{=}{ split range with \ref{inv4} (see below) }
		\qfun{i}{ 0 \le i \1\land i < n }{ i^3 } \3 | n \tfun n^3
	\hint{=}{ \ref{inv3} }
%		\qfun{i}{ 0 \le i \1\land i < n }{ i^3 } \3 | \qfun{i}{ i = n }{ i^3 }
%	\hint{=}{ one-point rule }
%		\qfun{i}{ 0 \le i \1\land i < n }{ i^3 } \2 | n \tfun n^3
%	\hint{=}{ \ref{inv3} }
		f \1 | n \tfun n^3
	\hint{=}{ \ref{inv0} }
		f \1 | n \tfun a
	\hint{=}{ \ref{a4} }
		f'
	\end{calculation}
\end{proof}

\begin{align*}
\invariant{inv4}
{ 0 \le n }
\end{align*}

\begin{align*}
\evassignment{evt}{a4}
{ f' \2 = f \1 | n \tfun a }
\end{align*}

\begin{align*}
\invariant{inv5}
{	\qforall{i}{0 \le i \1\land i < n}{ f.i = i^3 }		}
\end{align*}

\begin{proof}{evt/INV/inv5}
	\begin{align}
	\assert{asm0}{ \qforall{i}{}{ i \in \dom.f \2\equiv 0 \le i \land i < n } }
	\end{align}
	\begin{calculation}
		\qforall{i}{0 \le i \1\land i < n'}{ f'.i = i^3 }
	\hint{=}{ \ref{a0} }
		\qforall{i}{0 \le i \1\land i < n+1}{ f'.i = i^3 }
	\hint{=}{ split range with \ref{inv4} }
		\qforall{i}{0 \le i \1\land i < n}{ f'.i = i^3 } \2 \land f'.n = n^3
	\hint{=}{ \ref{a4} and \ref{inv0} }
		\qforall{i}{0 \le i \1\land i < n}{ f'.i = i^3 } % \2 \land a = n^3
	\hint{=}{ \eqref{asm0} with \ref{a4}  }
		\qforall{i}{0 \le i \1\land i < n  }
			{ f.i = i^3 } 
	\hint{=}{ \ref{inv5} }
%		\qforall{i}{0 \le i \1\land i < n 
%			\1\land i \in \dom.f \setminus \dom.(n \tfun a) }
%			{ f'.i = i^3 } 
%	\hint{=}{  }
		\true
	\end{calculation}
	\begin{subproof}{asm0}
	\easy
	\end{subproof}
\end{proof}

\begin{align*}
\invariant{inv6}
{	\dom.f = \qset{i}{0 \le i \land i < n }{ i }		}
\end{align*}

\end{machine}

\end{document}