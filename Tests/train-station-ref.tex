
\documentclass[12pt]{amsart}
\usepackage{geometry} % see geometry.pdf on how to lay out the page. There's lots.
\usepackage{bsymb}
\usepackage{unitb}
\usepackage{calculation}
\usepackage{ulem}
\usepackage{hyperref}
\normalem
\geometry{a4paper} % or letter or a5paper or ... etc
% \geometry{landscape} % rotated page geometry

% See the ``Article customise'' template for some common customisations

\title{}
\author{}
\date{} % delete this line to display the current date

%%% BEGIN DOCUMENT
\setcounter{tocdepth}{4}
\begin{document}

\maketitle
\tableofcontents

%\section{}
%\subsection{}

\newcommand{\Train}{\text{TRAIN}}
\newcommand{\Blk}{\text{BLK}}

\section{Initial model}
\begin{machine}{m0}

\begin{align*}
	\false \tag{DEFAULT} \label{default}
\end{align*}

\newset{TRAIN}{\Train}

\begin{use:set}{\Train} \end{use:set}

\begin{align*}
\variable{	in : \set [ \Train ]}
\end{align*}

\begin{align*}
\dummy{	t : \Train}
\end{align*}

%\begin{align*}
\newevent{m0:enter}{enter} 
\newevent{m0:leave}{leave}
%\end{align*}
%\newcompound{m0:enter}
%\newcompound{m0:leave}

\begin{align*}
\indices{m0:leave}{	t : \Train}
\end{align*}
\begin{align*}
\indices{m0:enter}{	t : \Train}
\end{align*}

\weakento{m0:leave}{0}{default}{c0}

\begin{align*}
\cschedule{m0:leave}{c0}
	{	t &\in in } \\ 
\evassignment{m0:leave}{a0}
	{	in' &= in \setminus \{ t \} } \\
\evassignment{m0:enter}{a1}
	{	in' &= in \bunion \{ t \} }
\end{align*}

\begin{align*}
&\progress{m0:prog0}
	{	t \in in }{ \neg t \in in }
\refine{m0:prog0}{discharge}{m0:tr0}{}
&\transient{m0:leave}{0}{m0:tr0}
	{	t \in in }
\end{align*}

\end{machine}

\section{First refinement}
\begin{machine}{m1}



\refines{m0}

\newset{BLK}{\Blk}
%\newset{TRAIN}{\Train}

%\begin{use:set}{\Train} \end{use:set}
\begin{use:set}{\Blk} \end{use:set}
\begin{use:fun}{\Train}{\Blk} \end{use:fun}

%%\begin{variable}
%%	in : \set [ \Train ]
%%\end{variable}

\begin{align*}
\variable{	loc : \Train \pfun \Blk}
\end{align*}

%\begin{dummy}
%	t : \Train
%\end{dummy}

%\newevent{m1:enter}{enter}
%
%\newevent{m1:leave}{leave}

%\begin{indices}{m1:leave}
%	t : \Train
%\end{indices}
%
%\begin{indices}{m1:enter}
%	t : \Train
%\end{indices}

\begin{align*}
%\cschedule{m1:leave}{c0}
%	{	t &\in in } \\ 
%\evassignment{m1:leave}{a0}
%	{	in' &= in \setminus \{ t \} }
\\ \initialization{in1}
	{ in = \emptyset }
\end{align*}

%\begin{align*}
%\evassignment{m1:enter}{a0}
%	{	in' &= in \bunion \{ t \} }
%\end{align*}

\begin{align*}
\invariant{inv0}
	{	\dom.loc = in }
\end{align*}

\begin{align*}
\constant{	ent,ext : \Blk} ; \quad
\constant{	plf : \set [ \Blk ]}
\end{align*}

\subsection{New requirements}
\begin{align*}
\safety{saf0}
	{ \neg t \in in& }{ t \in in \land loc.t = ent }
\\ \safety{saf1}
	{ t \in in \land loc.t = ent& }{ t \in in \land loc.t \in plf }
\\ \safety{saf2}
	{ t \in in \land loc.t \in plf& }{ t \in in \land loc.t = ext }
\\ \safety{saf3}
	{ t \in in \land loc.t = ext& }{ \neg t \in in }
\end{align*}

\subsection{Proofs}

\subsubsection{Invariant \ref{inv0}}

\begin{align*}
\evassignment{m0:leave}{a2}
	{ loc' &= \{ t \} \domsub loc }
\\ \evassignment{m0:enter}{a3}
	{ loc' &= loc \1| t \tfun ent }
\\ \initialization{in0}
	{ loc \, &= \emptyfun }
\end{align*}

\subsubsection{Safety \ref{saf1}, \ref{saf2}}

This takes care of \eqref{saf2}:

\begin{align*}
\evguard{m0:leave}{grd1}
	{ loc.t = ext }
\end{align*}

This takes care of \eqref{saf1}

\begin{align*}
\evguard{m0:leave}{grd0}
	{ t \in in }
\\ \assumption{asm0}
	{ \neg ext \in plf \1\land \neg ext = ent }
\end{align*}

\subsubsection{Side conditions}

In order to take care of the schedulability of \ref{m0:leave}, we need to strengthen the coarse schedule by adding \ref{c1}. The side conditions for schedule replacement requires us to prove \ref{m1:prog0} and \ref{saf3} in order to prove refinement.
\replace{m0:leave}{1}{c0}{c0,c1}{}{m1:prog0}{saf3}

\begin{align*}
\cschedule{m0:leave}{c1}
	{ loc.t = ext }
%\\ \safety{saf4}
%	{ t \in in \land loc.t = ext }{ \neg t \in in }
\\ \progress{m1:prog0}
	{ t \in in }{ t \in in \land loc.t = ext }
\end{align*}

The first step in implementing \eqref{m1:prog0} is to break it down in a few cases. The property says ``a train inside the station eventually reaches the exit''. Now, we're going to break the predicate ``the train is inside the station'' into ``the train is at the entrance'', ``the train is at a platform'' and ``the train is at the exit'', as described by the following assumption:
\hide{
	\begin{align*}
	\dummy{		b : \Blk}
	\end{align*}
}
\begin{align*}
\assumption{asm1}
{	\qforall{b}{}{ b \in \Blk \2\equiv b \in plf \1\lor b = ent \1\lor b = ext }	}
\end{align*}

Which allows us to appy \emph{disjunction} to \eqref{m1:prog0}.

\begin{align*}
& { t \in in }\3\mapsto{ t \in in \land loc.t = ext }
\refine{m1:prog0}{disjunction}{prog1,prog2,prog3}{}
& \progress{prog2}
	{ t \in in \land loc.t = ext }{ t \in in \land loc.t = ext }
\\ & \progress{prog1}
	{ t \in in \land loc.t = ent }{ t \in in \land loc.t = ext }
\\ & \progress{prog3}
	{ t \in in \land loc.t \in plf }{ t \in in \land loc.t = ext }
\end{align*} 
%
\hide{ \refine{prog2}{implication}{}{} } %
%
\eqref{prog2} is true by implication. This leaves us \eqref{prog1} and \eqref{prog3}. \eqref{prog1} says that a train at the entrance reaches the exit. However, \eqref{saf1} says that a train can only leave the entrance through a platform:

\begin{align*}
	& { t \in in \land loc.t = ent }\3\mapsto{ t \in in \land loc.t = ext } \tag{\ref{prog1}}
\refine{prog1}{transitivity}{prog4,prog3}{}
& \progress{prog4}
	{ t \in in \land loc.t = ent }{ t \in in \land loc.t \in plf } 
\\ & { t \in in \land loc.t \in plf }\3\mapsto{ t \in in \land loc.t = ext } \tag{\ref{prog3}}
\end{align*}

We introduce new events to satisfy \eqref{prog3} and \eqref{prog4}
%\begin{align*}
\newevent{m1:movein}{move\_in} 
\newevent{m1:moveout}{move\_out} 
%\end{align*}
\hide{
	\refine{prog3}{discharge}{m1:tr0,saf2}{}
	\refine{prog4}{discharge}{m1:tr1,saf1}{}
}
\begin{align*}
%\\ & \safety{saf5}
%	{ t \in in \land loc.t \in plf }{ t \in in \land loc.t = ext }
& \transient{m1:movein}{0}{m1:tr1}
	{ t \in in \1\land loc.t = ent }
\\ & \transient{m1:moveout}{0}{m1:tr0}
	{ t \in in \1\land loc.t \in plf }
%\\ & \safety{saf6}
%	{ t \in in \land loc.t \in plf }{ t \in in \land loc.t = ext }
\end{align*}


\subsubsection{New events} 


\begin{align*}
\indices{m1:moveout}{	t : \Train}
\end{align*}

We adjust \ref{m1:moveout} to satisfy \ref{m1:tr0}.

\weakento{m1:moveout}{0}{default}{c1}{}

\begin{align*}
\evassignment{m1:moveout}{a2}{ loc' = loc \1 | t \tfun ext }
%\\ \evassignment{m1:moveout}{a3}{ in' = in }
\\ \assumption{asm2}
	{ \qexists{b}{}{b \in plf} }
\\ \cschedule{m1:moveout}{c1}{ t \in in \land loc.t \in plf }
\\ \evguard{m1:moveout}{g1}{ t \in in }
%\evassignment{m1:moveout}{a2}{ \qexists{b}{b \in plf }{ loc' = loc \1| t \tfun b } }
\end{align*}

We adjust \ref{m1:movein} to satisfy \ref{m1:tr1}.

\weakento{m1:movein}{0}{default}{c1,c2}{}

\begin{align*}
\indices{m1:movein}{	t : \Train} ; \quad
\param{m1:movein}{ b : \Blk }
\end{align*}

\begin{align*}
\evassignment{m1:movein}{a2}
	{ loc' = loc \1| t \tfun b }
%	{ \qexists{b}{b \in plf }{ loc' = loc \1| t \tfun b } }
%\end{align*}
%\begin{align*}
%\\ \evassignment{m1:movein}{a3}{ in' & = in }
\\ \assumption{asm3}
	{ \neg ent &\in plf }
\\ \cschedule{m1:movein}{c1}{ t &\in in } 
\\ \cschedule{m1:movein}{c2}{ loc.t &= ent }
\\ \evguard{m1:movein}{g1}{ t &\in in }
\end{align*}
%
\end{machine}

\section{Second refinement}

\begin{machine}{m2}

\refines{m1} 

%\newevent{m2:enter}{enter}
%\newevent{m2:leave}{leave}
%\newevent{m2:moveout}{move\_out}
%\newevent{m2:movein}{move\_in}
%
%\hide{
%\newset{BLK}{\Blk}
%\newset{TRAIN}{\Train}
%
%\begin{use:set}{\Train} \end{use:set}
%\begin{use:set}{\Blk} \end{use:set}
%\begin{use:fun}{\Train}{\Blk} \end{use:fun}
%
%\begin{variable}
%	in : \set [ \Train ]
%\end{variable}
%
%\begin{variable}
%	loc : \Train \pfun \Blk
%\end{variable}
%
\begin{align*}
\dummy{	t_0,t_1 : \Train}
\end{align*}
%
%\begin{indices}{m2:leave}
%	t : \Train
%\end{indices}
%
%\begin{indices}{m2:enter}
%	t : \Train
%\end{indices}
%
%\begin{align*}
%\cschedule{m2:leave}{c0}
%	{	t &\in in } \\ 
%\evassignment{m2:leave}{a0}
%	{	in' &= in \setminus \{ t \} }
%\\ \initialization{in1}
%	{ in = \emptyset }
%\end{align*}
%
%\begin{align*}
%\evassignment{m2:enter}{a0}
%	{	in' &= in \bunion \{ t \} }
%\end{align*}
%
%\begin{align*}
%\invariant{inv0}
%	{	\dom.loc = in }
%\end{align*}
%
%\begin{constant}
%	ent,ext : \Blk
%\end{constant}
%
%\begin{constant}
%	plf : \set [ \Blk ]
%\end{constant}
%
%\subsection{New requirements}
%\begin{align*}
%\safety{saf0}
%	{ \neg t \in in& }{ t \in in \land loc.t = ent }
%\\ \safety{saf1}
%	{ t \in in \land loc.t = ent& }{ t \in in \land loc.t \in plf }
%\\ \safety{saf2}
%	{ t \in in \land loc.t \in plf& }{ t \in in \land loc.t = ext }
%\\ \safety{saf3}
%	{ t \in in \land loc.t = ext& }{ \neg t \in in }
%\end{align*}
%
%\subsection{Proofs}
%
%\subsubsection{Invariant \ref{inv0}}
%
%\begin{align*}
%\evassignment{m2:leave}{a2}
%	{ loc' &= \{ t \} \domsub loc }
%\\ \evassignment{m2:enter}{a2}
%	{ loc' &= loc \1| t \tfun ent }
%\\ \initialization{in0}
%	{ loc = \emptyfun }
%\end{align*}
%
%\subsubsection{Safety \ref{saf1}, \ref{saf2}}
%
%\begin{align*}
%\evguard{m2:leave}{grd0}
%	{ t \in in }
%\\ \evguard{m2:leave}{grd1}
%	{ loc.t = ext }
%\end{align*}
%
%\begin{align*}
%\assumption{asm0}
%	{ \neg ext \in plf \1\land \neg ext = ent }
%\end{align*}
%
%\subsubsection{Side conditions}
%
%\begin{align*}
%\cschedule{m2:leave}{c1}
%	{ loc.t = ext }
%\end{align*}
%
%\begin{dummy}
%	b : \Blk
%\end{dummy}
%
%\begin{align*}
%\assumption{asm1}
%{	\qforall{b}{}{ b \in \Blk \2\equiv b \in plf \1\lor b = ent \1\lor b = ext }	}
%\end{align*}
%
%\subsubsection{New events} 
%
%\begin{indices}{m2:moveout}
%	t : \Train
%\end{indices}
%
%\begin{align*}
%\evassignment{m2:moveout}{a2}{ loc' = loc \1 | t \tfun ext }
%\\ \evassignment{m2:moveout}{a3}{ in' = in }
%\\ \assumption{asm2}
%	{ \qexists{b}{}{b \in plf} }
%\\ \cschedule{m2:moveout}{c1}{ t \in in \land loc.t \in plf }
%\\ \evguard{m2:moveout}{g1}{ t \in in }
%\end{align*}
%
%\begin{indices}{m2:movein}
%	t : \Train
%\end{indices}
%
%\begin{align*}
%\\ \evassignment{m2:movein}{a3}{ in' & = in }
%\\ \assumption{asm3}
%	{ \neg ent &\in plf }
%\\ \cschedule{m2:movein}{c1}{ t &\in in } 
%\\ \cschedule{m2:movein}{c2}{ loc.t &= ent }
%\\ \evguard{m2:movein}{g1}{ t &\in in }
%\end{align*}
%}
%
\subsection{New Requirement}
\begin{align*}
\invariant{m2:inv0}
	{	\qforall{t_0,t_1}{t_0 \in in \land t_1 \in in \land loc.t_0 = loc.t_1}{t_0 = t_1}	}
\end{align*}
%
\subsection{Design}
%
\replace{m1:movein}{1}{}{c0}{}{m2:prog0}{m2:saf0}
\begin{align*}
\progress{m2:prog0}{\true}{\qexists{b}{b \in plf}{ \qforall{t}{t \in in}{ \neg loc.t = b}} }
\\ \safety{m2:saf0}{\qexists{b}{b \in plf}{ \qforall{t}{t \in in}{ \neg loc.t = b}} }{\false} \2{\textbf{except}} \text{\ref{m1:movein}}
\end{align*}

\begin{align*}
\evguard{m1:movein}{g0}
	{	\qexists{b}{b \in plf}{ \qforall{t}{t \in in}{ \neg loc.t = b}} 	}
\\ \cschedule{m1:movein}{c0}
	{	\qexists{b}{b \in plf}{ \qforall{t}{t \in in}{ \neg loc.t = b}} 	}
\\ \evguard{m1:movein}{g3}
	{ \qforall{t}{t \in in}{ \neg loc.t = b} }
\\ \evguard{m0:enter}{g1}
	{	\qforall{t}{t \in in}{ \neg loc.t = ent} 	}
%\evguard{m0:leave}{g1}
%	{	\qforall{t}{t \in in}{ \neg loc.t = ext} 	}
\\ \evguard{m1:moveout}{g2}
	{	\qforall{t}{t \in in}{ \neg loc.t = ext} 	}
%%	{ \qexists{b}{}{ loc'  }
\end{align*}
%

\begin{proof}{m0:leave/INV/m2:inv0}
\begin{free:var}{t_0}{t_0}
\begin{free:var}{t_1}{t_1}
	\begin{align}
	\assume{hyp0}{\neg t_0 = t_1}
	\\ \assume{hyp1}{ t_0 \in in' }
	\\ \assume{hyp2}{ t_1 \in in' }
	\\ \assert{hyp5}{ t_0 \in in }
	\\ \assert{hyp6}{ t_1 \in in }
	\\ \assert{hyp3}{ \neg t_0 = t } % \label{hyp7}
	\\ \assert{hyp4}{ \neg t_1 = t } % \label{hyp8}
	\\ \goal{\neg (loc'.t_0 = loc'.t_1)} \notag
	\end{align}
	\hide{
		\assert{hyp7}{ t_0 \in \dom.loc \setminus \{ t \} }
		\assert{hyp8}{ t_1 \in \dom.loc \setminus \{ t \} }
		}
	\begin{calculation}
		loc'.t_0 = loc'.t_1
	\hint{=}{ \ref{a2} \ref{m2:inv0} }
		(\{ t \} \domsub loc).t_0 = (\{ t \} \domsub loc).t_1 
	\hint{=}{ \hide{ \eqref{hyp7} and \eqref{hyp8} } 
			\igeqref{hyp3} and \igeqref{hyp4} }
%		loc.t_0 = loc.t_1 
%		loc.t_0 = (\{ t \} \domsub loc).t_1 
%	\hint{=}{ \eqref{hyp8} }
		loc.t_0 = loc.t_1 
	\hint{=}{ \eqref{hyp5}, \eqref{hyp6}, 
			\eqref{hyp0} with \ref{m2:inv0} 
			} 
		\false
	\end{calculation}
	\hide
	{	\begin{subproof}{hyp3} \easy \end{subproof}
		\begin{subproof}{hyp4} \easy \end{subproof}
		\begin{subproof}{hyp5} \easy \end{subproof}
		\begin{subproof}{hyp6} \easy \end{subproof}
		}
	\begin{subproof}{hyp7} 
	\begin{calculation}
		t_0 \in \dom.loc \setminus \{ t \}
	\hint{=}{ \ref{inv0} }
		t_0 \in in \setminus \{ t \} 
	\hint{=}{ \ref{a0} }
		t_0 \in in'
	\hint{=}{ \eqref{hyp1} }
		\true
	\end{calculation}
	\end{subproof}
	Similarly for \eqref{hyp8}
	\hide{
		\begin{subproof}{hyp8} 
		\begin{calculation}
			t_1 \in \dom.loc \setminus \{ t \}
		\hint{=}{ \ref{inv0} }
			t_1 \in in \setminus \{ t \} 
		\hint{=}{ \ref{a0} }
			t_1 \in in'
		\hint{=}{ \ref{hyp2} }
			\true
		\end{calculation}
		\end{subproof}
		}
\end{free:var}
\end{free:var}
\end{proof}

\end{machine}

\end{document}