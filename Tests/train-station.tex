
\documentclass[12pt]{amsart}
\usepackage{geometry} % see geometry.pdf on how to lay out the page. There's lots.
\usepackage{bsymb}
\usepackage{unitb}
\usepackage{calculation}
\usepackage{ulem}
\normalem
\geometry{a4paper} % or letter or a5paper or ... etc
% \geometry{landscape} % rotated page geometry

% See the ``Article customise'' template for some common customisations

\title{}
\author{}
\date{} % delete this line to display the current date

%%% BEGIN DOCUMENT
\begin{document}

\maketitle
\tableofcontents

%\section{}
%\subsection{}

\begin{machine}{m0}

\newset{TRAIN}{\TRAIN}
\newset{BLK}{\BLK}
%\newset{LOC}{\LOC}
\newset{LOC}{\LOC}
%\end{set}

%
%\hide{
	\begin{variable}
		in : \set[\TRAIN]
	\end{variable}
%}
%

\newevent{enter}

\newevent{leave}

\begin{transient}{leave}{tr0}
	t \in in
\end{transient}

\begin{dummy}
	t : \TRAIN
\end{dummy}

\begin{variable}
	loc : \TRAIN \pfun \LOC
\end{variable}

\begin{invariant}{inv2}
%	loc \in in \tfun \BLK
	\dom . loc = in
\end{invariant}

\begin{evassignment}{enter}{a1}
	in' = in \bunion \{ t \}
\end{evassignment}

\begin{proof}{enter/INV/inv2}
	\begin{calculation}
		in'
	\hint{=}{ \ref{a1} }
		in \bunion \{ t \}
	\hint{=}{ \ref{inv2} }
		\dom.loc \bunion \{ t \}
	\hint{=}{ function calculus }
		\dom.loc \bunion \dom.(t \tfun ent)
	\hint{=}{ $\dom$ over $|$ }
		\dom.(loc   |   t \tfun ent)
	\hint{=}{ \ref{a2} }
		\dom. ( loc' )
	\end{calculation}
\end{proof}

\begin{evassignment}{enter}{a2}
	loc' = loc | (t \tfun ent)
\end{evassignment}

\begin{proof}{leave/INV/inv2}
	\begin{calculation}
		in'
	\hint{=}{ \ref{a0} }
		in \setminus \{ t \}
	\hint{=}{ \ref{inv2} }
		\dom.loc \setminus \{ t \}
	\hint{=}{ domain of singleton functions }
		\dom.(\{ t \} \domsub loc)
	\hint{=}{ \ref{a3} } 
		\dom. ( loc' )
	\end{calculation}
\end{proof}

\begin{evassignment}{leave}{a3}
	loc' = \{ t \} \domsub loc 
\end{evassignment}

\begin{cschedule}{leave}{c0}
	t \in in
\end{cschedule}

\begin{initialization}{in0}
	in = \emptyset
\end{initialization}
\begin{initialization}{in1}
	loc = \emptyfun
\end{initialization}

\begin{use:set}{\TRAIN} \end{use:set}
\begin{use:set}{\LOC} \end{use:set}
\begin{use:set}{\BLK} \end{use:set}
\begin{use:fun}{\TRAIN}{\BLK} \end{use:fun}
\begin{use:fun}{\TRAIN}{\LOC} \end{use:fun}

\begin{indices}{leave}
	t: \TRAIN
\end{indices}

\begin{indices}{enter}
	t: \TRAIN
\end{indices}

\begin{evassignment}{leave}{a0}
%	in' = in | (t \rightarrow \false)
	in' = in \setminus \{ t \}
%	in'.t = \false 
\end{evassignment}


\begin{constant}
	ent,plf,ext : \LOC
\end{constant}

\begin{assumption}{axm0}
	\LOC = \{ ent, plf, ext \}
\end{assumption}

%\begin{invariant}{inv0}
%	a = n^3
%\end{invariant}
%
%%\begin{proof}{INIT/INV/inv0}
%%\begin{calculation}
%%	
%%\end{calculation}
%%\end{proof}
%
%\newevent{evt}
%
%
%\begin{initialization}{init0}
%	n = 0 \land a = 0
%\end{initialization}
%
%
%\begin{evassignment}{evt}{a0}
%	n' = n + 1
%\end{evassignment}
%
%We use the proof obligation of \ref{inv0} to deduce a proper assignment to $a$:
%
%\begin{proof}{evt/INV/inv0}
%	\begin{calculation}
%		(n')^3
%	\hint{=}{ \ref{a0} }
%		(n+1)^3
%	\hint{=}{ arithmetic }
%		n^3 + 3 \cdot n^2 + 3 \cdot n + 1
%	\hint{=}{ \ref{inv0} }
%		a + 3 \cdot n^2 + 3 \cdot n + 1
%	\hint{=}{ we add a variable $b$: \ref{inv1}, see below }
%		a + b
%	\hint{=}{ \ref{a1} }
%		a'
%	\end{calculation}
%\end{proof}
%
%\begin{variable}
%	b : \Int
%\end{variable}
%
%\begin{invariant}{inv1}
%	b = 3 \cdot n^2 + 3 \cdot n + 1
%\end{invariant}
%%%%\begin{invariant}{inv2}
%%%%	b ~=~ 3 \cdot n^2 + 3 \cdot n + 1
%%%%\end{invariant}
%%%
%\begin{evassignment}{evt}{a1}
%	a' = a + b
%\end{evassignment}
%
%We now have a new invariant to preserve. It is easy to see how to establish it initially:
%
%\begin{initialization}{in1}
%	b = 1
%\end{initialization}
%
%\begin{itemize}
%\item label initialization predicates
%\item test case: train system
%\item refinement
%
%	add refinement environments for: changing schedules, transforming progress properties
%\item po labels: check that proofs match a po
%\item translate the tags in proof obligations
%	
%	create toc entries with the proof environment, in unitb.sty
%\item spacing commands
%\end{itemize}
%
%\section{Todo:}
%\begin{itemize}
%\item testing the input validation, error messages
%\item proof structures (proof by cases, etc)
%\item types
%\item invariant theorems
%\item error checking
%\item better LaTeX formatting
%\item lazy proof checking
%\item \sout{continuous checking}
%\item \sout{\emph{bug}: last of empty list of calculation steps}
%\item \sout{error report: report line number instead of step number when proof is incorrect}
%\item generate documentation
%\end{itemize}
%
%I'm now describing how I came up with the proof. It came to me in a dream and I forgot it in another dream.
%
%\begin{proof}{evt/INV/inv1}
%	\begin{calculation}
%		3 \cdot (n')^2 + 3 \cdot n' + 1
%	\hint{=}{ \ref{a0} }
%		3 \cdot (n+1)^2 + 3 \cdot (n+1) + 1
%	\hint{=}{ arithmetic }
%		3 \cdot (n^2+2\cdot n+1) + 3 \cdot (n+1) + 1
%	\hint{=}{ arithmetic }
%		3 \cdot n^2+6\cdot n+3 + 3 \cdot n + 3 + 1
%	\hint{=}{ \ref{inv1} }
%		b+6\cdot n+3+3
%	\hint{=}{ \ref{inv2}, see below }
%		b+c
%	\hint{=}{ \ref{a2}, see below }
%		b'
%	\end{calculation}
%\end{proof}
%
%\begin{variable}
%	c : \Int
%\end{variable}
%
%\begin{invariant}{inv2}
%	c = 6 \cdot n + 6
%\end{invariant}
%
%\begin{evassignment}{evt}{a2}
%	b' = b + c
%\end{evassignment}
%
%Invariant \ref{inv2} is easy to satisfy initially:
%
%\begin{initialization}{in2}
%	c = 6
%\end{initialization}
%
%\begin{proof}{evt/INV/inv2}
%	\begin{calculation}
%		6 \cdot (n') + 6
%	\hint{=}{ \ref{a0} }
%		6 \cdot (n+1) + 6
%	\hint{=}{ arithmetic }
%		6 \cdot n + 6 + 6
%	\hint{=}{ \ref{inv2} }
%		c + 6
%	\hint{=}{ \ref{a3} }
%		c'
%	\end{calculation}
%\end{proof}
%
%\begin{evassignment}{evt}{a3}
%	c' = c + 6
%\end{evassignment}
%
\end{machine}

\end{document}