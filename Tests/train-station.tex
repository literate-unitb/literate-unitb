
\documentclass[12pt]{amsart}
\usepackage{geometry} % see geometry.pdf on how to lay out the page. There's lots.
\usepackage{bsymb}
\usepackage{unitb}
\usepackage{calculation}
\usepackage{ulem}
\normalem
\geometry{a4paper} % or letter or a5paper or ... etc
% \geometry{landscape} % rotated page geometry

% See the ``Article customise'' template for some common customisations

\title{}
\author{}
\date{} % delete this line to display the current date

%%% BEGIN DOCUMENT
\begin{document}

\maketitle
\tableofcontents

%\section{}
%\subsection{}
\newcommand{\TRAIN}{\text{TRAIN}}\newcommand{\BLK}{\text{BLK}}\newcommand{\LOC}{\text{LOC}}
\begin{machine}{train0}

\newset{\TRAIN}
\newset{\BLK}
%\newset{\LOC}
\newset{\LOC}
%\end{set}

%
%\hide{
	\begin{align*}
\variable{		in : \set[\TRAIN]}
	\end{align*}
%}
%

\newevent{enter}{enter}

\newevent{leave}{leave}

\begin{align*}
\transientB{leave}{tr0}
{ \index{t}{t' = t} }
{	t \in in}
\end{align*}
\begin{align*}
\dummy{	t,t_0,t_1,t_2,t_3 : \TRAIN}
\end{align*}

\begin{align*}
\dummy{	p,q : \BLK}
\end{align*}


\begin{align*}
\variable{	loc : \TRAIN \pfun \BLK}
\end{align*}

\begin{align*}
\invariant{inv2}
%	loc \in in \fun \BLK
{	\dom . loc = in }
\end{align*}

\begin{align*}
\evassignment{enter}{a1}
{	in' = in \bunion \{ t \} }
\end{align*}

\begin{proof}{enter/INV/inv2}
	\begin{calculation}
		in'
	\hint{=}{ \ref{a1} }
		in \bunion \{ t \}
	\hint{=}{ \ref{inv2} }
		\dom.loc \bunion \{ t \}
	\hint{=}{ function calculus }
		\dom.loc \bunion \dom.(t \fun ent)
	\hint{=}{ $\dom$ over $|$ }
		\dom.(loc  \1 |   t \fun ent)
	\hint{=}{ \ref{a2} }
		\dom. loc' 
	\end{calculation}
\end{proof}

\begin{align*}
\evassignment{enter}{a2}
{	loc' = loc  \1| (t \fun ent)	}
\end{align*}

\begin{proof}{leave/INV/inv2}
	\begin{calculation}
		in'
	\hint{=}{ \ref{a0} }
		in \setminus \{ t \}
	\hint{=}{ \ref{inv2} }
		\dom.loc \setminus \{ t \}
	\hint{=}{ domain of domain subtraction }
		\dom.(\{ t \} \domsub loc)
	\hint{=}{ \ref{a3} } 
		\dom. loc' 
	\end{calculation}
\end{proof}

\begin{align*}
\evassignment{leave}{a3}
{	loc' = \{ t \} \domsub loc 	}
\end{align*}

\begin{align*}
\cschedule{leave}{c0}
{	t \in in}
\end{align*}

\begin{align*}
\initialization{in0}
{	in = \emptyset	}
\end{align*}
\begin{align*}
\initialization{in1}
{	loc = \emptyfun	}
\end{align*}

\with{sets}


\with{functions}
%\with{functions}

\begin{align*}
\indices{leave}{	t: \TRAIN}
\end{align*}

\begin{align*}
\indices{enter}{	t: \TRAIN}
\end{align*}

\begin{align*}
\evassignment{leave}{a0}
%	in' = in | (t \rightarrow \false)
{	in' = in \setminus \{ t \}	}
%	in'.t = \false 
\end{align*}

\begin{align*}
\constant{	ent,ext : \BLK}
\end{align*}
\begin{align*}
\constant{	PLF : \set [\BLK]}
\end{align*}

\begin{align*}
\assumption{axm0}
{	\BLK &= \{ ent, ext \} \bunion PLF	} \\
%\safety{s0}{ \neg t \in in & }{ t \in in \land loc.t = ent } \\
%\safety{s1}{ t \in in \land loc.t = ent & }{ t \in in \land loc.t \in PLF } \\
\constraint{co0}
{	\neg t \in in \land t \in in' & \2\implies  loc'.t = ent } % \lor \neg t \in in'
\end{align*}
%
\begin{proof}{leave/CO/co0}
	\begin{free:var}{t}{t_0}
	
	\begin{align}
	\assert{hyp6}{ \neg t = t_0 } \\
	\assume{hyp7}{ \neg t_0 \in in } \\
	\goal{ \neg t_0 \in in' }
	\end{align}
%	\begin{assume}{hyp0}{ \neg t_0 \in in \land  t_0 \in in' }{ loc'.t_0 = ent }
	\begin{calculation}
		t_0 \in in'
	\hint{=}{ \ref{a0} }
		t_0 \in in \setminus \{ t \}
	\hint{=}{ $\in$ over $\setminus$ }
		t_0 \in in \1\land \neg t_0 \in \{ t \} 
	\hint{=}{ \eqref{hyp7} }
		\false \1\land \neg t_0 \in \{ t \} 
	\hint{=}{ zero of $\land$ }
		\false
%%		loc'.t_0 = ent
%%	\hint{\follows}{ }
%%		\false
%%	\hint{=}{ \eqref{hyp0} \ref{a3} }
%%		\true
%		\neg t_0 \in in \land t_0 \in in'
%	\hint{=}{ \ref{grd0} }
%		\neg t_0 = t \land \neg t_0 \in in \land t_0 \in in' 
%	\hint{=}{ \ref{a0} }
%		\neg t_0 = t \land \neg t_0 \in in \land t_0 \in in \bunion \{ t \} 
%	\hint{=}{ contradiction }
%		\false
%	\hint{\implies}{ $\false \implies p$ }
%		loc'.t_0 = ent
	\end{calculation}
	\begin{subproof}{hyp6} 
	\begin{calculation}
		t = t_0
	\hint{\implies}{ Leibniz }
		t \in in \2\equiv t_0 \in in
	\hint{\implies}{ reflexivity }
		t \in in \2\implies t_0 \in in
	\hint{=}{ \eqref{grd0}; \eqref{hyp7} }
		\false
	\end{calculation}
	\end{subproof}
%	\end{assume}
	\end{free:var}
\end{proof}
%
\begin{align*}
\constraint{co1}
	{ & t \in in \1\land loc.t = ent  \1\land \neg loc.t \in PLF  \\
\3\implies & t \in in' \2\land (loc'.t \in PLF \2\lor loc'.t = ent) }
\end{align*}

\begin{proof}{enter/CO/co1}
	\begin{free:var}{t}{t_0}
	
	\noindent \textbf{assume:}
	\begin{align}
	\assume{hyp2}{		t_0 \in in	} \\
	\assume{hyp3}{		loc.t_0 = ent  }
	\end{align}
	
	\noindent \textbf{prove}
	\begin{align}
%	\goal{ \neg loc.t_0 \in PLF 
%			\implies t_0 \in in' 
%			\land (loc'.t_0 \in PLF \lor loc'.t_0 = ent)}
	\goal{	t_0 \in in' 
			\1\land loc'.t_0 = ent}
	\end{align}
%	\begin{assume}{hyp2}{t_0 \in in}
%		{  loc.t_0 = ent \land \neg loc.t_0 \in PLF 
%			\implies t_0 \in in' \land (loc'.t_0 \in PLF \lor loc'.t_0 = ent) }
%	\begin{assume}{hyp3}{ loc.t_0 = ent  }
%		{ \neg loc.t_0 \in PLF 
%			\implies t_0 \in in' 
%			\land (loc'.t_0 \in PLF \lor loc'.t_0 = ent)}
	\begin{by:parts}
	
		\begin{part:a}{ t_0 \in in' }
			% TODO: new assumption PO should have distinct names (maybe line numbers?)
		\begin{calculation}
			t_0 \in in'
		\hint{=}{ \ref{a1} }
			t_0 \in in \bunion \{ t \}
		\hint{=}{ \eqref{hyp2} }
			\true
		\end{calculation}

		\end{part:a}
		\begin{part:a}{ loc'.t_0 = ent }
		
		\begin{by:cases}
		
		\begin{case}{hyp4}{ t_0 = t}
		\begin{calculation}
			loc'.t_0 = ent
		\hint{=}{ \ref{a2} }
			(loc | t \fun ent).t_0 = ent
		\hint{=}{ \eqref{hyp4} }
			\true
		\end{calculation}

		\end{case}

		\begin{case}{hyp5}{ \neg t_0 = t}
		\begin{align}
		\assert{hyp6}{ t_0 \in \dom.loc \setminus \dom.(t \fun ent) }
		\end{align}
		\begin{calculation}
			loc'.t_0 = ent
		\hint{=}{ \ref{a2} }
			(loc | t \fun ent).t_0 = ent
		\hint{=}{ \eqref{hyp5} \eqref{hyp6} }
			loc.t_0 = ent
		\hint{=}{ \eqref{hyp3} }
			\true
		\end{calculation}
		\begin{subproof}{hyp6}
		\easy
%		\begin{calculation}
%			\true
%		\hint{=}{ }
%			\true
%		\end{calculation}
		\end{subproof}
		\end{case}
		
		\end{by:cases}
		
		\end{part:a}		
	\end{by:parts}
%	\end{assume}
%	\end{assume}
	\end{free:var}
\end{proof}

\discharge{s0}{co0}
\discharge{s1}{co1}

% test the error message for
%\begin{guard}{co0}
%
%\end{guard}

\begin{align}
\evguard{leave}{grd0}
{	loc.t = ext \land t \in in	} \\
%\end{align}
%
%\begin{align}
\evguard{enter}{grd1}
{	\neg t \in in	}
\end{align}

\begin{proof}{enter/CO/co0}
	% assume 
	% forall
	% case
	% split
	% TODO: add a case distinction that isn't complete
	\begin{free:var}{t}{t_0}

	\begin{by:cases}

	\begin{case}{h0}{t_0 = t}
	\begin{calculation}
		\neg t_0 \in in \land t_0 \in in' \implies  loc'.t_0 = ent 
	\hint{=}{ \eqref{h0} and \ref{a1}  }
		\neg t \in in \implies  loc'.t = ent 
	\hint{=}{ \ref{a2} }
		\true
	\end{calculation}

	\end{case}

	\begin{case}{h1}{ \neg t_0 = t }
%	\easy
	\begin{calculation}
		\neg t_0 \in in \land t_0 \in in' 
	\hint{=}{ \ref{a1}  }
		\neg t_0 \in in \land  t_0 \in in \bunion \{ t \}
	\hint{=}{ \eqref{h1} }
		\neg t_0 \in in \land  t_0 \in in
	\hint{=}{ contradiction }
		\false
	\hint{\implies}{ pred. calc. }
		loc'.t_0 = ent 
	\end{calculation}

	\end{case}

	\end{by:cases}
	\end{free:var}
\end{proof}

\begin{proof}{leave/CO/co1}
	\begin{free:var}{t}{t_0}
	\begin{calculation}
		t_0 \in in' \land (loc'.t_0 \in PLF \lor loc'.t_0 = ent)
	\hint{=}{  \ref{a0}  and \ref{a3} } % step 1
		t_0 \in in \setminus \{ t \} 
		\land ( (\{ t \} \domsub loc).t_0 \in PLF \lor (\{ t \} \domsub loc).t_0 = ent)
	\hint{=}{ }	% step 2
			( 		t_0 \in in \setminus \{ t \} 
			 \land  ( \{ t \} \domsub loc).t_0 \in PLF )
		\lor ( 		t_0 \in (in \setminus \{ t \} )
			 \land (\{ t \} \domsub loc).t_0 = ent)
	\hint{=}{ \ref{inv2} }	% step 3
			t_0 \in in \setminus \{ t \} 
		\land ( loc.t_0 \in PLF \lor  loc.t_0 = ent)
	\hint{=}{ } % step 4
		t_0 \in in \land \neg t_0  = t 
		\land ( loc.t_0 \in PLF \lor  loc.t_0 = ent)
	\hint{=}{ \ref{asm2}, \ref{asm4}, \ref{asm3} } % step 5
		t_0 \in in \land \neg t_0  = t \land \neg loc.t_0 = ext
	\hint{=}{  \ref{grd0} } % step 6
	 	t_0 \in in \land \neg loc.t_0 = ext \land loc.t = ext
	\hint{\follows}{  \ref{grd0}, \ref{asm2} } % step 7
%		\true
	 	t_0 \in in \land loc.t_0 = ent 
	\hint{=}{ \ref{asm2}  } % step 8
	 	t_0 \in in \land loc.t_0 = ent  \land \neg loc.t_0 \in PLF 
	\end{calculation}
	\end{free:var}
\end{proof}

\begin{align*}
\invariant{inv1}
{	\qforall{t}{t \in in}{loc.t \in \BLK}	}
\end{align*}

\begin{align*}
\assumption{asm2}
{	\neg ent = ext \land \neg ent \in PLF \land \neg ext \in PLF	}
\end{align*}

\begin{align*}
\assumption{asm3}
{	\qforall{p}{}{ \neg p = ext \equiv p \in \{ent\} \bunion PLF }	}
\end{align*}

\begin{align*}
\assumption{asm4}
{	\qforall{p}{}{ \neg p = ent \equiv p \in \{ext\} \bunion PLF }	}
\end{align*}

\begin{align*}
\assumption{asm5}
{	\qforall{p}{}{ p = ent \lor p = ext \equiv \neg p \in PLF }	}
\end{align*}

\begin{proof}{INIT/INV/inv2}
	\begin{calculation}
		\dom. loc
	\hint{=}{ \ref{in1} }
		 \dom . \oftype{ \emptyfun }{ \pfun[\TRAIN, \BLK] }  
	\hint{=}{ empty functions }
		\oftype{ \emptyset }{\set[\TRAIN]}
	\hint{=}{ \ref{in0} }
		in
	\end{calculation}
\end{proof}

\weakento{leave}{default}{c0}

\end{machine}

\end{document}