
\documentclass[12pt]{amsart}
\usepackage{geometry} % see geometry.pdf on how to lay out the page. There's lots.
\usepackage{unitb}
\usepackage{calculation}
\usepackage{ulem}
\normalem
\geometry{a4paper} % or letter or a5paper or ... etc
% \geometry{landscape} % rotated page geometry

% See the ``Article customise'' template for some common customisations

\title{}
\author{}
\date{} % delete this line to display the current date

%%% BEGIN DOCUMENT
\begin{document}

\maketitle
\tableofcontents

%\section{}
%\subsection{}

\begin{machine}{m0}

\hide{
	\begin{variable}
		n,a : \Int
	\end{variable}
}

\begin{invariant}{inv0}
	a = n^3
\end{invariant}
%
%%\begin{proof}{INIT/INV/inv0}
%%\begin{calculation}
%%	
%%\end{calculation}
%%\end{proof}

\newevent{evt}

%
\begin{initialization}{init0}
	n = 0 \land a = 0
\end{initialization}
%
%
\begin{evassignment}{evt}{a0}
	n' = n + 1
\end{evassignment}

We use the proof obligation of \ref{inv0} to deduce a proper assignment to $a$:

\begin{proof}{evt/INV/inv0}
	\begin{calculation}
		(n')^3
	\hint{=}{ \ref{a0} }
		(n+1)^3
	\hint{=}{ arithmetic }
		n^3 + 3 \cdot n^2 + 3 \cdot n + 1
	\hint{=}{ \ref{inv0} }
		a + 3 \cdot n^2 + 3 \cdot n + 1
	\hint{=}{ we add a variable $b$: \ref{inv1}, see below }
		a + b
	\hint{=}{ \ref{a1} }
		a'
	\end{calculation}
\end{proof}

\begin{variable}
	b : \Int
\end{variable}

\begin{invariant}{inv1}
	b = 3 \cdot n^2 + 3 \cdot n + 1
\end{invariant}
%%%%\begin{invariant}{inv2}
%%%%	b ~=~ 3 \cdot n^2 + 3 \cdot n + 1
%%%%\end{invariant}
%%%
\begin{evassignment}{evt}{a1}
	a' = a + b
\end{evassignment}

We now have a new invariant to preserve. It is easy to see how to establish it initially:
%
\begin{initialization}{in1}
	b = 1
\end{initialization}

\begin{itemize}
\item label initialization predicates
\item dummy section
\item test case: train system
\item refinement

	add refinement environments for: changing schedules, transforming progress properties
\item po labels: check that proofs match a po
\item translate the tags in proof obligations
	
	create toc entries with the proof environment, in unitb.sty
\item spacing commands
\end{itemize}

\section{Todo:}
\begin{itemize}
\item spacing in the middle of event and set declarations
\item testing the input validation, error messages
\item proof structures (proof by cases, etc)
\item types
\item invariant theorems
\item error checking
\item better LaTeX formatting
\item lazy proof checking
\item \sout{continuous checking}
\item \sout{\emph{bug}: last of empty list of calculation steps}
\item \sout{error report: report line number instead of step number when proof is incorrect}
\item generate documentation
\item handle the ``\textbackslash input'' commands in latex and use it to produce a project hierarchy
\end{itemize}

I'm now describing how I came up with the proof. It came to me in a dream and I forgot it in another dream.

\begin{proof}{evt/INV/inv1}
	\begin{calculation}
		3 \cdot (n')^2 + 3 \cdot n' + 1
	\hint{=}{ \ref{a0} }
		3 \cdot (n+1)^2 + 3 \cdot (n+1) + 1
	\hint{=}{ arithmetic }
		3 \cdot (n^2+2\cdot n+1) + 3 \cdot (n+1) + 1
	\hint{=}{ arithmetic }
		3 \cdot n^2+6\cdot n+3 + 3 \cdot n + 3 + 1
	\hint{=}{ \ref{inv1} }
		b+6\cdot n+3+3
	\hint{=}{ \ref{inv2}, see below }
		b+c
	\hint{=}{ \ref{a2}, see below }
		b'
	\end{calculation}
\end{proof}

\begin{variable}
	c : \Int
\end{variable}

\begin{invariant}{inv2}
	c = 6 \cdot n + 6
\end{invariant}

\begin{evassignment}{evt}{a2}
	b' = b + c
\end{evassignment}

Invariant \ref{inv2} is easy to satisfy initially:

\begin{initialization}{in2}
	c = 6
\end{initialization}

\begin{proof}{evt/INV/inv2}
	\begin{calculation}
		6 \cdot (n') + 6
	\hint{=}{ \ref{a0} }
		6 \cdot (n+1) + 6
	\hint{=}{ arithmetic }
		6 \cdot n + 6 + 6
	\hint{=}{ \ref{inv2} }
		c + 6
	\hint{=}{ \ref{a3} }
		c'
	\end{calculation}
\end{proof}
%
\begin{evassignment}{evt}{a3}
	c' = c + 6
\end{evassignment}

\end{machine}

\end{document}