\documentclass[12pt]{amsart}
\usepackage[margin=0.5in]{geometry} 
  % see geometry.pdf on how to lay out the page. There's lots.
\usepackage{bsymb}
\usepackage{calculation}
\usepackage{ulem}
\usepackage{hyperref}
\usepackage{unitb}

\newcommand{\REQ}{\text{REQ}}

\begin{document}

\section{Strategy}

\begin{itemize}
  \item 
\end{itemize}
\section{Model m0 --- Requests and non-deterministic handling}
  \begin{block}
  \item   \textbf{machine} m0
  \item   \textbf{variables}
  \begin{block}
    \item   $in$
  \end{block}
  \item   \textbf{progress}
\begin{block}
\item[ \eqref{m0:prog0} ]$t \in in  \quad \mapsto\quad \neg t \in in $ %
\end{block}

  \item   \textbf{transient}
\begin{block}
\item[ \eqref{m0:tr0} ]$t \in in  \qquad \text{(\ref{m0:leave}: [t := t' | t' = t])}$ %
\end{block}

  \item   \textbf{events}
  \begin{block}
    \item   \noindent \ref{m0:enter} [t] \textbf{event}
\begin{block}
\item \textbf{during}
\begin{block}
\item[ \eqref{m0:enterdefault} ]$\false$ %
\end{block}
\item \textbf{begin}
\begin{block}
\item[ \eqref{m0:entera1} ]$in' = in \bunion \{ t \} $ %
\end{block}
\item \textbf{end} \\
\end{block}

    \item   \noindent \ref{m0:leave} [t] \textbf{event}
\begin{block}
  \item   \textbf{during}
  \begin{block}
  \item[ \eqref{m0:leavelv:c0} ]$t \in in $ %
  \end{block}
  \item   \textbf{begin}
  \begin{block}
  \item[ \eqref{m0:leavelv:a0} ]$in \bcmsuch in' = in \setminus \{ t \} $ %
  \end{block}
  \item   \textbf{end} \\
\end{block}

  \end{block}
  \item   \textbf{end} \\
\end{block}

\begin{machine}{m0}
  \with{sets}
  \newset{\REQ} 
  \newevent{add}{add}
  \newevent{handle}{handle}
  \[ \variable{ req : \set [\REQ] } \]
  \begin{description}
    \comment{req}{set of pending requests}
  \end{description}
  \[ \param{add}{ r : \REQ } \]
  \[ \param{handle}{ r : \REQ } \]
  \begin{align*}
      \initialization{m0:init0}{ req = \emptyset } \\
      \evguard{add}{m0:guard}{ \neg r \in req } \\
      \evbcmeq{add}{m0:act0}{req}{ req \bunion \{ r \} } \\
      \cschedule{handle}{m0:sch0}{ \neg req = \emptyset } \\
      \evguard{handle}{m0:grd0}{ r \in req } \\
      \evbcmeq{handle}{m0:act0}{req}{ req \setminus \{ r \} } 
  \end{align*}

\noindent
\end{machine}
\section{Model m1 --- Version numbers and individual fairness}
  \begin{block}
  \item   \textbf{machine} m1
  \item   \textbf{variables}
  \begin{block}
    \item   $in$
    \item   $loc$
  \end{block}
  \item   %!TEX root=../main8.tex
\textbf{invariants}
\begin{block}
\item[ \eqref{m1:inv0} ]{$qe \in \intervalR{p}{q} \tfun \G $} %
\item[ \eqref{m1:inv1} ]{$p \le q $} %
\end{block}

  \item   \textbf{progress}
\begin{block}
\item[ \eqref{m1:prog0} ]$r \in \dom.pshL \1\land pshL.r = x  \quad \mapsto\quad p < q \land qe.p = x \1\land \neg r \in \dom.pshL $ %
\item[ \eqref{m1:prog1} ]$r \in \dom.pshR \1\land pshR.r = x  \quad \mapsto\quad p < q \1\land qe.(q-1) = x \1\land \neg r \in \dom.pshR $ %
\item[ \eqref{m1:prog2} ]$r \in popR  \quad \mapsto\quad \neg r \in popR $ %
\item[ \eqref{m1:prog3} ]$r \in popL  \quad \mapsto\quad \neg r \in popL  $ %
\end{block}

  \item   \textbf{safety}
\begin{block}
\item[ \eqref{m1:saf0} ]$\neg t \in in  \textbf{\quad unless \quad} t \in in \land loc.t = ent $ %
\item[ \eqref{m1:saf1} ]$t \in in \land loc.t = ent  \textbf{\quad unless \quad} t \in in \land loc.t \in plf $ %
\item[ \eqref{m1:saf2} ]$t \in in \land loc.t \in plf  \textbf{\quad unless \quad} t \in in \land loc.t = ext $ %
\item[ \eqref{m1:saf3} ]$t \in in \land loc.t = ext  \textbf{\quad unless \quad} \neg t \in in $ %
\end{block}

  \item   %!TEX root=../puzzle.tex

  \item   \textbf{events}
  \begin{block}
    \item   \noindent \ref{m0:enter} [t] \textbf{event}
\begin{block}
  \item   \textbf{when}
  \begin{block}
  \item[ \eqref{m0:enterent:grd1} ]$\neg t \in in $ %
  \end{block}
  \item   \textbf{begin}
  \begin{block}
  \item[ \eqref{m0:entera1} ]$in \bcmsuch in' = in \bunion \{ t \} $ %
  \item[ \eqref{m0:entera3} ]$loc \bcmsuch loc' = loc \1| t \fun ent $ %
  \end{block}
  \item   \textbf{end} \\
\end{block}

    \item   \noindent \ref{m0:leave} [t] \textbf{event}
\begin{block}
  \item   \textbf{during}
  \begin{block}
  \item[ \eqref{m0:leavelv:c0} ]$t \in in $ %
  \item[ \eqref{m0:leavelv:c1} ]$loc.t = ext $ %
  \end{block}
  \item   \textbf{when}
  \begin{block}
  \item[ \eqref{m0:leavelv:grd0} ]$t \in in $ %
  \item[ \eqref{m0:leavelv:grd1} ]$loc.t = ext $ %
  \end{block}
  \item   \textbf{begin}
  \begin{block}
  \item[ \eqref{m0:leavelv:a0} ]$in \bcmsuch in' = in \setminus \{ t \} $ %
  \item[ \eqref{m0:leavelv:a2} ]$loc \bcmsuch loc' = \{ t \} \domsub loc $ %
  \end{block}
  \item   \textbf{end} \\
\end{block}

    \item   \noindent \ref{m1:movein} [t] \textbf{event}
\begin{block}
  \item   \textbf{during}
  \begin{block}
  \item[ \eqref{m1:moveinmi:c1} ]$t \in in $ %
  \item[ \eqref{m1:moveinmi:c2} ]$loc.t = ent $ %
  \end{block}
  \item   \textbf{any} b
  \item   \textbf{when}
  \begin{block}
  \item[ \eqref{m1:moveinmi:g1} ]$t \in in $ %
  \item[ \eqref{m1:moveinmi:grd0} ]$loc.t = ent $ %
  \item[ \eqref{m1:moveinmi:grd7} ]$b \in plf $ %
  \end{block}
  \item   \textbf{begin}
  \begin{block}
  \item[ \eqref{m1:moveinmi:a2} ]$loc \bcmsuch loc' = loc \1| t \fun b $ %
  \end{block}
  \item   \textbf{end} \\
\end{block}

    \item   \noindent \ref{m1:moveout} [t] \textbf{event}
\begin{block}
\item \textbf{during}
\begin{block}
\item[ \eqref{m1:moveoutc1} ]$t \in in  %
		\1\land loc.t \in plf $ %
\end{block}
\item \textbf{when}
\begin{block}
\item[ \eqref{m1:moveoutmo:g1} ]$t \in in $ %
\item[ \eqref{m1:moveoutmo:g2} ]$loc.t \in plf $ %
\end{block}
\item \textbf{begin}
\begin{block}
\item[ \eqref{m1:moveoutSKIP:in} ]$in' = in$ %
\item[ \eqref{m1:moveouta2} ]$loc' = loc \1 | t \fun ext $ %
\end{block}
\item \textbf{end} \\
\end{block}

  \end{block}
  \item   \textbf{end} \\
\end{block}

\begin{machine}{m1}
    \refines{m0}
  \[ \variable{ ver : \Int } \]
  \begin{description}
    \comment{ver}{ serial number of the current data structure state }
  \end{description}
  \[ \indices{handle}{ v : \Int } \]
  \promote{handle}{r}
  \removecoarse{handle}{m0:sch0}
  \removeguard{handle}{m0:grd0}
  % \removeact{handle}{m0:act0}
  \[\witness{handle}{r}{r \in req}\]
  \[\witness{handle}{v}{v = ver}\]
  \begin{align}
      \initialization{m1:init0}{ ver = 0 } \\
      \cschedule{handle}{m1:sch0}{ r \in req } \\
      \cschedule{handle}{m1:sch1}{ v = ver } \\
      % \evguard{handle}{m1:grd0}{ r = r0 } \\
      \evbcmeq{handle}{m1:act0}{ver}{ ver + 1 } 
  \end{align}
\end{machine}
\section{Model m2 --- Specialized events}
  %!TEX root=../main9.tex
\begin{block}
  \item   \textbf{machine} m2
  \item   \textbf{refines} m0
  \item   \textbf{variables}
  \begin{block}
    \item   $req$
    \item   $req0$
    \item   $reqA$
    \item   $reqB$
  \end{block}
  \item   %!TEX root=../main9.tex
\textbf{invariants}
\begin{block}
\item[ \eqref{m1:inv0} ]{$reqA \bunion reqB = req$} %
\item[ \eqref{m1:inv1} ]{$reqA \binter reqB = req$} %
\end{block}

  \item   \textbf{events}
  \begin{block}
    \item   %!TEX root=../main9.tex
\noindent \ref{handle}  \textbf{event}
\begin{block}
  \item   \textbf{during}
  \begin{block}
  \item[ \eqref{handlem0:sch0} ]{$\neg req = \emptyset$} %
  \end{block}
  \item   \textbf{any} r
  \item   \textbf{when}
  \begin{block}
  \item[ \eqref{handlegrd0} ]{$r \in req$} %
  \end{block}
  \item   \textbf{begin}
  \begin{block}
  \item[ \eqref{handleact0} ]{$req \bcmeq req \setminus \{ r \}$} %
  \item[ \eqref{handleact1} ]{$req0 \bcmeq req$} %
  \end{block}
  \item   \textbf{end} \\
\end{block}

    \item   %!TEX root=../main9.tex
\noindent \ref{req}  \textbf{event}
\begin{block}
  \item   \textbf{during}
  \begin{block}
  \item[ (\ref{req}/default) ]{$\false$} %
  \end{block}
  \item   \textbf{any} r
  \item   \textbf{when}
  \begin{block}
  \item[ \eqref{reqgrd0} ]{$\neg r \in req$} %
  \end{block}
  \item   \textbf{begin}
  \begin{block}
  \item[ \eqref{reqact0} ]{$req \bcmeq req \bunion \{ r \}$} %
  \item[ \eqref{reqact1} ]{$req0 \bcmeq req$} %
  \end{block}
  \item   \textbf{end} \\
\end{block}

  \end{block}
  \item   \textbf{end} \\
\end{block}

\begin{machine}{m2}
  \refines{m1}
  \[ \variable{pshL,pshR,popR,popL : \set [\REQ]} \]
  \begin{description}
    \comment{pshL}{replaces $req$. Set of push\_left requests }
    \comment{pshR}{replaces $req$. Set of push\_right requests }
    \comment{popL}{replaces $req$. Set of pop\_left requests }
    \comment{popR}{replaces $req$. Set of pop\_right requests }
  \end{description}
  \begin{align}
    \initialization{m2:init0}{pshL = \emptyset} \\
    \initialization{m2:init1}{popL = \emptyset} \\
    \initialization{m2:init2}{pshR = \emptyset} \\
    \initialization{m2:init3}{popR = \emptyset} 
  \end{align}
  \subsection{Specialize \emph{handle}}
  % \begin{align*}
    \refiningevent{handle}{handle:popL}{handle\_popL}
    \refiningevent{handle}{handle:popR}{handle\_popR} 
    \refiningevent{handle}{handle:pushL}{handle\_pushL}
    \refiningevent{handle}{handle:pushR}{handle\_pushR}
  % \end{align*}
  \splitevent{handle}{handle:popL,handle:popR,handle:pushR,handle:pushL}
  \begin{align*}
    \definition{Req}{ pshL \bunion 
        pshR \bunion 
        popL \bunion 
        popR } 
  \end{align*}
  \begin{align}
    \invariant{m2:inv0}{ Req = req } \\
    \invariant{m2:inv1}{ pshL \binter pshR = \emptyset } \\
    \invariant{m2:inv2}{ pshL \binter popL = \emptyset } \\
    \invariant{m2:inv3}{ pshL \binter popR = \emptyset } \\
    \invariant{m2:inv4}{ pshR \binter popL = \emptyset } \\
    \invariant{m2:inv5}{ pshR \binter popR = \emptyset } \\
    \invariant{m2:inv6}{ popL \binter popR = \emptyset } \\
    \cschedule{handle:pushL}{m2:sch0}{ r \in pshL } \\
    \evbcmeq{handle:pushL}{m2:act0}{pshL}{pshL \setminus \{ r \}} \\
    \cschedule{handle:pushR}{m2:sch0}{ r \in pshR } \\
    \evbcmeq{handle:pushR}{m2:act0}{pshR}{pshR \setminus \{ r \}} \\
    \cschedule{handle:popL}{m2:sch0}{ r \in popL } \\
    \evbcmeq{handle:popL}{m2:act0}{popL}{popL \setminus \{ r \}} \\
    \cschedule{handle:popR}{m2:sch0}{ r \in popR } \\
    \evbcmeq{handle:popR}{m2:act0}{popR}{popR \setminus \{ r \}} 
    % \invariant{m2:inv1}{ popL \subseteq req } \\
    % \invariant{m2:inv2}{ pshR \subseteq req } \\
    % \invariant{m2:inv3}{ popR \subseteq req } 
  \end{align}
  \[ \variable{ ppd : \set [\REQ] } \]
  \begin{align}
    \initialization{m2:init4}{ ppd = \emptyset } \\
    \evbcmeq{handle:popR}{m2:act1}{ppd}{ ppd \bunion \{r\} } \\
    \evbcmeq{handle:popL}{m2:act1}{ppd}{ ppd \bunion \{r\} }
  \end{align}
  \newevent{return}{return}
  \[ \indices{return}{ r : \REQ } \]
  \begin{align}
    \cschedule{return}{m2:sch0}{ r \in ppd } \\
    \evbcmeq{return}{m2:act0}{ppd}{ ppd \setminus \{ r \} }
  \end{align}
\subsection{Specialize \emph{add}}
    \refiningevent{add}{add:popL}{add\_popL} 
    \refiningevent{add}{add:popR}{add\_popR}
    \refiningevent{add}{add:pushL}{add\_pushL}
    \refiningevent{add}{add:pushR}{add\_pushR}
  \hide{\splitevent{add}{add:popL,add:popR,add:pushR,add:pushL}}
  \begin{align}
    \evbcmeq{add:popL}{m2:act0}{popL}{popL \bunion \{r\}} \\
    \evbcmeq{add:pushL}{m2:act1}{pshL}{pshL \bunion \{r\}} \\
    \evbcmeq{add:popR}{m2:act2}{popR}{popR \bunion \{r\}} \\
    \evbcmeq{add:pushR}{m2:act3}{pshR}{pshR \bunion \{r\}} 
  \end{align}
\subsection{Data refinement}
  \removevar{req}
  \removeinit{m0:init0}
  \[\initwitness{req}{req = \emptyset}\]
  \removeact{handle:popR}{m0:act0}
  \removeact{handle:pushR}{m0:act0}
  \removeact{handle:popL}{m0:act0}
  \removeact{handle:pushL}{m0:act0}
  \removecoarse{handle:popR}{m1:sch0}
  \removecoarse{handle:pushR}{m1:sch0}
  \removecoarse{handle:popL}{m1:sch0}
  \removecoarse{handle:pushL}{m1:sch0}
  \removeguard{add:popR}{m0:guard}
  \removeguard{add:pushR}{m0:guard}
  \removeguard{add:popL}{m0:guard}
  \removeguard{add:pushL}{m0:guard}
  \begin{align*}
  \evguard{add:popR}{m2:grd0}
    { \neg r \in Req } \\
  \evguard{add:pushR}{m2:grd0}
    { \neg r \in Req } \\
  \evguard{add:popL}{m2:grd0}
    { \neg r \in Req } \\
  \evguard{add:pushL}{m2:grd0}
    { \neg r \in Req } 
  \end{align*}
  \removeact{add:popR}{m0:act0}
  \removeact{add:pushR}{m0:act0}
  \removeact{add:popL}{m0:act0}
  \removeact{add:pushL}{m0:act0}
\end{machine}
  \newcommand{\OBJ}{\mathcal{O}bj}
\section{Model m3 --- The Contents}
  \begin{block}
  \item   \textbf{machine} m3
  \item   \textbf{variables}
  \begin{block}
    \item   $in$
    \item   $isgn$
    \item   $loc$
    \item   $osgn$
  \end{block}
  \item   %!TEX root=../main.tex
\textbf{invariants}
\begin{block}
\item[ \eqref{m3:inv0} ]{$\dom.delta = Pcs $} %
\item[ \eqref{m3:inv1} ]{$b \2\subseteq d \setminus \qset{p,q}{q \in delta.p}{q} $} %
\end{block}

  \item   \textbf{events}
  \begin{block}
    \item   \noindent \ref{m0:enter} [t] \textbf{event}
\begin{block}
  \item   \textbf{when}
  \begin{block}
  \item[ \eqref{m0:enterent:grd1} ]$\neg t \in in $ %
  \item[ \eqref{m0:enteret:g1} ]$\neg ent \in \ran.loc $ %
  \end{block}
  \item   \textbf{begin}
  \begin{block}
  \item[ \eqref{m0:entera1} ]$in \bcmsuch in' = in \bunion \{ t \} $ %
  \item[ \eqref{m0:entera3} ]$loc \bcmsuch loc' = loc \1| t \fun ent $ %
  \item[ \eqref{m0:enterm3:ent:act0} ]$isgn,osgn \bcmsuch isgn' = isgn$ %
  \item[ \eqref{m0:enterm3:ent:act1} ]$isgn,osgn \bcmsuch osgn' = osgn$ %
  \end{block}
  \item   \textbf{end} \\
\end{block}

    \item   \noindent \ref{m0:leave} [t] \textbf{event}
\begin{block}
  \item   \textbf{during}
  \begin{block}
  \item[ \eqref{m0:leavelv:c0} ]$t \in in $ %
  \item[ \eqref{m0:leavelv:c1} ]$loc.t = ext $ %
  \end{block}
  \item   \textbf{when}
  \begin{block}
  \item[ \eqref{m0:leavelv:grd0} ]$t \in in $ %
  \item[ \eqref{m0:leavelv:grd1} ]$loc.t = ext $ %
  \end{block}
  \item   \textbf{begin}
  \begin{block}
  \item[ \eqref{m0:leavelv:a0} ]$in \bcmsuch in' = in \setminus \{ t \} $ %
  \item[ \eqref{m0:leavelv:a2} ]$loc \bcmsuch loc' = \{ t \} \domsub loc $ %
  \item[ \eqref{m0:leavem3:ext:act0} ]$isgn,osgn \bcmsuch isgn' = isgn$ %
  \item[ \eqref{m0:leavem3:ext:act1} ]$isgn,osgn \bcmsuch osgn' = osgn$ %
  \end{block}
  \item   \textbf{end} \\
\end{block}

    \item   \noindent \ref{m1:movein} [t] \textbf{event}
\begin{block}
  \item   \textbf{during}
  \begin{block}
  \item[ \eqref{m1:moveinmi:c0} ]$\neg ~ plf \subseteq \ran.loc $ %
  \item[ \eqref{m1:moveinmi:c1} ]$t \in in $ %
  \item[ \eqref{m1:moveinmi:c2} ]$loc.t = ent $ %
  \end{block}
  \item   \textbf{any} b
  \item   \textbf{when}
  \begin{block}
  \item[ \eqref{m1:moveinmi:g0} ]$\neg b \in \ran.loc  	$ %
  \item[ \eqref{m1:moveinmi:g1} ]$t \in in $ %
  \item[ \eqref{m1:moveinmi:grd0} ]$loc.t = ent $ %
  \item[ \eqref{m1:moveinmi:grd7} ]$b \in plf $ %
  \end{block}
  \item   \textbf{begin}
  \begin{block}
  \item[ \eqref{m1:moveinm3:mi:act0} ]$isgn,osgn \bcmsuch isgn' = isgn$ %
  \item[ \eqref{m1:moveinm3:mi:act1} ]$isgn,osgn \bcmsuch osgn' = osgn$ %
  \item[ \eqref{m1:moveinmi:a2} ]$loc \bcmsuch loc' = loc \1| t \fun b $ %
  \end{block}
  \item   \textbf{end} \\
\end{block}

    \item   \noindent \ref{m1:moveout} [t] \textbf{event}
\begin{block}
\item \textbf{during}
\begin{block}
\item[ \eqref{m1:moveoutc1} ]$t \in in  %
		\1\land loc.t \in plf $ %
\item[ \eqref{m1:moveoutm3:mo:sch0} ]$loc.t \in osgn $ %
\end{block}
\item \textbf{when}
\begin{block}
\item[ \eqref{m1:moveoutm3:mo:grd0} ]$loc.t \in osgn $ %
\item[ \eqref{m1:moveoutmo:g1} ]$t \in in $ %
\item[ \eqref{m1:moveoutmo:g2} ]$loc.t \in plf $ %
\end{block}
\item \textbf{begin}
\begin{block}
\item[ \eqref{m1:moveoutSKIP:in} ]$in' = in$ %
\item[ \eqref{m1:moveouta2} ]$loc' = loc \1 | t \fun ext $ %
\item[ \eqref{m1:moveoutm3:mo:act0} ]$isgn' = isgn$ %
\item[ \eqref{m1:moveoutm3:mo:act1} ]$osgn'  \2 = osgn  %
	\setminus \{ loc.t \} $ %
\end{block}
\item \textbf{end} \\
\end{block}

  \end{block}
  \item   \textbf{end} \\
\end{block}

\begin{machine}{m3}
  \refines{m2}
  \with{functions}
  \with{intervals}

  \[ \newset{\OBJ}  \]
  \[ \variable{ p,q : \Int } \]
  \[ \variable{ qe : \Int \pfun \OBJ } \]
  \begin{align}
    \invariant{m3:inv0}{ qe \in \intervalR{p}{q} \tfun \OBJ } \\
    \invariant{m3:inv1}{ p \le q } \\
    \initialization{m3:init0}{ p = 0 \land q = 0} \\
    \initialization{m3:init1}{ qe = \emptyfun }
  \end{align}
  \subsection{Push}
    \[\variable{insL : \REQ \pfun \OBJ} \]
    \[\variable{insR : \REQ \pfun \OBJ} \]
  \begin{description}
    \comment{insL}{parameter for the \emph{push left} operation} 
    \comment{insR}{parameter for the \emph{push right} operation} 
  \end{description}
  \begin{align}
    \invariant{m3:inv2}{ insL \in pshL \tfun \OBJ } \\
    \invariant{m3:inv3}{ insR \in pshR \tfun \OBJ } \\
    \initialization{m3:init2}{ insL = \emptyfun } \\
    \initialization{m3:init4}{ insR = \emptyfun }
  \end{align}
  \begin{align}
    \evbcmeq{handle:pushL}{m3:act0}{p}{p - 1} \\
    \evbcmeq{handle:pushL}{m3:act1}{qe}{ qe \2| (p \0- 1 \fun insL.r)} \\
    \evbcmeq{handle:pushR}{m3:act0}{q}{q + 1} \\
    \evbcmeq{handle:pushR}{m3:act1}{qe}{ qe \2| (q \fun insR.r)} 
  \end{align}
  
  \[ \param{add:pushL}{ obj : \OBJ } \]

  \begin{align}
    \evbcmeq{add:pushL}{m3:act0}
      {insL}{ insL \1| r \fun obj } \\
    \evbcmeq{handle:pushL}{m3:act2}
      {insL}{ \{ r \} \domsub insL }
  \end{align}

  \[ \param{add:pushR}{ obj : \OBJ } \]
  
  \begin{align}
    \evbcmeq{add:pushR}{m3:act0}
      {insR}{ insR \1| r \fun obj } \\
    \evbcmeq{handle:pushR}{m3:act2}
      {insR}{ \{ r \} \domsub insR }
  \end{align}
  \[ \variable{ res : \REQ \pfun \OBJ } \]
  \[ \variable{ result : \OBJ } \]
  \[ \variable{ emp : \Bool } \]
  \subsection{Pop}
  \begin{align}
    \initialization{m3:init3}{ res = \emptyfun } \\
    \invariant{m3:inv4}{ res \in ppd \pfun \OBJ } \\
    \evbcmeq{return}{m3:act0}{res}{ \{r\} \domsub res} \\
    \evbcmsuch{return}{m3:act1}{result}{ r \in \dom.res \implies result' = res.r } \\
    \evbcmeq{return}{m3:act2}{emp}{ (r \in \dom.res) } 
  \end{align}
  \splitevent{handle:popL}{handle:popL:empty,handle:popL:non:empty}
  \splitevent{handle:popR}{handle:popR:empty,handle:popR:non:empty}
  \refiningevent{handle:popL}{handle:popL:empty}{handle\_popL\_empty}
  \refiningevent{handle:popL}{handle:popL:non:empty}{handle\_popL\_non\_empty}
  \refiningevent{handle:popR}{handle:popR:empty}{handle\_popR\_empty}
  \refiningevent{handle:popR}{handle:popR:non:empty}{handle\_popR\_non\_empty}
  \[ \dummy{v : \Int} \]
  \replace{handle:popL}{}{m3:sch0}{m1:sch1}{m3:prog0}
  \begin{align*}
    & \progress{m3:prog0}{v = ver}{p = q \lor p < q}
    \refine{m3:prog0}{implication}{}{}
  \end{align*}
  \begin{align}
    \cschedule{handle:popL:non:empty}{m3:sch0}{ p < q } \\
    \cschedule{handle:popL:empty}{m3:sch0}{ p = q } \\
    \evbcmeq{handle:popL:non:empty}{m3:act0}{res}{ res \1| (r \fun qe.p) } \\
    \evbcmeq{handle:popL:non:empty}{m3:act1}{p}{p+1} \\
    \evbcmeq{handle:popL:non:empty}{m3:act2}{qe}{ \{p\} \domsub qe }
  \end{align}
  \replace{handle:popR}{}{m3:sch0}{m1:sch1}{m3:prog0}
  \begin{align}
    \cschedule{handle:popR:non:empty}{m3:sch0}{ p < q } \\
    \cschedule{handle:popR:empty}{m3:sch0}{ p = q } \\
    \evbcmeq{handle:popR:non:empty}{m3:act0}{res}{ res \1| (r \fun qe.(q\0-1)) } \\
    \evbcmeq{handle:popR:non:empty}{m3:act1}{q}{q-1} \\
    \evbcmeq{handle:popR:non:empty}{m3:act2}{qe}{ \{q-1\} \domsub qe }
  \end{align}
\end{machine}
\section{Model m4 --- Memory nodes}
\newcommand{\Addr}{\text{Addr}}
  \begin{block}
  \item   \textbf{machine} m4
  \item   \textbf{variables}
  \begin{block}
    \item   $b$
    \item   \begin{block}
      \item    indicates termination of the process 
    \end{block}
    \item   $c$
    \item   \begin{block}
      \item    one-bit channel used to communicate 
    \end{block}
    \item   $fs$
    \item   \begin{block}
      \item    $p \in fs$ if process $p$ has flicked the switch 
    \end{block}
    \item   $n$
    \item   $vs$
    \item   \begin{block}
      \item   set of visited processes
    \end{block}
  \end{block}
  \item   \textbf{events}
  \begin{block}
    \item   %!TEX root=../puzzle.tex
\noindent \ref{count}  \textbf{event}
\begin{block}
  \item   \textbf{during}
  \begin{block}
  \item[ \eqref{countsch1} ]{$c = 1$} %
  \end{block}
  \item   \textbf{when}
  \begin{block}
  \item[ \eqref{countm3:grd0} ]{$c = 1$} %
  \end{block}
  \item   \textbf{begin}
  \begin{block}
  \item[ \eqref{countm3:act0} ]{$c \bcmeq 0$} %
  \item[ \eqref{countm3:act1} ]{$n \bcmeq n+1$} %
  \end{block}
  \item   \textbf{end} \\
\end{block}

    \item   %!TEX root=../puzzle.tex
\noindent \ref{flick} [p] \textbf{event}
  \item   \begin{block}
    \item   stuff
  \end{block}
\begin{block}
  \item   \textbf{during}
  \begin{block}
  \item[ \eqref{flickm3:csch1} ]{$c = 0 $} %
  \item[ \eqref{flickm3:csch2} ]{$\neg p \in fs $} %
  \item[ \eqref{flicksch2} ]{$p \in vs$} %
  \end{block}
  \item   \textbf{when}
  \begin{block}
  \item[ \eqref{flickgrd1} ]{$p \in vs$} %
  \item[ \eqref{flickm3:grd1} ]{$c = 0 $} %
  \item[ \eqref{flickm3:grd2} ]{$\neg p \in fs $} %
  \end{block}
  \item   \textbf{begin}
  \begin{block}
  \item[ \eqref{flickm3:act0} ]{$c \bcmeq 1$} %
  \item[ \eqref{flickm3:act2} ]{$fs \bcmeq fs \bunion \{ p \} $} %
  \end{block}
  \item   \textbf{end} \\
\end{block}

    \item   \noindent \ref{term}  \textbf{event}
\begin{block}
  \item   \textbf{during}
  \begin{block}
  \item[ \eqref{termsch3} ]$n = \card.\Pcs$ %
  \end{block}
  \item   \textbf{when}
  \begin{block}
  \item[ \eqref{termgrd0} ]$vs = \Pcs$ %
  \end{block}
  \item   \textbf{begin}
  \begin{block}
  \item[ \eqref{termact0} ]$b \bcmeq \true $ %
  \end{block}
  \item   \textbf{end} \\
\end{block}

    \item   \noindent \ref{visit} [p] \textbf{event}
\begin{block}
  \item   \textbf{begin}
  \begin{block}
  \item[ \eqref{visitact1} ]$vs \bcmeq vs \bunion \{ p \} $ %
  \end{block}
  \item   \textbf{end} \\
\end{block}

  \end{block}
  \item   \textbf{end} \\
\end{block}

\begin{machine}{m4}
  \refines{m3}
  \newset{\Addr}
  \[ \variable{ rep : \Int \pfun \Addr } \]
  \[ \constant{ dummy : \Addr } \]
  \begin{align*}
    \assumption{m4:asm0}{ \neg \finite{\Addr} } \\
    \invariant{m4:inv8}{ \finite.(\ran.rep) } \\
    \invariant{m4:inv0}{ \injective{rep} } \\
    \invariant{m4:inv1}
      { rep \in \intervalR{p}{q} \tfun \Addr } \\
    \invariant{m4:inv2}{ \neg dummy \in \ran.rep } \\
    \initialization{m4:init0}{ rep  = \emptyfun } \\
    \evbcmeq{handle:popL:non:empty}{m4:act0}
      {rep}{ \{ p \} \domsub rep } \\
    \evbcmeq{handle:popR:non:empty}{m4:act0}
      {rep}{ \{ q-1 \} \domsub rep } \\
    \evbcmeq{handle:pushL}{m4:act0}
      {rep}{ rep \2| (p\0-1 \fun n) } \\
    \evguard{handle:pushL}{m4:grd0}
      { \neg n \in \ran.rep \bunion \{dummy\} } \\
    \evbcmeq{handle:pushR}{m4:act0}
      {rep}{ rep \2| (q \fun n) }  \\
    \evguard{handle:pushR}{m4:grd0}
      { \neg n \in \ran.rep \bunion \{dummy\} } 
  \end{align*}
  \[ \param{handle:pushL}{ n : \Addr } \]
  \[ \param{handle:pushR}{ n : \Addr } \]
\end{machine}
  \input{lock-free-deque/machine_m5}
\begin{machine}{m5}
  \refines{m4}
\subsection{Data structure}
    % \refiningevent{handle}{handle:popL}{handle\_popL}
    % \refiningevent{handle}{handle:popR}{handle\_popR} 
    \refiningevent{handle:pushL}{handle:pushL:empty}{handle\_pushL\_empty}
    \refiningevent{handle:pushL}{handle:pushL:non:empty}{handle\_pushL\_non\_empty}
    \refiningevent{handle:pushR}{handle:pushR:empty}{handle\_pushR\_empty}
    \refiningevent{handle:pushR}{handle:pushR:non:empty}{handle\_pushR\_non\_empty}
  \splitevent{handle:pushR}{handle:pushR:empty,handle:pushR:non:empty}
  \splitevent{handle:pushL}{handle:pushL:empty,handle:pushL:non:empty}
    \refiningevent{handle:popL:non:empty}{handle:popL:one}{handle\_popL\_one}
    \refiningevent{handle:popL:non:empty}{handle:popL:more}{handle\_popL\_more}
    \refiningevent{handle:popR:non:empty}{handle:popR:one}{handle\_popR\_one}
    \refiningevent{handle:popR:non:empty}{handle:popR:more}{handle\_popR\_more}
  \splitevent{handle:popR:non:empty}{handle:popR:one,handle:popR:more}
  \splitevent{handle:popL:non:empty}{handle:popL:one,handle:popL:more}
  \[ \definition{Node}{ 
      \left[
      \begin{array}{l}
        'item : \OBJ \\ 
      , 'left : \Addr \\
      , 'right : \Addr  
      \end{array} \right]
      } \]
  \[ \variable{ link : \Addr \pfun 
      \{ 'item : \OBJ 
      , 'left : \Addr
      , 'right : \Addr  \} } \]
  \begin{align}
    \invariant{m5:inv1}
      { link \in \ran.rep \tfun Node } \\
    \invariant{m5:inv2}
      { \qforall{i}{\betweenR{p}{i}{q}}{ qe.i = link.(rep.i).'item } } \\
    \initialization{m5:init1}
      { link = \emptyfun } 
  \end{align}
\subsection{Pop}
  \begin{align}
    \initialization{m5:in0}{ RH = dummy } \\
    \initialization{m5:in1}{ LH = dummy } \\
    \evbcmeq{handle:popL:one}{m5:act1}{link}
      { \{ LH \} \domsub link } \\
    \evbcmeq{handle:popL:one}{m5:act2}{LH}
      { dummy } \\
    \evbcmeq{handle:popL:one}{m5:act5}{RH}
      { dummy } \\
    \evbcmeq{handle:popR:one}{m5:act1}{link}
      { \{ RH \} \domsub link } \\
    \evbcmeq{handle:popR:one}{m5:act2}{RH}
      { dummy } \\
    \evbcmeq{handle:popR:one}{m5:act4}{LH}
      { dummy } \\
    \evbcmeq{handle:popL:more}{m5:act1}{link}
      { \{ LH \} \domsub link } \\
    \evbcmeq{handle:popL:more}{m5:act2}
      {LH}{ link.LH.'right } \\
    \evbcmeq{handle:popR:more}{m5:act1}{link}
      { \{ RH \} \domsub link } \\
    \evbcmeq{handle:popR:more}{m5:act2}{RH}
      { link.RH.'left } 
  \end{align}
\subsection{Push}
\paragraph{\eqref{m5:inv0}}
  \[ \variable{ RH,LH : \Addr } \]
  \removeguard{handle:pushL:empty}{m4:grd0}
  \removeguard{handle:pushL:non:empty}{m4:grd0}
  \removeguard{handle:pushR:empty}{m4:grd0}
  \removeguard{handle:pushR:non:empty}{m4:grd0}
  \begin{align}
    \evguard{handle:pushL:empty}{m5:grd0}
      { \neg n \in \dom.link \bunion \{dummy\} } \\
    \evguard{handle:pushL:non:empty}{m5:grd0}
      { \neg n \in \dom.link \bunion \{dummy\} } \\
    \evbcmeq{handle:pushL:empty}{m5:act1}
      {link}{ \begin{array}{ll}        
            & link \\
        \2| & n \fun \left[ \begin{array}{l}
           'item  := insL.r, \\
           'left  := dummy, \\
           'right := dummy 
        \end{array} \right] 
      \end{array}
          } \\
    \cschedule{handle:pushL:empty}{m5:sch1}
      { LH = dummy } \\
    \cschedule{handle:pushL:non:empty}{m5:sch1}
      { \neg LH = dummy } \\
    \evbcmeq{handle:pushL:non:empty}{m5:act1}
      {link}{ \begin{array}{ll}        
            & link \\
        \2| & n \fun \left[ \begin{array}{l}
           'item  := insL.r, \\
           'left  := dummy, \\
           'right := LH 
        \end{array} \right] \\
          \2 | & LH \fun (link.LH) [ 'left := n ] 
      \end{array}
          } \\
    \evbcmeq{handle:pushL:empty}{m5:act2}
      {LH}{ n } \\
    \evbcmeq{handle:pushL:empty}{m5:act3}
      {RH}{ n } \\
    \evbcmeq{handle:pushL:non:empty}{m5:act2}
      {LH}{ n } \\
    \evguard{handle:pushR:empty}{m5:grd0}
      { \neg n \in \dom.link \bunion \{dummy\} } \\
    \evbcmeq{handle:pushR:empty}{m5:act1}
      {link}{ \begin{array}{ll}        
            & link \\
        \2| & n \fun \left[ 
        \begin{array}{l}
           'item  := insR.r, \\
           'left  := dummy, \\
           'right := dummy 
        \end{array} \right] 
      \end{array}
          } \\
    \evbcmeq{handle:pushR:empty}{m5:act2}
      {RH}{ n } \\
    \evbcmeq{handle:pushR:empty}{m5:act3}
      {LH}{ n } \\
    % \cschedule{handle:pushR:empty}{m5:sch0}
    %   { r \in pshR } \\
    \evguard{handle:pushR:non:empty}{m5:grd0}
      { \neg n \in \dom.link \bunion \{dummy\} } \\
    \cschedule{handle:pushR:empty}{m5:sch1}
      { RH = dummy } \\
    \cschedule{handle:pushR:non:empty}{m5:sch1}
      { \neg RH = dummy } \\
    \evbcmeq{handle:pushR:non:empty}{m5:act1}
      {link}{ \begin{array}{ll}        
            & link \\
        \2| & n \fun \left[ 
        \begin{array}{l}
           'item  := insR.r, \\
           'left  := RH, \\
           'right := dummy 
        \end{array} \right] \\
          \2 | & RH \fun (link.RH) [ 'right := n ]
      \end{array}
          } \\
    \evbcmeq{handle:pushR:non:empty}{m5:act2}
      {RH}{ n }
  \end{align}
  % \replace{handle:pushL}{}{m5:sch0}{m2:sch0}{m5:prog0}
  % \replace{handle:pushR}{}{m5:sch0}{m2:sch0}{m5:prog1}
  \[ \dummy{ r : \REQ } \]
  % \begin{align}
    % \progress{m5:prog0}{ r \in pshL }{ r \in \dom.nL } \\
    % \progress{m5:prog1}{ r \in pshR }{ r \in \dom.nR }
  % \end{align}
\paragraph{\eqref{m5:inv3}}
  % \begin{align}
    % \invariant{m5:inv6}
    %   { \qforall{r}{ r \in \dom.nL }{item.(nL.r).'item = insL.r} }\\
    % \evbcmeq{handle:pushL}{m5:act1}{nL}{ \{ r \} \domsub nL }
  % \end{align}  
  % \begin{align}
    % \invariant{m5:inv7}
      % { \qforall{r}{ r \in \dom.nR }{item.(nR.r).'item = insR.r} }\\
    % \evbcmeq{handle:pushR}{m5:act1}{nR}{ \{ r \} \domsub nR }
  % \end{align} 
\subsection{ Linking } 
% \subsection{ \eqref{m5:inv10} } 
  \begin{align}
    \invariant{m5:inv11}
      { \qforall{i}{\betweenR{p}{i}{q-1}}{ link.(rep.i).'right = rep.(i+1) } } \\
    \invariant{m5:inv12}
      { \qforall{i}{\betweenR{p}{i}{q-1}}{ link.(rep.(i+1)).'left = rep.i } } \\
    \invariant{m5:inv13}
      { p < q \2\implies LH = rep.p  } \\
    \invariant{m5:inv14}
      { p < q \2\implies RH = rep.(q-1) } \\
    \invariant{m5:inv15}
      { p = q \2\implies LH = dummy  } \\
    \invariant{m5:inv16}
      { p = q \2\implies RH = dummy }
  \end{align}
\subsection{New Progress Properties}
  \begin{align}
    \invariant{m5:inv7}{ \finite.(\dom.link) } 
  \end{align}
  \removevar{qe,rep,p,q}
  \removeact{handle:popL:one}{m3:act0}
  \removeact{handle:popL:more}{m3:act0}
  \begin{align}
  \evbcmeq{handle:popL:one}{m5:act3}
      {res}{ res \1| (r \fun link.LH.'item) } \\
  \evbcmeq{handle:popL:more}{m5:act3}
      {res}{ res \1| (r \fun link.LH.'item) } 
  \end{align}
  \removeact{handle:popL:one}{m3:act2} \\
  \removeact{handle:popL:more}{m3:act2} \\
  \removeact{handle:popR:one}{m3:act0} \\
  \removeact{handle:popR:more}{m3:act0} \\
  % \evbcmeq{handle:popR:non:empty}{m3:act0}{res}{ res \1| (r \fun qe.(q\0-1)) } \\
  \begin{align}
    \evbcmeq{handle:popR:one}{m5:act3}
      {res}{ res \1| (r \fun link.RH.'item) } \\
    \evbcmeq{handle:popR:more}{m5:act3}
      {res}{ res \1| (r \fun link.RH.'item) }
  \end{align}
  \removeact{handle:popR:one}{m3:act2} \\
  \removeact{handle:popR:more}{m3:act2} \\
  \removeact{handle:pushL:empty}{m3:act1} \\
  \removeact{handle:pushL:non:empty}{m3:act1} \\
  \removeact{handle:pushR:empty}{m3:act1} \\ 
  \removeact{handle:pushR:non:empty}{m3:act1} \\ 
  \removeinit{m3:init1}
  \removeinit{m4:init0}
  \removeinit{m3:init0}
  \begin{align}
    \initwitness{qe}{qe = \emptyfun} \\
    \invariant{m5:inv4}{ p = q \equiv LH = dummy } \\
    \invariant{m5:inv5}{ p = q \equiv RH = dummy } \\
    \initwitness{rep}{ rep = \emptyfun } \\
    \initwitness{p}{ p = 0 } \\
    \initwitness{q}{ q = 0 }
  \end{align}
  \removecoarse{handle:popR:one}{m3:sch0}
  \removecoarse{handle:popR:more}{m3:sch0}
  \begin{align*}
    \cschedule{handle:popR:one}{m5:sch0}{ \neg RH = dummy } \\
    \cschedule{handle:popR:one}{m5:sch1}{ RH = LH } \\
    \cschedule{handle:popR:more}{m5:sch0}{ \neg RH = dummy } \\
    \cschedule{handle:popR:more}{m5:sch1}{ \neg RH = LH } 
  \end{align*}
\removecoarse{handle:popL:one}{m3:sch0} 
\removecoarse{handle:popL:more}{m3:sch0} 
\removecoarse{handle:popL:empty}{m3:sch0} 
\removecoarse{handle:popR:empty}{m3:sch0}
\begin{align*}
  \cschedule{handle:popL:one}{m5:sch0}{ \neg LH = dummy } \\
  \cschedule{handle:popL:one}{m5:sch1}{ LH = RH } \\
  \cschedule{handle:popL:more}{m5:sch0}{ \neg LH = dummy } \\
  \cschedule{handle:popL:more}{m5:sch1}{ \neg LH = RH } \\
  \cschedule{handle:popR:empty}{m5:sch0}{ RH = dummy } \\
  \cschedule{handle:popL:empty}{m5:sch0}{ LH = dummy } 
\end{align*}
\removeact{handle:popL:one}{m3:act1} 
\removeact{handle:popL:more}{m3:act1}
\removeact{handle:popR:one}{m3:act1}
\removeact{handle:popR:more}{m3:act1}
\removeact{handle:pushL:empty}{m3:act0}
\removeact{handle:pushL:non:empty}{m3:act0}
\removeact{handle:pushR:empty}{m3:act0}
\removeact{handle:pushR:non:empty}{m3:act0}

  \removeact{handle:popL:one}{m4:act0}
  \removeact{handle:popL:more}{m4:act0}
  \removeact{handle:popR:one}{m4:act0}
  \removeact{handle:popR:more}{m4:act0}
  \removeact{handle:pushL:empty}{m4:act0}
  \removeact{handle:pushR:empty}{m4:act0}
  \removeact{handle:pushL:non:empty}{m4:act0}
  \removeact{handle:pushR:non:empty}{m4:act0}

\end{machine}

\begin{machine}{m6}
  \refines{m5}
\[ \variable{ tailL,tailR : \Addr \pfun \{ 'item : \OBJ 
      , 'left : \Addr
      , 'right : \Addr  \} } \]
\[ \definition{Shared}{ link | tailL | tailR } \]
\end{machine}

\section{Complete Development}

\begin{block}
  \item   \textbf{machine} m0
  \item   \textbf{variables}
  \begin{block}
    \item   $in$
  \end{block}
  \item   \textbf{progress}
\begin{block}
\item[ \eqref{m0:prog0} ]$t \in in  \quad \mapsto\quad \neg t \in in $ %
\end{block}

  \item   \textbf{transient}
\begin{block}
\item[ \eqref{m0:tr0} ]$t \in in  \qquad \text{(\ref{m0:leave}: [t := t' | t' = t])}$ %
\end{block}

  \item   \textbf{events}
  \begin{block}
    \item   \noindent \ref{m0:enter} [t] \textbf{event}
\begin{block}
\item \textbf{during}
\begin{block}
\item[ \eqref{m0:enterdefault} ]$\false$ %
\end{block}
\item \textbf{begin}
\begin{block}
\item[ \eqref{m0:entera1} ]$in' = in \bunion \{ t \} $ %
\end{block}
\item \textbf{end} \\
\end{block}

    \item   \noindent \ref{m0:leave} [t] \textbf{event}
\begin{block}
  \item   \textbf{during}
  \begin{block}
  \item[ \eqref{m0:leavelv:c0} ]$t \in in $ %
  \end{block}
  \item   \textbf{begin}
  \begin{block}
  \item[ \eqref{m0:leavelv:a0} ]$in \bcmsuch in' = in \setminus \{ t \} $ %
  \end{block}
  \item   \textbf{end} \\
\end{block}

  \end{block}
  \item   \textbf{end} \\
\end{block}

\begin{block}
  \item   \textbf{machine} m1
  \item   \textbf{variables}
  \begin{block}
    \item   $in$
    \item   $loc$
  \end{block}
  \item   %!TEX root=../main8.tex
\textbf{invariants}
\begin{block}
\item[ \eqref{m1:inv0} ]{$qe \in \intervalR{p}{q} \tfun \G $} %
\item[ \eqref{m1:inv1} ]{$p \le q $} %
\end{block}

  \item   \textbf{progress}
\begin{block}
\item[ \eqref{m1:prog0} ]$r \in \dom.pshL \1\land pshL.r = x  \quad \mapsto\quad p < q \land qe.p = x \1\land \neg r \in \dom.pshL $ %
\item[ \eqref{m1:prog1} ]$r \in \dom.pshR \1\land pshR.r = x  \quad \mapsto\quad p < q \1\land qe.(q-1) = x \1\land \neg r \in \dom.pshR $ %
\item[ \eqref{m1:prog2} ]$r \in popR  \quad \mapsto\quad \neg r \in popR $ %
\item[ \eqref{m1:prog3} ]$r \in popL  \quad \mapsto\quad \neg r \in popL  $ %
\end{block}

  \item   \textbf{safety}
\begin{block}
\item[ \eqref{m1:saf0} ]$\neg t \in in  \textbf{\quad unless \quad} t \in in \land loc.t = ent $ %
\item[ \eqref{m1:saf1} ]$t \in in \land loc.t = ent  \textbf{\quad unless \quad} t \in in \land loc.t \in plf $ %
\item[ \eqref{m1:saf2} ]$t \in in \land loc.t \in plf  \textbf{\quad unless \quad} t \in in \land loc.t = ext $ %
\item[ \eqref{m1:saf3} ]$t \in in \land loc.t = ext  \textbf{\quad unless \quad} \neg t \in in $ %
\end{block}

  \item   %!TEX root=../puzzle.tex

  \item   \textbf{events}
  \begin{block}
    \item   \noindent \ref{m0:enter} [t] \textbf{event}
\begin{block}
  \item   \textbf{when}
  \begin{block}
  \item[ \eqref{m0:enterent:grd1} ]$\neg t \in in $ %
  \end{block}
  \item   \textbf{begin}
  \begin{block}
  \item[ \eqref{m0:entera1} ]$in \bcmsuch in' = in \bunion \{ t \} $ %
  \item[ \eqref{m0:entera3} ]$loc \bcmsuch loc' = loc \1| t \fun ent $ %
  \end{block}
  \item   \textbf{end} \\
\end{block}

    \item   \noindent \ref{m0:leave} [t] \textbf{event}
\begin{block}
  \item   \textbf{during}
  \begin{block}
  \item[ \eqref{m0:leavelv:c0} ]$t \in in $ %
  \item[ \eqref{m0:leavelv:c1} ]$loc.t = ext $ %
  \end{block}
  \item   \textbf{when}
  \begin{block}
  \item[ \eqref{m0:leavelv:grd0} ]$t \in in $ %
  \item[ \eqref{m0:leavelv:grd1} ]$loc.t = ext $ %
  \end{block}
  \item   \textbf{begin}
  \begin{block}
  \item[ \eqref{m0:leavelv:a0} ]$in \bcmsuch in' = in \setminus \{ t \} $ %
  \item[ \eqref{m0:leavelv:a2} ]$loc \bcmsuch loc' = \{ t \} \domsub loc $ %
  \end{block}
  \item   \textbf{end} \\
\end{block}

    \item   \noindent \ref{m1:movein} [t] \textbf{event}
\begin{block}
  \item   \textbf{during}
  \begin{block}
  \item[ \eqref{m1:moveinmi:c1} ]$t \in in $ %
  \item[ \eqref{m1:moveinmi:c2} ]$loc.t = ent $ %
  \end{block}
  \item   \textbf{any} b
  \item   \textbf{when}
  \begin{block}
  \item[ \eqref{m1:moveinmi:g1} ]$t \in in $ %
  \item[ \eqref{m1:moveinmi:grd0} ]$loc.t = ent $ %
  \item[ \eqref{m1:moveinmi:grd7} ]$b \in plf $ %
  \end{block}
  \item   \textbf{begin}
  \begin{block}
  \item[ \eqref{m1:moveinmi:a2} ]$loc \bcmsuch loc' = loc \1| t \fun b $ %
  \end{block}
  \item   \textbf{end} \\
\end{block}

    \item   \noindent \ref{m1:moveout} [t] \textbf{event}
\begin{block}
\item \textbf{during}
\begin{block}
\item[ \eqref{m1:moveoutc1} ]$t \in in  %
		\1\land loc.t \in plf $ %
\end{block}
\item \textbf{when}
\begin{block}
\item[ \eqref{m1:moveoutmo:g1} ]$t \in in $ %
\item[ \eqref{m1:moveoutmo:g2} ]$loc.t \in plf $ %
\end{block}
\item \textbf{begin}
\begin{block}
\item[ \eqref{m1:moveoutSKIP:in} ]$in' = in$ %
\item[ \eqref{m1:moveouta2} ]$loc' = loc \1 | t \fun ext $ %
\end{block}
\item \textbf{end} \\
\end{block}

  \end{block}
  \item   \textbf{end} \\
\end{block}

%!TEX root=../main9.tex
\begin{block}
  \item   \textbf{machine} m2
  \item   \textbf{refines} m0
  \item   \textbf{variables}
  \begin{block}
    \item   $req$
    \item   $req0$
    \item   $reqA$
    \item   $reqB$
  \end{block}
  \item   %!TEX root=../main9.tex
\textbf{invariants}
\begin{block}
\item[ \eqref{m1:inv0} ]{$reqA \bunion reqB = req$} %
\item[ \eqref{m1:inv1} ]{$reqA \binter reqB = req$} %
\end{block}

  \item   \textbf{events}
  \begin{block}
    \item   %!TEX root=../main9.tex
\noindent \ref{handle}  \textbf{event}
\begin{block}
  \item   \textbf{during}
  \begin{block}
  \item[ \eqref{handlem0:sch0} ]{$\neg req = \emptyset$} %
  \end{block}
  \item   \textbf{any} r
  \item   \textbf{when}
  \begin{block}
  \item[ \eqref{handlegrd0} ]{$r \in req$} %
  \end{block}
  \item   \textbf{begin}
  \begin{block}
  \item[ \eqref{handleact0} ]{$req \bcmeq req \setminus \{ r \}$} %
  \item[ \eqref{handleact1} ]{$req0 \bcmeq req$} %
  \end{block}
  \item   \textbf{end} \\
\end{block}

    \item   %!TEX root=../main9.tex
\noindent \ref{req}  \textbf{event}
\begin{block}
  \item   \textbf{during}
  \begin{block}
  \item[ (\ref{req}/default) ]{$\false$} %
  \end{block}
  \item   \textbf{any} r
  \item   \textbf{when}
  \begin{block}
  \item[ \eqref{reqgrd0} ]{$\neg r \in req$} %
  \end{block}
  \item   \textbf{begin}
  \begin{block}
  \item[ \eqref{reqact0} ]{$req \bcmeq req \bunion \{ r \}$} %
  \item[ \eqref{reqact1} ]{$req0 \bcmeq req$} %
  \end{block}
  \item   \textbf{end} \\
\end{block}

  \end{block}
  \item   \textbf{end} \\
\end{block}

\pagebreak
\begin{block}
  \item   \textbf{machine} m3
  \item   \textbf{variables}
  \begin{block}
    \item   $in$
    \item   $isgn$
    \item   $loc$
    \item   $osgn$
  \end{block}
  \item   %!TEX root=../main.tex
\textbf{invariants}
\begin{block}
\item[ \eqref{m3:inv0} ]{$\dom.delta = Pcs $} %
\item[ \eqref{m3:inv1} ]{$b \2\subseteq d \setminus \qset{p,q}{q \in delta.p}{q} $} %
\end{block}

  \item   \textbf{events}
  \begin{block}
    \item   \noindent \ref{m0:enter} [t] \textbf{event}
\begin{block}
  \item   \textbf{when}
  \begin{block}
  \item[ \eqref{m0:enterent:grd1} ]$\neg t \in in $ %
  \item[ \eqref{m0:enteret:g1} ]$\neg ent \in \ran.loc $ %
  \end{block}
  \item   \textbf{begin}
  \begin{block}
  \item[ \eqref{m0:entera1} ]$in \bcmsuch in' = in \bunion \{ t \} $ %
  \item[ \eqref{m0:entera3} ]$loc \bcmsuch loc' = loc \1| t \fun ent $ %
  \item[ \eqref{m0:enterm3:ent:act0} ]$isgn,osgn \bcmsuch isgn' = isgn$ %
  \item[ \eqref{m0:enterm3:ent:act1} ]$isgn,osgn \bcmsuch osgn' = osgn$ %
  \end{block}
  \item   \textbf{end} \\
\end{block}

    \item   \noindent \ref{m0:leave} [t] \textbf{event}
\begin{block}
  \item   \textbf{during}
  \begin{block}
  \item[ \eqref{m0:leavelv:c0} ]$t \in in $ %
  \item[ \eqref{m0:leavelv:c1} ]$loc.t = ext $ %
  \end{block}
  \item   \textbf{when}
  \begin{block}
  \item[ \eqref{m0:leavelv:grd0} ]$t \in in $ %
  \item[ \eqref{m0:leavelv:grd1} ]$loc.t = ext $ %
  \end{block}
  \item   \textbf{begin}
  \begin{block}
  \item[ \eqref{m0:leavelv:a0} ]$in \bcmsuch in' = in \setminus \{ t \} $ %
  \item[ \eqref{m0:leavelv:a2} ]$loc \bcmsuch loc' = \{ t \} \domsub loc $ %
  \item[ \eqref{m0:leavem3:ext:act0} ]$isgn,osgn \bcmsuch isgn' = isgn$ %
  \item[ \eqref{m0:leavem3:ext:act1} ]$isgn,osgn \bcmsuch osgn' = osgn$ %
  \end{block}
  \item   \textbf{end} \\
\end{block}

    \item   \noindent \ref{m1:movein} [t] \textbf{event}
\begin{block}
  \item   \textbf{during}
  \begin{block}
  \item[ \eqref{m1:moveinmi:c0} ]$\neg ~ plf \subseteq \ran.loc $ %
  \item[ \eqref{m1:moveinmi:c1} ]$t \in in $ %
  \item[ \eqref{m1:moveinmi:c2} ]$loc.t = ent $ %
  \end{block}
  \item   \textbf{any} b
  \item   \textbf{when}
  \begin{block}
  \item[ \eqref{m1:moveinmi:g0} ]$\neg b \in \ran.loc  	$ %
  \item[ \eqref{m1:moveinmi:g1} ]$t \in in $ %
  \item[ \eqref{m1:moveinmi:grd0} ]$loc.t = ent $ %
  \item[ \eqref{m1:moveinmi:grd7} ]$b \in plf $ %
  \end{block}
  \item   \textbf{begin}
  \begin{block}
  \item[ \eqref{m1:moveinm3:mi:act0} ]$isgn,osgn \bcmsuch isgn' = isgn$ %
  \item[ \eqref{m1:moveinm3:mi:act1} ]$isgn,osgn \bcmsuch osgn' = osgn$ %
  \item[ \eqref{m1:moveinmi:a2} ]$loc \bcmsuch loc' = loc \1| t \fun b $ %
  \end{block}
  \item   \textbf{end} \\
\end{block}

    \item   \noindent \ref{m1:moveout} [t] \textbf{event}
\begin{block}
\item \textbf{during}
\begin{block}
\item[ \eqref{m1:moveoutc1} ]$t \in in  %
		\1\land loc.t \in plf $ %
\item[ \eqref{m1:moveoutm3:mo:sch0} ]$loc.t \in osgn $ %
\end{block}
\item \textbf{when}
\begin{block}
\item[ \eqref{m1:moveoutm3:mo:grd0} ]$loc.t \in osgn $ %
\item[ \eqref{m1:moveoutmo:g1} ]$t \in in $ %
\item[ \eqref{m1:moveoutmo:g2} ]$loc.t \in plf $ %
\end{block}
\item \textbf{begin}
\begin{block}
\item[ \eqref{m1:moveoutSKIP:in} ]$in' = in$ %
\item[ \eqref{m1:moveouta2} ]$loc' = loc \1 | t \fun ext $ %
\item[ \eqref{m1:moveoutm3:mo:act0} ]$isgn' = isgn$ %
\item[ \eqref{m1:moveoutm3:mo:act1} ]$osgn'  \2 = osgn  %
	\setminus \{ loc.t \} $ %
\end{block}
\item \textbf{end} \\
\end{block}

  \end{block}
  \item   \textbf{end} \\
\end{block}

\pagebreak
\begin{block}
  \item   \textbf{machine} m4
  \item   \textbf{variables}
  \begin{block}
    \item   $b$
    \item   \begin{block}
      \item    indicates termination of the process 
    \end{block}
    \item   $c$
    \item   \begin{block}
      \item    one-bit channel used to communicate 
    \end{block}
    \item   $fs$
    \item   \begin{block}
      \item    $p \in fs$ if process $p$ has flicked the switch 
    \end{block}
    \item   $n$
    \item   $vs$
    \item   \begin{block}
      \item   set of visited processes
    \end{block}
  \end{block}
  \item   \textbf{events}
  \begin{block}
    \item   %!TEX root=../puzzle.tex
\noindent \ref{count}  \textbf{event}
\begin{block}
  \item   \textbf{during}
  \begin{block}
  \item[ \eqref{countsch1} ]{$c = 1$} %
  \end{block}
  \item   \textbf{when}
  \begin{block}
  \item[ \eqref{countm3:grd0} ]{$c = 1$} %
  \end{block}
  \item   \textbf{begin}
  \begin{block}
  \item[ \eqref{countm3:act0} ]{$c \bcmeq 0$} %
  \item[ \eqref{countm3:act1} ]{$n \bcmeq n+1$} %
  \end{block}
  \item   \textbf{end} \\
\end{block}

    \item   %!TEX root=../puzzle.tex
\noindent \ref{flick} [p] \textbf{event}
  \item   \begin{block}
    \item   stuff
  \end{block}
\begin{block}
  \item   \textbf{during}
  \begin{block}
  \item[ \eqref{flickm3:csch1} ]{$c = 0 $} %
  \item[ \eqref{flickm3:csch2} ]{$\neg p \in fs $} %
  \item[ \eqref{flicksch2} ]{$p \in vs$} %
  \end{block}
  \item   \textbf{when}
  \begin{block}
  \item[ \eqref{flickgrd1} ]{$p \in vs$} %
  \item[ \eqref{flickm3:grd1} ]{$c = 0 $} %
  \item[ \eqref{flickm3:grd2} ]{$\neg p \in fs $} %
  \end{block}
  \item   \textbf{begin}
  \begin{block}
  \item[ \eqref{flickm3:act0} ]{$c \bcmeq 1$} %
  \item[ \eqref{flickm3:act2} ]{$fs \bcmeq fs \bunion \{ p \} $} %
  \end{block}
  \item   \textbf{end} \\
\end{block}

    \item   \noindent \ref{term}  \textbf{event}
\begin{block}
  \item   \textbf{during}
  \begin{block}
  \item[ \eqref{termsch3} ]$n = \card.\Pcs$ %
  \end{block}
  \item   \textbf{when}
  \begin{block}
  \item[ \eqref{termgrd0} ]$vs = \Pcs$ %
  \end{block}
  \item   \textbf{begin}
  \begin{block}
  \item[ \eqref{termact0} ]$b \bcmeq \true $ %
  \end{block}
  \item   \textbf{end} \\
\end{block}

    \item   \noindent \ref{visit} [p] \textbf{event}
\begin{block}
  \item   \textbf{begin}
  \begin{block}
  \item[ \eqref{visitact1} ]$vs \bcmeq vs \bunion \{ p \} $ %
  \end{block}
  \item   \textbf{end} \\
\end{block}

  \end{block}
  \item   \textbf{end} \\
\end{block}

\pagebreak
\input{lock-free-deque/machine_m5}
\pagebreak
\input{lock-free-deque/machine_m6}

\end{document}
