\documentclass[12pt]{amsart}
\usepackage[margin=0.5in]{geometry} 
  % see geometry.pdf on how to lay out the page. There's lots.
\usepackage{bsymb}
\usepackage{../unitb}
\usepackage{calculational}
\usepackage{ulem}
\usepackage{hyperref}

\newcommand{\REQ}{\text{REQ}}

\begin{document}
  % \begin{block}
  \item   \textbf{machine} m0
  \item   \textbf{variables}
  \begin{block}
    \item   $in$
  \end{block}
  \item   \textbf{progress}
\begin{block}
\item[ \eqref{m0:prog0} ]$t \in in  \quad \mapsto\quad \neg t \in in $ %
\end{block}

  \item   \textbf{transient}
\begin{block}
\item[ \eqref{m0:tr0} ]$t \in in  \qquad \text{(\ref{m0:leave}: [t := t' | t' = t])}$ %
\end{block}

  \item   \textbf{events}
  \begin{block}
    \item   \noindent \ref{m0:enter} [t] \textbf{event}
\begin{block}
\item \textbf{during}
\begin{block}
\item[ \eqref{m0:enterdefault} ]$\false$ %
\end{block}
\item \textbf{begin}
\begin{block}
\item[ \eqref{m0:entera1} ]$in' = in \bunion \{ t \} $ %
\end{block}
\item \textbf{end} \\
\end{block}

    \item   \noindent \ref{m0:leave} [t] \textbf{event}
\begin{block}
  \item   \textbf{during}
  \begin{block}
  \item[ \eqref{m0:leavelv:c0} ]$t \in in $ %
  \end{block}
  \item   \textbf{begin}
  \begin{block}
  \item[ \eqref{m0:leavelv:a0} ]$in \bcmsuch in' = in \setminus \{ t \} $ %
  \end{block}
  \item   \textbf{end} \\
\end{block}

  \end{block}
  \item   \textbf{end} \\
\end{block}

\begin{machine}{m0}

  \newset{\REQ}

We need a variable to keep track of all the requests to mutate the
data structure.

  \[ \variable{ req,req_0 : \set [\REQ] } \]

... and new events:
% % \hide{
  \newevent{req}{req}
  \newevent{handle}{handle}
  % \newevent{req}{req\_b} 
  % \newevent{handle}{handle\_b} 

\[\param{req}{r : \REQ }\]
\with{sets}
\begin{align*}
  & \evguard{req}{grd0}{ \neg r \in req }  \\
  & \evbcmeq{req}{act0}{ req }{ req \bunion \{ r \} } \\
  & \evbcmeq{req}{act1}{ req_0 }{ req } 
\end{align*}
\[\param{handle}{r : \REQ }\]
\begin{align*}
  \evguard{handle}{grd0}{ r \in req }  \\
  \evbcmeq{handle}{act0}{ req }{ req \setminus \{ r \} } \\
  \evbcmeq{handle}{act1}{ req_0 }{ req }
\end{align*}

% \section{Requirements}
%   \dummy{ R : \set[\REQ] }
\begin{align*}
  \constraint{co0}{ req_0' = req \1\lor (req_0' = req_0 \1\land req' = req) }
\end{align*}
\begin{align*}
  & \progress{prog0}
    { \neg req = \emptyset }
    { \neg req_0 \subseteq req }
 \refine{prog0}{ensure}{handle}{ using \ref{handle} }
  & \progress{prog1}
    { V = req}
    { req \subset V 
      \1\lor req = \emptyset \1\lor \neg req \subseteq req_0}
 \refine{prog1}{ensure}{handle}{ using \ref{handle} }
\end{align*}
\begin{align*}
  \dummy{ R,V : \set [\REQ] } \\
  \cschedule{handle}{m0:sch0}{ \neg req = \emptyset } \\
  \initialization{m0:in0}{ req = \emptyset }
\end{align*}
\end{machine}

% \begin{block}
  \item   \textbf{machine} m1
  \item   \textbf{variables}
  \begin{block}
    \item   $in$
    \item   $loc$
  \end{block}
  \item   %!TEX root=../main8.tex
\textbf{invariants}
\begin{block}
\item[ \eqref{m1:inv0} ]{$qe \in \intervalR{p}{q} \tfun \G $} %
\item[ \eqref{m1:inv1} ]{$p \le q $} %
\end{block}

  \item   \textbf{progress}
\begin{block}
\item[ \eqref{m1:prog0} ]$r \in \dom.pshL \1\land pshL.r = x  \quad \mapsto\quad p < q \land qe.p = x \1\land \neg r \in \dom.pshL $ %
\item[ \eqref{m1:prog1} ]$r \in \dom.pshR \1\land pshR.r = x  \quad \mapsto\quad p < q \1\land qe.(q-1) = x \1\land \neg r \in \dom.pshR $ %
\item[ \eqref{m1:prog2} ]$r \in popR  \quad \mapsto\quad \neg r \in popR $ %
\item[ \eqref{m1:prog3} ]$r \in popL  \quad \mapsto\quad \neg r \in popL  $ %
\end{block}

  \item   \textbf{safety}
\begin{block}
\item[ \eqref{m1:saf0} ]$\neg t \in in  \textbf{\quad unless \quad} t \in in \land loc.t = ent $ %
\item[ \eqref{m1:saf1} ]$t \in in \land loc.t = ent  \textbf{\quad unless \quad} t \in in \land loc.t \in plf $ %
\item[ \eqref{m1:saf2} ]$t \in in \land loc.t \in plf  \textbf{\quad unless \quad} t \in in \land loc.t = ext $ %
\item[ \eqref{m1:saf3} ]$t \in in \land loc.t = ext  \textbf{\quad unless \quad} \neg t \in in $ %
\end{block}

  \item   %!TEX root=../puzzle.tex

  \item   \textbf{events}
  \begin{block}
    \item   \noindent \ref{m0:enter} [t] \textbf{event}
\begin{block}
  \item   \textbf{when}
  \begin{block}
  \item[ \eqref{m0:enterent:grd1} ]$\neg t \in in $ %
  \end{block}
  \item   \textbf{begin}
  \begin{block}
  \item[ \eqref{m0:entera1} ]$in \bcmsuch in' = in \bunion \{ t \} $ %
  \item[ \eqref{m0:entera3} ]$loc \bcmsuch loc' = loc \1| t \fun ent $ %
  \end{block}
  \item   \textbf{end} \\
\end{block}

    \item   \noindent \ref{m0:leave} [t] \textbf{event}
\begin{block}
  \item   \textbf{during}
  \begin{block}
  \item[ \eqref{m0:leavelv:c0} ]$t \in in $ %
  \item[ \eqref{m0:leavelv:c1} ]$loc.t = ext $ %
  \end{block}
  \item   \textbf{when}
  \begin{block}
  \item[ \eqref{m0:leavelv:grd0} ]$t \in in $ %
  \item[ \eqref{m0:leavelv:grd1} ]$loc.t = ext $ %
  \end{block}
  \item   \textbf{begin}
  \begin{block}
  \item[ \eqref{m0:leavelv:a0} ]$in \bcmsuch in' = in \setminus \{ t \} $ %
  \item[ \eqref{m0:leavelv:a2} ]$loc \bcmsuch loc' = \{ t \} \domsub loc $ %
  \end{block}
  \item   \textbf{end} \\
\end{block}

    \item   \noindent \ref{m1:movein} [t] \textbf{event}
\begin{block}
  \item   \textbf{during}
  \begin{block}
  \item[ \eqref{m1:moveinmi:c1} ]$t \in in $ %
  \item[ \eqref{m1:moveinmi:c2} ]$loc.t = ent $ %
  \end{block}
  \item   \textbf{any} b
  \item   \textbf{when}
  \begin{block}
  \item[ \eqref{m1:moveinmi:g1} ]$t \in in $ %
  \item[ \eqref{m1:moveinmi:grd0} ]$loc.t = ent $ %
  \item[ \eqref{m1:moveinmi:grd7} ]$b \in plf $ %
  \end{block}
  \item   \textbf{begin}
  \begin{block}
  \item[ \eqref{m1:moveinmi:a2} ]$loc \bcmsuch loc' = loc \1| t \fun b $ %
  \end{block}
  \item   \textbf{end} \\
\end{block}

    \item   \noindent \ref{m1:moveout} [t] \textbf{event}
\begin{block}
\item \textbf{during}
\begin{block}
\item[ \eqref{m1:moveoutc1} ]$t \in in  %
		\1\land loc.t \in plf $ %
\end{block}
\item \textbf{when}
\begin{block}
\item[ \eqref{m1:moveoutmo:g1} ]$t \in in $ %
\item[ \eqref{m1:moveoutmo:g2} ]$loc.t \in plf $ %
\end{block}
\item \textbf{begin}
\begin{block}
\item[ \eqref{m1:moveoutSKIP:in} ]$in' = in$ %
\item[ \eqref{m1:moveouta2} ]$loc' = loc \1 | t \fun ext $ %
\end{block}
\item \textbf{end} \\
\end{block}

  \end{block}
  \item   \textbf{end} \\
\end{block}

\begin{machine}{m1}
  \refines{m0}
\[ \indices{handle}{ b : \Bool } \]
\[ \dummy{ b : \Bool } \]
\[ \variable{ ch : \Bool } \]
\begin{align*}
  & \cschedule{handle}{m1:sch0}{ b = ch } \\
  & \witness{handle}{b}{b = ch} \\
  & \progress{m1:prog1}{ b = ch }{ b = ch }
\end{align*}
\replace{handle}{m1:sch0}{m1:prog1}
  \begin{liveness}{m1:prog1}
    \progstep{\true}{req = \emptyset \1\lor \neg req \subseteq req_0}
      {induction}{}{ \var{req}{down}{\emptyset} }
  \begin{flatstep}
    \progstep
      {V = req}
      {req \subset V
        \1\lor req = \emptyset \1\lor \neg req \subseteq req_0}
      {discharge}{}{}
      \begin{step}
        \trstep{handle}{ \index{b}{ \true }  }
          { req = V \1\land \neg req = \emptyset  }
        \safstep
          { V = req }
          { req \subset V \1\lor \neg req \subseteq req_0 }
          {}
      \end{step}
  \end{flatstep}
  \end{liveness}
\end{machine}

% % %!TEX root=../main9.tex
\begin{block}
  \item   \textbf{machine} m2
  \item   \textbf{refines} m0
  \item   \textbf{variables}
  \begin{block}
    \item   $req$
    \item   $req0$
    \item   $reqA$
    \item   $reqB$
  \end{block}
  \item   %!TEX root=../main9.tex
\textbf{invariants}
\begin{block}
\item[ \eqref{m1:inv0} ]{$reqA \bunion reqB = req$} %
\item[ \eqref{m1:inv1} ]{$reqA \binter reqB = req$} %
\end{block}

  \item   \textbf{events}
  \begin{block}
    \item   %!TEX root=../main9.tex
\noindent \ref{handle}  \textbf{event}
\begin{block}
  \item   \textbf{during}
  \begin{block}
  \item[ \eqref{handlem0:sch0} ]{$\neg req = \emptyset$} %
  \end{block}
  \item   \textbf{any} r
  \item   \textbf{when}
  \begin{block}
  \item[ \eqref{handlegrd0} ]{$r \in req$} %
  \end{block}
  \item   \textbf{begin}
  \begin{block}
  \item[ \eqref{handleact0} ]{$req \bcmeq req \setminus \{ r \}$} %
  \item[ \eqref{handleact1} ]{$req0 \bcmeq req$} %
  \end{block}
  \item   \textbf{end} \\
\end{block}

    \item   %!TEX root=../main9.tex
\noindent \ref{req}  \textbf{event}
\begin{block}
  \item   \textbf{during}
  \begin{block}
  \item[ (\ref{req}/default) ]{$\false$} %
  \end{block}
  \item   \textbf{any} r
  \item   \textbf{when}
  \begin{block}
  \item[ \eqref{reqgrd0} ]{$\neg r \in req$} %
  \end{block}
  \item   \textbf{begin}
  \begin{block}
  \item[ \eqref{reqact0} ]{$req \bcmeq req \bunion \{ r \}$} %
  \item[ \eqref{reqact1} ]{$req0 \bcmeq req$} %
  \end{block}
  \item   \textbf{end} \\
\end{block}

  \end{block}
  \item   \textbf{end} \\
\end{block}

% \begin{machine}{m2}
%   \refines{m0}

% We now partitial $req$ into requests for operation A ($reqA$) and
% requests for operation B ($reqB$).

% \[ \variable{ reqA, reqB : \set [\REQ] } \]
% \begin{align*}
%   \invariant{m1:inv0}{ reqA \bunion reqB = req } \\
%   \invariant{m1:inv1}{ reqA \binter reqB = req } 
% \end{align*}
% And consequ
% \begin{align*}
%   \initialization{m1:in0}{ reqA = \emptyset \land reqB = \emptyset }
% \end{align*}
% \end{machine}

\end{document}
