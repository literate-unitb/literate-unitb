\documentclass[12pt]{amsart}
\usepackage[margin=0.5in]{geometry} 
	% see geometry.pdf on how to lay out the page. There's lots.
\usepackage{bsymb}
\usepackage{unitb}
\usepackage{calculational}
\usepackage{ulem}
\usepackage{hyperref}
\normalem
\geometry{a4paper} % or letter or a5paper or ... etc
% \geometry{landscape} % rotated page geometry

% See the ``Article customise'' template for some common
% customisations

\title{}
\author{}
\date{} % delete this line to display the current date

%%% BEGIN DOCUMENT
\setcounter{tocdepth}{4}
\begin{document}

\maketitle
\tableofcontents

\newcommand{\stmt}[1]{ $#1$ }
\newcommand{\comment}[2]{ 
	% \begin{tabular}{|l|l|} 
	% 	\hline
		\begin{description}
		\item[{#1}] ~ #2 
		\end{description}
	% 	\hline 
	% \end{tabular} 
	}
\newcommand{\commentbox}[1]{ 
	\begin{block} \item \small{ #1 } \end{block} }
\newcommand{\G}{\text{Val}}
% \renewcommand{\H}{\text{H}}

\section{Initial model --- A sequence of objects}
\begin{machine}{m0}

\with{functions}
\with{sets}
\with{intervals}
\newset{\G}

\begin{align*}
& \variable{p,q : \Int } 
\\ & \variable{qe : \Int \pfun \G }
\\ & \variable{ emp : \Bool }
\\ & \variable{ res : \G }
\end{align*}

\begin{align*}
\invariant{m0:inv0}{ qe &\1\in \intervalR{p}{q} \tfun \G }
% \invariant{m0:inv0}{ \dom.qe = \intervalR{p}{q} }
\\ \invariant{m0:inv1}{ p &\1\le q }
\end{align*}

\begin{align*}
\initialization{m0:init0}{p = 0 \land q = 0 \land qe = \emptyfun }
\end{align*}

\newevent{m0:push:left}{push\_left}
\param{m0:push:left}{x : \G}

\begin{align*}
\evbcmeq{m0:push:left}{m0:act0}{qe}{ qe \2| p\0-1 \fun x }
\\ \evbcmeq{m0:push:left}{m0:act1}{p}{ p-1 }
\end{align*}

\newevent{m0:push:right}{push\_right}
\param{m0:push:right}{ x : \G }

\begin{align*}
\evbcmeq{m0:push:right}{m0:act0}{qe}{ qe \2| q \fun x }
\\ \evbcmeq{m0:push:right}{m0:act2}{q}{ q+1 }
\end{align*}

\newevent{m0:pop:left:empty}{pop\_left\_empty}

\begin{align*}
\evguard{m0:pop:left:empty}{m0:grd0}
	{ p = q }
\\ \evbcmeq{m0:pop:left:empty}{m0:act4}
	{emp}{ \true }
\end{align*}

\newevent{m0:pop:left:non:empty}{pop\_left\_non-empty}

\begin{align*}
\evguard{m0:pop:left:non:empty}{m0:grd0}
	{ p < q }
\\ \evbcmeq{m0:pop:left:non:empty}{m0:act0}
	{qe}{ \{ p \} \domsub qe }
\\ \evbcmeq{m0:pop:left:non:empty}{m0:act1}
	{p}{ p\0+1 }
\\ \evbcmeq{m0:pop:left:non:empty}{m0:act3}
	{res}{ qe.p }
\\ \evbcmeq{m0:pop:left:non:empty}{m0:act4}
	{emp}{ \false }
\end{align*}


\newevent{m0:pop:right:empty}{pop\_right\_empty}

\begin{align*}
\evguard{m0:pop:right:empty}{m0:grd0}
	{ &p = q }
% \\ \evguard{m0:pop:right}{m0:grd0}{ p &< q }
\\ \evbcmeq{m0:pop:right:empty}{m0:act4}
	{emp}{ \true }
\end{align*}

\newevent{m0:pop:right:non:empty}{pop\_right\_non-empty}

\begin{align*}
\evguard{m0:pop:right:non:empty}{m0:grd0}
	{ &p < q }
\\ \evbcmeq{m0:pop:right:non:empty}{m0:act0}
	{qe}{ \{ q-1 \} \domsub qe }
\\ \evbcmeq{m0:pop:right:non:empty}{m0:act2}
	{q}{ q\0-1 }
% \\ \evguard{m0:pop:right}{m0:grd0}{ p &< q }
\\ \evbcmeq{m0:pop:right:non:empty}{m0:act3}
	{res}{ qe.(q\1-1) }
\\ \evbcmeq{m0:pop:right:non:empty}{m0:act4}
	{emp}{ \false }
\end{align*}

% \begin{align*}
\noindent \ref{m0:push:left}  \textbf{event}
\begin{block}
  \item   \textbf{any} x
  \item   \textbf{begin}
  \begin{block}
  \item[ \eqref{m0:push:leftm0:act0} ]$qe \bcmeq qe \2| p\0-1 \fun x $ %
  \item[ \eqref{m0:push:leftm0:act1} ]$p \bcmeq p-1 $ %
  \end{block}
  \item   \textbf{end} \\
\end{block}

% \end{align*}

\begin{block}
  \item   \textbf{machine} m0
  \item   \textbf{variables}
  \begin{block}
    \item   $in$
  \end{block}
  \item   \textbf{progress}
\begin{block}
\item[ \eqref{m0:prog0} ]$t \in in  \quad \mapsto\quad \neg t \in in $ %
\end{block}

  \item   \textbf{transient}
\begin{block}
\item[ \eqref{m0:tr0} ]$t \in in  \qquad \text{(\ref{m0:leave}: [t := t' | t' = t])}$ %
\end{block}

  \item   \textbf{events}
  \begin{block}
    \item   \noindent \ref{m0:enter} [t] \textbf{event}
\begin{block}
\item \textbf{during}
\begin{block}
\item[ \eqref{m0:enterdefault} ]$\false$ %
\end{block}
\item \textbf{begin}
\begin{block}
\item[ \eqref{m0:entera1} ]$in' = in \bunion \{ t \} $ %
\end{block}
\item \textbf{end} \\
\end{block}

    \item   \noindent \ref{m0:leave} [t] \textbf{event}
\begin{block}
  \item   \textbf{during}
  \begin{block}
  \item[ \eqref{m0:leavelv:c0} ]$t \in in $ %
  \end{block}
  \item   \textbf{begin}
  \begin{block}
  \item[ \eqref{m0:leavelv:a0} ]$in \bcmsuch in' = in \setminus \{ t \} $ %
  \end{block}
  \item   \textbf{end} \\
\end{block}

  \end{block}
  \item   \textbf{end} \\
\end{block}

\end{machine}

\newcommand{\REQ}{\text{Req}}

\section{First Refinement --- Requests}
\begin{machine}{m1}
\refines{m0} \\
\newset{\REQ}

\[ \variable{pshL,pshR : \REQ \pfun \G} \]

\subsection{Liveness of push*}

\hide{ \dummy{r : \REQ}; \dummy{x : \G} }
\begin{align*}
\progress{m1:prog0}
	{ r \in \dom.pshL \1\land pshL.r = x }
	{ p < q \land qe.p = x \1\land \neg r \in \dom.pshL }
\\ \progress{m1:prog1}
	{ r \in \dom.pshR \1\land pshR.r = x }
	{ p < q \1\land qe.(q-1) = x \1\land \neg r \in \dom.pshR }
\end{align*}

\subsubsection{Proving \ref{m1:prog0}}
\indices{m0:push:left}{ r : \REQ }
\begin{align*}
\refine{m1:prog0}{ensure}{m0:push:left}{ \index{r}{r' = r} }
\end{align*}
\begin{align*}
\\ \evbcmeq{m0:push:left}{m1:a0}{pshL}{ \{ r \} \domsub pshL }
\\ \evbcmeq{m0:push:right}{m1:a0}{pshR}{ \{ r \} \domsub pshR }
\\ \evguard{m0:push:left}{m1:grd0}{ &x = pshL.r }
\\ \evguard{m0:push:left}{m1:grd1}{ &r \in \dom.pshL }
\\ \cschedule{m0:push:left}{m1:sch0}{ &r \in \dom.pshL }
\end{align*}
% \begin{align*}
% \removecoarse{m0:push:left}{default} % \weakento{m0:push:left}{default}{m1:sch0}
% \end{align*}

\subsubsection{Proving \ref{m1:prog1}}
\[ \indices{m0:push:right}{ r : \REQ } \]
\begin{align*}
\refine{m1:prog1}{ensure}{m0:push:right}{ \index{r}{r' = r} }
\end{align*}

\begin{align*}
% \removecoarse{m0:push:right}{default} % \weakento{m0:push:right}{default}{m1:sch0}
\\ \cschedule{m0:push:right}{m1:sch0}{ r \in \dom.pshR }
\\ \evguard{m0:push:right}{m1:grd0}{ r \in \dom.pshR }
\\ \evguard{m0:push:right}{m1:grd1}{ pshR.r = x }
\end{align*}

\subsection{Liveness on pop*}

\begin{align*}
\variable{ popR, popL : \set [\REQ]}
\\ \variable{ resR, resL : \REQ \pfun \G }
\\ \invariant{m1:inv0}{ \dom.resR \binter popR = \emptyset }
\\ \invariant{m1:inv1}{ \dom.resL \binter popL = \emptyset }
\end{align*}

\begin{align*}
\progress{m1:prog2}
	{ r \in popR }
	{ \neg r \in popR }
\\ \progress{m1:prog3}
	{ r \in popL }
	{ \neg r \in popL  }
\end{align*}

\subsubsection{Proving \ref{m1:prog2}}
\begin{align*}
	\refine{m1:prog2}{ensure}{m0:pop:right:non:empty,m0:pop:right:empty
	}{ \index{r}{r' = r} }
% \\  &\progress{m1:prog4}
% 	{ r \in popR \land p < q }
% 	{ \neg r \in popR \land (emp \lor (r \in \dom.resR \land resR.r
% 	 \= res))  }
% \\  &\progress{m1:prog5}
% 	{ r \in popR \land p = q }
% 	{ \neg r \in popR \land (emp \lor (r \in \dom.resR \land resR.r
% 	 \= res))  }
% \\	\refine{m1:prog4}{ensure}{m0:pop:right:non:empty}{ \index{r}{r' = r} }
% \\	\refine{m1:prog5}{ensure}{m0:pop:right:empty}{ \index{r}{r' = r} }		
	\\ \refine{m1:prog3}{ensure}{m0:pop:left:empty,m0:pop:left:non:empty}
	{}
\end{align*}
\indices{m0:pop:right:non:empty}{ r : \REQ }
\indices{m0:pop:right:empty}{ r : \REQ }
\begin{align*}
\cschedule{m0:pop:right:non:empty}{m1:sch0}{ r \in popR }
\\ \cschedule{m0:pop:right:non:empty}{m1:sch1}{ p < q }
\\ \cschedule{m0:pop:right:empty}{m1:sch0}{ r \in popR }
\\ \cschedule{m0:pop:right:empty}{m1:sch1}{ p = q }
% \\ \removecoarse{m0:pop:right:non:empty}{default} % \weakento{m0:pop:right:non:empty}{default}{m1:sch0,m1:sch1}
% \\ \removecoarse{m0:pop:right:empty}{default} % \weakento{m0:pop:right:empty}{default}{m1:sch0,m1:sch1}
\\ \evguard{m0:pop:right:non:empty}{m1:grd0}{ r \in popR }
\\ \evbcmeq{m0:pop:right:non:empty}{m1:a2}
	{popR}{ popR \setminus \{ r \} }
% \\ \evbcmeq{m0:pop:right:empty}{m1:a2}
% 	{popR}{ popR \setminus \{ r \} }
\\ \evbcmeq{m0:pop:right:non:empty}{m1:a3}
	{resR}{ resR \1| r \fun qe.(q-1) }
\end{align*}


\subsubsection{Proving \ref{m1:prog3}}
\begin{align*}
% \refine{m1:prog3}{disjunction}{m1:prog4,m1:prog5}{}
% \\ \progress{m1:prog4}
% 	{r \in popL \land p < q }
% 	{ \neg r \in popL \land r \in \dom.resL \land resL.r = res }
% \\ \progress{m1:prog5}
% 	{r \in popL \land p = q }
% 	{ \neg r \in popL \land emp  }
% \\ \refine{m1:prog4}{ensure}{m0:pop:left}{ \index{r}{r' = r} }
\refine{m1:prog3}{ensure}{m0:pop:left:non:empty}{ \index{r}{r' = r} }
\end{align*}
\indices{m0:pop:left:non:empty}{ r : \REQ }
\indices{m0:pop:left:empty}{ r : \REQ }
% theorem prover found that qe.(q-1) (from the event pop:right),
% should be changed for qe.p
\begin{align*}
\cschedule{m0:pop:left:non:empty}{m1:sch0}{ r \in popL }
\\ \cschedule{m0:pop:left:non:empty}{m1:sch1}{ p < q }
% \\ \removecoarse{m0:pop:left:non:empty}{default} % \weakento{m0:pop:left:non:empty}{default}{m1:sch0,m1:sch1}
\\ \evguard{m0:pop:left:non:empty}{m1:grd0}{ r \in popL }
\\ \evbcmeq{m0:pop:left:non:empty}{m1:a4}
	{popL}{ popL \setminus \{ r \} }
% \\ \evbcmeq{m0:pop:left:empty}{m1:a4}
% 	{popL}{ popL \setminus \{ r \} }
\\ \evbcmeq{m0:pop:left:non:empty}{m1:a5}
	{resL}{ resL \1 | r \fun qe.p }
	%
	%
	% other events
\end{align*}

\subsection{Initialization of \ref{m1:inv0} and \ref{m1:inv1}}

\begin{align*}
\initialization{m1:init0}{resR = \emptyfun}
\\ \initialization{m1:init1}{resL = \emptyfun}
\end{align*}

\comment{\ref{m1:inv0}}{this is for fun}

\comment{$emp$}{this encodes whether the last call to pop found the
queue empty ; more text, more text, still more text, and here's more
text}

\begin{block}
  \item   \textbf{machine} m1
  \item   \textbf{variables}
  \begin{block}
    \item   $in$
    \item   $loc$
  \end{block}
  \item   %!TEX root=../main8.tex
\textbf{invariants}
\begin{block}
\item[ \eqref{m1:inv0} ]{$qe \in \intervalR{p}{q} \tfun \G $} %
\item[ \eqref{m1:inv1} ]{$p \le q $} %
\end{block}

  \item   \textbf{progress}
\begin{block}
\item[ \eqref{m1:prog0} ]$r \in \dom.pshL \1\land pshL.r = x  \quad \mapsto\quad p < q \land qe.p = x \1\land \neg r \in \dom.pshL $ %
\item[ \eqref{m1:prog1} ]$r \in \dom.pshR \1\land pshR.r = x  \quad \mapsto\quad p < q \1\land qe.(q-1) = x \1\land \neg r \in \dom.pshR $ %
\item[ \eqref{m1:prog2} ]$r \in popR  \quad \mapsto\quad \neg r \in popR $ %
\item[ \eqref{m1:prog3} ]$r \in popL  \quad \mapsto\quad \neg r \in popL  $ %
\end{block}

  \item   \textbf{safety}
\begin{block}
\item[ \eqref{m1:saf0} ]$\neg t \in in  \textbf{\quad unless \quad} t \in in \land loc.t = ent $ %
\item[ \eqref{m1:saf1} ]$t \in in \land loc.t = ent  \textbf{\quad unless \quad} t \in in \land loc.t \in plf $ %
\item[ \eqref{m1:saf2} ]$t \in in \land loc.t \in plf  \textbf{\quad unless \quad} t \in in \land loc.t = ext $ %
\item[ \eqref{m1:saf3} ]$t \in in \land loc.t = ext  \textbf{\quad unless \quad} \neg t \in in $ %
\end{block}

  \item   %!TEX root=../puzzle.tex

  \item   \textbf{events}
  \begin{block}
    \item   \noindent \ref{m0:enter} [t] \textbf{event}
\begin{block}
  \item   \textbf{when}
  \begin{block}
  \item[ \eqref{m0:enterent:grd1} ]$\neg t \in in $ %
  \end{block}
  \item   \textbf{begin}
  \begin{block}
  \item[ \eqref{m0:entera1} ]$in \bcmsuch in' = in \bunion \{ t \} $ %
  \item[ \eqref{m0:entera3} ]$loc \bcmsuch loc' = loc \1| t \fun ent $ %
  \end{block}
  \item   \textbf{end} \\
\end{block}

    \item   \noindent \ref{m0:leave} [t] \textbf{event}
\begin{block}
  \item   \textbf{during}
  \begin{block}
  \item[ \eqref{m0:leavelv:c0} ]$t \in in $ %
  \item[ \eqref{m0:leavelv:c1} ]$loc.t = ext $ %
  \end{block}
  \item   \textbf{when}
  \begin{block}
  \item[ \eqref{m0:leavelv:grd0} ]$t \in in $ %
  \item[ \eqref{m0:leavelv:grd1} ]$loc.t = ext $ %
  \end{block}
  \item   \textbf{begin}
  \begin{block}
  \item[ \eqref{m0:leavelv:a0} ]$in \bcmsuch in' = in \setminus \{ t \} $ %
  \item[ \eqref{m0:leavelv:a2} ]$loc \bcmsuch loc' = \{ t \} \domsub loc $ %
  \end{block}
  \item   \textbf{end} \\
\end{block}

    \item   \noindent \ref{m1:movein} [t] \textbf{event}
\begin{block}
  \item   \textbf{during}
  \begin{block}
  \item[ \eqref{m1:moveinmi:c1} ]$t \in in $ %
  \item[ \eqref{m1:moveinmi:c2} ]$loc.t = ent $ %
  \end{block}
  \item   \textbf{any} b
  \item   \textbf{when}
  \begin{block}
  \item[ \eqref{m1:moveinmi:g1} ]$t \in in $ %
  \item[ \eqref{m1:moveinmi:grd0} ]$loc.t = ent $ %
  \item[ \eqref{m1:moveinmi:grd7} ]$b \in plf $ %
  \end{block}
  \item   \textbf{begin}
  \begin{block}
  \item[ \eqref{m1:moveinmi:a2} ]$loc \bcmsuch loc' = loc \1| t \fun b $ %
  \end{block}
  \item   \textbf{end} \\
\end{block}

    \item   \noindent \ref{m1:moveout} [t] \textbf{event}
\begin{block}
\item \textbf{during}
\begin{block}
\item[ \eqref{m1:moveoutc1} ]$t \in in  %
		\1\land loc.t \in plf $ %
\end{block}
\item \textbf{when}
\begin{block}
\item[ \eqref{m1:moveoutmo:g1} ]$t \in in $ %
\item[ \eqref{m1:moveoutmo:g2} ]$loc.t \in plf $ %
\end{block}
\item \textbf{begin}
\begin{block}
\item[ \eqref{m1:moveoutSKIP:in} ]$in' = in$ %
\item[ \eqref{m1:moveouta2} ]$loc' = loc \1 | t \fun ext $ %
\end{block}
\item \textbf{end} \\
\end{block}

  \end{block}
  \item   \textbf{end} \\
\end{block}

\end{machine}
 
% \noindent \ref{m0:push:right} [r] \textbf{event}
\begin{block}
  \item   \textbf{during}
  \begin{block}
  \item[ \eqref{m0:push:rightm1:sch0} ]$r \in \dom.pshR $ %
  \end{block}
  \item   \textbf{any} x
  \item   \textbf{when}
  \begin{block}
  \item[ \eqref{m0:push:rightm1:grd0} ]$r \in \dom.pshR $ %
  \item[ \eqref{m0:push:rightm1:grd1} ]$pshR.r = x $ %
  \end{block}
  \item   \textbf{begin}
  \begin{block}
  \item[ \eqref{m0:push:rightm0:act0} ]$qe \bcmeq qe \2| q \fun x $ %
  \item[ \eqref{m0:push:rightm0:act2} ]$q \bcmeq q+1 $ %
  \item[ \eqref{m0:push:rightm1:a0} ]$pshR \bcmeq \{ r \} \domsub pshR $ %
  \end{block}
  \item   \textbf{end} \\
\end{block}

% \noindent \ref{m0:push:left} [r] \textbf{event}
\begin{block}
  \item   \textbf{during}
  \begin{block}
  \item[ \eqref{m0:push:leftm1:sch0} ]$r \in \dom.pshL $ %
  \end{block}
  \item   \textbf{any} x
  \item   \textbf{when}
  \begin{block}
  \item[ \eqref{m0:push:leftm1:grd0} ]$x = pshL.r $ %
  \item[ \eqref{m0:push:leftm1:grd1} ]$r \in \dom.pshL $ %
  \end{block}
  \item   \textbf{begin}
  \begin{block}
  \item[ \eqref{m0:push:leftm0:act0} ]$qe \bcmeq qe \2| p\0-1 \fun x $ %
  \item[ \eqref{m0:push:leftm0:act1} ]$p \bcmeq p-1 $ %
  \item[ \eqref{m0:push:leftm1:a0} ]$pshL \bcmeq \{ r \} \domsub pshL $ %
  \end{block}
  \item   \textbf{end} \\
\end{block}

% \noindent \ref{m0:pop:left} [r] \textbf{event}
\begin{block}
  \item   \textbf{during}
  \begin{block}
  \item[ \eqref{m0:pop:leftm1:sch0} ]$r \in popL $ %
  \end{block}
  \item   \textbf{when}
  \begin{block}
  \item[ \eqref{m0:pop:leftm1:grd0} ]$r \in popL $ %
  \end{block}
  \item   \textbf{begin}
  \begin{block}
  \item[ \eqref{m0:pop:leftm0:act0} ]$qe \bcmeq \{ p \} \domsub qe $ %
  \item[ \eqref{m0:pop:leftm0:act1} ]$p \bcmsuch (p = q \land p' = p) \lor (p < q \land p' = p\0+1) $ %
  \item[ \eqref{m0:pop:leftm0:act3} ]$res \bcmsuch (p = q \land res' = res) \lor (p < q \land res' = qe.p) $ %
  \item[ \eqref{m0:pop:leftm0:act4} ]$emp \bcmeq (p = q) $ %
  \item[ \eqref{m0:pop:leftm1:a4} ]$popL \bcmeq popL \setminus \{ r \} $ %
  \item[ \eqref{m0:pop:leftm1:a5} ]$resL \bcmsuch (p = q \land resL' = resL)  %
  	\lor (p < q \land resL' = resL | r \fun qe.p) $ %
  \end{block}
  \item   \textbf{end} \\
\end{block}


% \section{Memory nodes}

% \newcommand{\Cell}{\text{Cell}}
% \begin{machine}{m2}
% \refines{m1}

% \newset{\Cell}

% \newcommand{\injective}{\text{injective}}

% \begin{align*} 
% 	\variable{ obj &: \Int \pfun \Cell } 
% \\	\variable{ val &: \Cell \pfun \G } 
% \\ 	\variable{ node, live &: \set [ \Cell ] }
% \\ 	\invariant{m2:inv0}{ obj &\in \intervalR{p}{q} \tfun node }
% \\ 	\invariant{m2:inv1}{ val &\in live \tfun \G }
% \\ 	\invariant{m2:inv3}{ node &\subseteq live }
% \\ 	\invariant{m2:inv4}{ \injective&.obj }
% \end{align*}

% Note: $obj$ should be injective and surjective.

% Now \emph{la pi\`ece de r\'esistence}:

% \begin{align*}
% \invariant{m2:inv2}{ \qforall{i}{ \betweenR{p}{i}{q} }{ ~ qe.i =
% val.(obj.i) } }
% \end{align*}

% Notice that $i$ has not been declared as a dummy at this point; its
% type is simply inferred from the context.

% Note: Invariant \ref{m2:inv3} has been dictated (at this point of
% writing the model) by a WD proof obligation.

% \subsection{Invariance in \ref{m0:push:left}}

% \param{m0:push:left}{ n : \Cell }

% \ref{m2:inv0}:
% \begin{align*}
% \evassignment{m0:push:left}{m2:a0}{ obj' &= obj | (p-1) \fun n }
% \\ \evassignment{m0:push:left}{m2:a1}{ node' &= node \bunion \{ n \}
% }
% \end{align*}

% \ref{m2:inv1}:
% \begin{align*}
% \evassignment{m0:push:left}{m2:a2}{ val' = val }
% \\ \evassignment{m0:push:left}{m2:a3}{ live' = live }
% \end{align*}

% \ref{m2:inv3}
% \begin{align*}
% \evguard{m0:push:left}{m2:grd0}{ n \in live \setminus node }
% \end{align*}

% \ref{m2:inv2}:
% \begin{align*}
% \evguard{m0:push:left}{m2:grd1}{ val.n = x }
% % \evassignment{m0:push:left}{m2:a4}{  }
% \end{align*}

% \ref{m2:inv4}:
% \begin{align*}
% % \evguard{m0:push:left}{m2:grd1}{ val.n = x }
% % \evassignment{m0:push:left}{m2:a4}{  }
% \end{align*}

% Schedulability:
% \begin{align*}
% \cschedule{m0:push:left}{m2:sch0}{ pshL.r \in \ran.val }
% \\ \cschedule{m0:push:left}{m2:sch1}
% 	{ pshL.r \in \ran.(node \domsub val) }
% \end{align*}

% \replace{m0:push:left}{}{m2:sch0,m2:sch1}{m1:sch0}{m2:prog0}
% 	{m2:saf0}

% \begin{align*}
% \progress{m2:prog0}
% 	{ r \in \dom.pshL }
% 	{ r \in \dom.pshL \land pshL.r \in \ran.(node \domsub val) }
% \\ \safety{m2:saf0}{ r \in \dom.pshL \land pshL.r \in \ran.(node
% \domsub val) }
% 	{ \neg r \in \dom.pshL }
% \end{align*}

% \[ \variable{insL : \REQ \pfun \Cell }  \]

% \begin{align*}
% 	\invariant{m2:inv5}{ insL \in (\dom.pshL) \tfun live }
% \\	\invariant{m2:inv6}
% 		{ \qforall{r}
% 			{ r \in \dom.pshL }
% 			{ pshL.r = val.(insL.r) } }
% \\ 	\invariant{m2:inv7}{ \injective.insL }
% \\ 	\invariant{m2:inv8}{ \ran.insL \binter node = \emptyset }
% \\  \evassignment{m0:push:left}{m2:act0}
%			{ insL' = \{ r \} \domsub insL }
% \end{align*}

% \newcommand{\finter}{\binter}
% \begin{proof}[\ref{m0:push:left}/SAF/\ref{m2:saf0}]
% 	\begin{free:var}{r}{r_0}
% 	TODO: What are we actually proving?
% 	\begin{by:cases}
% 	The first case,
% 	\begin{case}{hyp0}{r = r_0}
% 		follows directly from the fact that \ref{m0:push:left}
% 		removes $r$ from the domain of $pshL$ hence establishing the
% 		RHS of \eqref{m2:saf0}.
% 		\easy
% 	\end{case}		
% 	\begin{case}{hyp1}{\neg r = r_0}
% 		Assuming:
% 		\begin{align*}
% 		\assume{hyp2}{ & r_0 \in \dom.pshL}
% 		% \\ \assume{hyp2}{ & r_0 \in \dom.pshL}
% 		\end{align*}
% 		We need to prove:
% 		\begin{align*}
% 		\\ \goal{& r_0 \in \dom.pshL' 
% 		\\	\land ~ & pshL'.r_0 \in \ran.(node' \domsub val') }
% 		\end{align*}
% 		\begin{by:parts}
% 		\begin{part:a}{ r_0 \in \dom.pshL' }
% 		This follows from \eqref{hyp1} and the fact that
% 		\ref{m0:push:left} only changes the domain of $pshL'$ at
% 		$r$.
% 			\easy
% 		\end{part:a}
% 		\begin{part:a}{pshL'.r_0 \in \ran.(node' \domsub val')}
% 		% \easy
% 		\begin{calculation}
% 			pshL'.r_0 \in \ran.(node' \domsub val')
% 		\hint{=}{ \eqref{m2:a2} }
% 			pshL'.r_0 \in \ran.(node'  \domsub val)
% 		\hint{=}{ \eqref{m1:a0} }
% 			(\{ r \} \domsub pshL).r_0 \in 
% 			\ran.(node' \domsub val)
% 		\hint{=}{ \eqref{hyp1} }
% 			pshL.r_0 \in \ran.(node' \domsub val)
% 		\hint{=}{ \eqref{m2:inv6} with \eqref{hyp2} }
% 			val.(insL.r_0) \in \ran.(node' \domsub val)
% 		% \hint{\follows}{ $.$ and }
% 		\hint{\follows}{  }
% 			insL.r_0 \in \dom.val \setminus node'
% 		\hint{=}{ \eqref{m2:a1} }
% 			insL.r_0 \in 
% 					\dom.val \setminus (node \bunion \{ n \}) 
% 		\hint{=}{ \eqref{m2:inv1} ,
% 				  \eqref{m2:inv6} ,
% 				  \eqref{m2:grd2} }
% 			insL.r_0 \in live \setminus (node \bunion \{ insL.r \}) 
% 			% \neg insL.r_0 \1\in node \bunion \{ insL.r \}
% 		\hint{=}{ 	\eqref{hyp2} ,
% 					\eqref{m2:inv5} }
% 			\neg insL.r_0 \in node \1\land \neg insL.r_0 = insL.r
% 		\hint{=}{ \eqref{m2:inv7}, 
% 				\eqref{hyp1}, 
% 				\eqref{hyp2}, 
% 				\eqref{m1:grd1},
% 				\eqref{m2:inv5} }
% 			\neg insL.r_0 \in node
% 		\hint{\follows}{ \eqref{m2:inv8}; 
% 				\eqref{hyp2}; 
% 				\eqref{m2:inv5} }
% 			\true
% 		% \hint{=}{ \eqref{m2:a2} \eqref{m2:a1} }
%		% 	pshL'.r_0 \in \ran.((node\domsub val) \finter (\{ n
%		% 	\}\domsub val))
% 		\end{calculation}
% 		\end{part:a}
% 		\end{by:parts}
% 	\end{case}		
% 	\end{by:cases}
% 	\end{free:var}
% \end{proof}

% \begin{align*}
% 	\evguard{m0:push:left}{m2:grd2}{ n = insL.r }
% \end{align*}

% We defer the proof of \ref{m2:prog0} to finish proving our initial
% invariants.

% \subsection{Invariance in \ref{m0:pop:right}}

% \param{m0:pop:right}{ n : \Cell }

% \ref{m2:inv0}:
% \begin{align*}
% \evassignment{m0:pop:right}{m2:a0}
% 	{ 	  (p = q \land obj' = obj)
% 	&\lor (p < q \land obj' = \{ q-1 \} \domsub obj) }
% \\ \evassignment{m0:pop:right}{m2:a1}
% 	{ 	 (p = q \land node' = node)
% 	&\lor (p < q \land node' = node \setminus \{ obj.(q\0-1) \}) }
% \end{align*}

% \ref{m2:inv1}:
% \begin{align*}
% \evassignment{m0:pop:right}{m2:a2}{ val' = val }
% \\ \evassignment{m0:pop:right}{m2:a3}{ live' = live }
% \end{align*}

% \ref{m2:inv5}, \ref{m2:inv6}, \ref{m2:inv7} and \ref{m2:inv8}:
% \begin{align*}
% % \evguard{m0:pop:right}{m2:grd1}{ val.n = x }
% \evassignment{m0:pop:right}{m2:a4}{ insL' = insL }
% \end{align*}

% \ref{m2:saf0}
% \begin{align*}
% % \evguard{m0:pop:right}{m2:grd1}{ val.n = x }
% % \evassignment{m0:pop:right}{m2:a4}{ insL' = insL }
% \end{align*}

% \begin{proof}[\ref{m0:pop:right}/SAF/\ref{m2:saf0}]
% \begin{free:var}{r}{r_0}
% \begin{by:cases}
% \begin{case}{pr2:hyp0}{p = q}
% 	\easy
% \end{case}
% \begin{case}{pr2:hyp1}{\neg p = q}
% 	Assuming:
% 	\begin{align*}
% 		\assume{pr2:hyp2}{r_0 & \in \dom.pshL}
% 	\\  \assume{pr2:hyp3}{pshL.r_0 & \in \ran.(node \domsub val)}
% 	\end{align*}
% 	Prove:
% 	\[ \goal{pshL'.r_0 \in \ran.(node' \domsub val')} \]
% 	% \easy
% 	\begin{calculation}
% 		pshL'.r_0 \in \ran.(node' \domsub val')
% 		% this step should not be marked as proved
% 	\hint{=}{ \eqref{m1:a0} and \eqref{m2:a2} }
% 		pshL.r_0 \in \ran.(node' \domsub val)
% 	\hint{=}{ \eqref{m2:inv6} with \eqref{pr2:hyp2} }
% 		val.(insL.r_0) \in \ran.(node' \domsub val)
% 		% this step should not be marked as proved
% 	\hint{\follows}{ $.$ and $\ran$ with \eqref{pr2:hyp2} }
% 		insL.r_0 \in \dom.val \setminus node'
% 	% \hint{=}{ \eqref{m2:inv5} }
% 	% 	val.()
% 	\hint{=}{ \eqref{m2:a1} }
% 		insL.r_0 \in \dom.val \setminus (node \setminus \{ obj.(q\0-1) \})
% 	\hint{=}{  }
		% insL.r_0 \in (\dom.val \setminus node) \bunion (\dom.val
		% \binter \{ obj.(q\0-1) \})
% 	% \hint{=}{ $obj.(q\0-1) \in \dom.val$ }
% 	% 	insL.r_0 \in (\dom.val \setminus node) \bunion \{ obj.(q\0-1) \}
% 	\hint{\follows}{  }
% 		insL.r_0 \in \dom.val \setminus node
% 	\hint{=}{ \stmt{insL.r_0 \in \dom.val} }
% 		\true
% 	\end{calculation}
% \end{case}
% \end{by:cases}
% \end{free:var}
% \end{proof}

% Schedulability:
% \begin{align*}
% % \cschedule{m0:pop:right}{m2:sch0}{ pshL.r \in \ran.val }
% \end{align*}
% % \replace{m0:pop:right}{}{m2:sch0}{m1:sch0}{m2:prog0}{m2:saf0}
% \begin{align*}
% % \progress{m2:prog0}{ r \in \dom.pshL }
% %			 { r \in \dom.pshL \land pshL.r
% %     \in \ran.val }
% % \\ \safety{m2:saf0}{ r \in \dom.pshL \land pshL.r \in \ran.val }{
% % \\ \\neg r \in \dom.pshL }
% \end{align*}

% \subsection{ Establishing the invariants }

% \begin{align*}
% 	\initialization{m2:init0}{ obj &= \emptyfun }
% \\	\initialization{m2:init1}{ node &= \emptyset }
% \\  \initialization{m2:init2}{ val &= \emptyfun }
% \\	\initialization{m2:init3}{ live &= \emptyset }
% \end{align*}


% \subsection{\ref{m0:push:right} and \ref{m0:pop:left}}
% Push right and pop left are the mirror image of push left and pop
% right and their proofs are left for an appendix.

% TODO: add new events to m1 and m2

% \end{machine}

% \noindent \ref{m0:pop:right} [r] \textbf{event}
\begin{block}
  \item   \textbf{during}
  \begin{block}
  \item[ \eqref{m0:pop:rightm1:sch0} ]$r \in popR $ %
  \end{block}
  \item   \textbf{any} n
  \item   \textbf{when}
  \begin{block}
  \item[ \eqref{m0:pop:rightm1:grd0} ]$r \in popR $ %
  \end{block}
  \item   \textbf{begin}
  \begin{block}
  \item[ \eqref{m0:pop:rightm0:act0} ]$qe \bcmeq \{ q-1 \} \domsub qe $ %
  \item[ \eqref{m0:pop:rightm0:act2} ]$q \bcmsuch (p = q \land q' = q) \lor (p < q \land q' = q\0-1) $ %
  \item[ \eqref{m0:pop:rightm0:act3} ]$res \bcmsuch (p = q \land res' = res) \lor (p < q \land res' = qe.(q\1-1)) $ %
  \item[ \eqref{m0:pop:rightm0:act4} ]$emp \bcmeq (p = q) $ %
  \item[ \eqref{m0:pop:rightm1:a2} ]$popR \bcmeq popR \setminus \{ r \} $ %
  \item[ \eqref{m0:pop:rightm1:a3} ]$resR \bcmsuch (p = q \land resR' = resR)  %
  	\lor (p < q \land resR' = resR | r \fun qe.(q-1) ) $ %
  \item[ \eqref{m0:pop:rightm2:a0} ]$obj \bcmsuch (p = q \land obj' = obj) %
  	\lor (p < q \land obj' = \{ q-1 \} \domsub obj) $ %
  \item[ \eqref{m0:pop:rightm2:a1} ]$node \bcmsuch (p = q \land node' = node) %
  	\lor (p < q \land node' = node \setminus \{ obj.(q\0-1) \}) $ %
  \end{block}
  \item   \textbf{end} \\
\end{block}

% \noindent \ref{m0:push:left} [r] \textbf{event}
\begin{block}
  \item   \textbf{during}
  \begin{block}
  \item[ \eqref{m0:push:leftm1:sch0} ]$r \in \dom.pshL $ %
  \item[ \eqref{m0:push:leftm2:sch0} ]$pshL.r \in \ran.val $ %
  \item[ \eqref{m0:push:leftm2:sch1} ]$pshL.r \in \ran.(node \domsub val) $ %
  \end{block}
  \item   \textbf{any} n,x
  \item   \textbf{when}
  \begin{block}
  \item[ \eqref{m0:push:leftm1:grd0} ]$x = pshL.r $ %
  \item[ \eqref{m0:push:leftm1:grd1} ]$r \in \dom.pshL $ %
  \item[ \eqref{m0:push:leftm2:grd0} ]$n \in live \setminus node $ %
  \item[ \eqref{m0:push:leftm2:grd1} ]$val.n = x $ %
  \item[ \eqref{m0:push:leftm2:grd2} ]$n = insL.r $ %
  \end{block}
  \item   \textbf{begin}
  \begin{block}
  \item[ \eqref{m0:push:leftm0:act0} ]$qe \bcmeq qe \2| p\0-1 \fun x $ %
  \item[ \eqref{m0:push:leftm0:act1} ]$p \bcmeq p-1$ %
  \item[ \eqref{m0:push:leftm1:a0} ]$pshL \bcmeq \{ r \} \domsub pshL $ %
  \item[ \eqref{m0:push:leftm2:a0} ]$obj \bcmeq obj | (p-1) \fun n $ %
  \item[ \eqref{m0:push:leftm2:a1} ]$node \bcmeq node \bunion \{ n \} $ %
  \item[ \eqref{m0:push:leftm2:act0} ]$insL \bcmeq \{ r \} \domsub insL $ %
  \end{block}
  \item   \textbf{end} \\
\end{block}


\end{document}













