\documentclass[12pt]{amsart}
\usepackage[margin=0.5in]{geometry} 
	% see geometry.pdf on how to lay out the page. There's lots.
\usepackage{bsymb}
\usepackage{../unitb}
\usepackage{calculation}
\usepackage{ulem}
\usepackage{hyperref}
\normalem
\geometry{a4paper} % or letter or a5paper or ... etc
% \geometry{landscape} % rotated page geometry

% See the ``Article customise'' template for some common
% customisations

\title{}
\author{}
\date{} % delete this line to display the current date

%%% BEGIN DOCUMENT
\setcounter{tocdepth}{4}
\begin{document}

\maketitle
\tableofcontents

% \begin{block}
  \item   \textbf{machine} m1
  \item   \textbf{variables}
  \begin{block}
    \item   $in$
    \item   $loc$
  \end{block}
  \item   %!TEX root=../main8.tex
\textbf{invariants}
\begin{block}
\item[ \eqref{m1:inv0} ]{$qe \in \intervalR{p}{q} \tfun \G $} %
\item[ \eqref{m1:inv1} ]{$p \le q $} %
\end{block}

  \item   \textbf{progress}
\begin{block}
\item[ \eqref{m1:prog0} ]$r \in \dom.pshL \1\land pshL.r = x  \quad \mapsto\quad p < q \land qe.p = x \1\land \neg r \in \dom.pshL $ %
\item[ \eqref{m1:prog1} ]$r \in \dom.pshR \1\land pshR.r = x  \quad \mapsto\quad p < q \1\land qe.(q-1) = x \1\land \neg r \in \dom.pshR $ %
\item[ \eqref{m1:prog2} ]$r \in popR  \quad \mapsto\quad \neg r \in popR $ %
\item[ \eqref{m1:prog3} ]$r \in popL  \quad \mapsto\quad \neg r \in popL  $ %
\end{block}

  \item   \textbf{safety}
\begin{block}
\item[ \eqref{m1:saf0} ]$\neg t \in in  \textbf{\quad unless \quad} t \in in \land loc.t = ent $ %
\item[ \eqref{m1:saf1} ]$t \in in \land loc.t = ent  \textbf{\quad unless \quad} t \in in \land loc.t \in plf $ %
\item[ \eqref{m1:saf2} ]$t \in in \land loc.t \in plf  \textbf{\quad unless \quad} t \in in \land loc.t = ext $ %
\item[ \eqref{m1:saf3} ]$t \in in \land loc.t = ext  \textbf{\quad unless \quad} \neg t \in in $ %
\end{block}

  \item   %!TEX root=../puzzle.tex

  \item   \textbf{events}
  \begin{block}
    \item   \noindent \ref{m0:enter} [t] \textbf{event}
\begin{block}
  \item   \textbf{when}
  \begin{block}
  \item[ \eqref{m0:enterent:grd1} ]$\neg t \in in $ %
  \end{block}
  \item   \textbf{begin}
  \begin{block}
  \item[ \eqref{m0:entera1} ]$in \bcmsuch in' = in \bunion \{ t \} $ %
  \item[ \eqref{m0:entera3} ]$loc \bcmsuch loc' = loc \1| t \fun ent $ %
  \end{block}
  \item   \textbf{end} \\
\end{block}

    \item   \noindent \ref{m0:leave} [t] \textbf{event}
\begin{block}
  \item   \textbf{during}
  \begin{block}
  \item[ \eqref{m0:leavelv:c0} ]$t \in in $ %
  \item[ \eqref{m0:leavelv:c1} ]$loc.t = ext $ %
  \end{block}
  \item   \textbf{when}
  \begin{block}
  \item[ \eqref{m0:leavelv:grd0} ]$t \in in $ %
  \item[ \eqref{m0:leavelv:grd1} ]$loc.t = ext $ %
  \end{block}
  \item   \textbf{begin}
  \begin{block}
  \item[ \eqref{m0:leavelv:a0} ]$in \bcmsuch in' = in \setminus \{ t \} $ %
  \item[ \eqref{m0:leavelv:a2} ]$loc \bcmsuch loc' = \{ t \} \domsub loc $ %
  \end{block}
  \item   \textbf{end} \\
\end{block}

    \item   \noindent \ref{m1:movein} [t] \textbf{event}
\begin{block}
  \item   \textbf{during}
  \begin{block}
  \item[ \eqref{m1:moveinmi:c1} ]$t \in in $ %
  \item[ \eqref{m1:moveinmi:c2} ]$loc.t = ent $ %
  \end{block}
  \item   \textbf{any} b
  \item   \textbf{when}
  \begin{block}
  \item[ \eqref{m1:moveinmi:g1} ]$t \in in $ %
  \item[ \eqref{m1:moveinmi:grd0} ]$loc.t = ent $ %
  \item[ \eqref{m1:moveinmi:grd7} ]$b \in plf $ %
  \end{block}
  \item   \textbf{begin}
  \begin{block}
  \item[ \eqref{m1:moveinmi:a2} ]$loc \bcmsuch loc' = loc \1| t \fun b $ %
  \end{block}
  \item   \textbf{end} \\
\end{block}

    \item   \noindent \ref{m1:moveout} [t] \textbf{event}
\begin{block}
\item \textbf{during}
\begin{block}
\item[ \eqref{m1:moveoutc1} ]$t \in in  %
		\1\land loc.t \in plf $ %
\end{block}
\item \textbf{when}
\begin{block}
\item[ \eqref{m1:moveoutmo:g1} ]$t \in in $ %
\item[ \eqref{m1:moveoutmo:g2} ]$loc.t \in plf $ %
\end{block}
\item \textbf{begin}
\begin{block}
\item[ \eqref{m1:moveoutSKIP:in} ]$in' = in$ %
\item[ \eqref{m1:moveouta2} ]$loc' = loc \1 | t \fun ext $ %
\end{block}
\item \textbf{end} \\
\end{block}

  \end{block}
  \item   \textbf{end} \\
\end{block}

% %!TEX root=../main9.tex
\begin{block}
  \item   \textbf{machine} m2
  \item   \textbf{refines} m0
  \item   \textbf{variables}
  \begin{block}
    \item   $req$
    \item   $req0$
    \item   $reqA$
    \item   $reqB$
  \end{block}
  \item   %!TEX root=../main9.tex
\textbf{invariants}
\begin{block}
\item[ \eqref{m1:inv0} ]{$reqA \bunion reqB = req$} %
\item[ \eqref{m1:inv1} ]{$reqA \binter reqB = req$} %
\end{block}

  \item   \textbf{events}
  \begin{block}
    \item   %!TEX root=../main9.tex
\noindent \ref{handle}  \textbf{event}
\begin{block}
  \item   \textbf{during}
  \begin{block}
  \item[ \eqref{handlem0:sch0} ]{$\neg req = \emptyset$} %
  \end{block}
  \item   \textbf{any} r
  \item   \textbf{when}
  \begin{block}
  \item[ \eqref{handlegrd0} ]{$r \in req$} %
  \end{block}
  \item   \textbf{begin}
  \begin{block}
  \item[ \eqref{handleact0} ]{$req \bcmeq req \setminus \{ r \}$} %
  \item[ \eqref{handleact1} ]{$req0 \bcmeq req$} %
  \end{block}
  \item   \textbf{end} \\
\end{block}

    \item   %!TEX root=../main9.tex
\noindent \ref{req}  \textbf{event}
\begin{block}
  \item   \textbf{during}
  \begin{block}
  \item[ (\ref{req}/default) ]{$\false$} %
  \end{block}
  \item   \textbf{any} r
  \item   \textbf{when}
  \begin{block}
  \item[ \eqref{reqgrd0} ]{$\neg r \in req$} %
  \end{block}
  \item   \textbf{begin}
  \begin{block}
  \item[ \eqref{reqact0} ]{$req \bcmeq req \bunion \{ r \}$} %
  \item[ \eqref{reqact1} ]{$req0 \bcmeq req$} %
  \end{block}
  \item   \textbf{end} \\
\end{block}

  \end{block}
  \item   \textbf{end} \\
\end{block}

\newcommand{\stmt}[1]{ $#1$ }
% \newcommand{\comment}[2]{ 
% 	% \begin{tabular}{|l|l|} 
% 	% 	\hline
% 		\begin{description}
% 		\item[{#1}] ~ #2 
% 		\end{description}
% 	% 	\hline 
% 	% \end{tabular} 
% 	}
\newcommand{\commentbox}[1]{ 
	\begin{block} \item \small{ #1 } \end{block} }
\newcommand{\G}{\text{Val}}
\newcommand{\Req}{\text{Req}}
% \renewcommand{\H}{\text{H}}

\section{Initial model --- Requests}
\begin{block}
  \item   \textbf{machine} m0
  \item   \textbf{variables}
  \begin{block}
    \item   $in$
  \end{block}
  \item   \textbf{progress}
\begin{block}
\item[ \eqref{m0:prog0} ]$t \in in  \quad \mapsto\quad \neg t \in in $ %
\end{block}

  \item   \textbf{transient}
\begin{block}
\item[ \eqref{m0:tr0} ]$t \in in  \qquad \text{(\ref{m0:leave}: [t := t' | t' = t])}$ %
\end{block}

  \item   \textbf{events}
  \begin{block}
    \item   \noindent \ref{m0:enter} [t] \textbf{event}
\begin{block}
\item \textbf{during}
\begin{block}
\item[ \eqref{m0:enterdefault} ]$\false$ %
\end{block}
\item \textbf{begin}
\begin{block}
\item[ \eqref{m0:entera1} ]$in' = in \bunion \{ t \} $ %
\end{block}
\item \textbf{end} \\
\end{block}

    \item   \noindent \ref{m0:leave} [t] \textbf{event}
\begin{block}
  \item   \textbf{during}
  \begin{block}
  \item[ \eqref{m0:leavelv:c0} ]$t \in in $ %
  \end{block}
  \item   \textbf{begin}
  \begin{block}
  \item[ \eqref{m0:leavelv:a0} ]$in \bcmsuch in' = in \setminus \{ t \} $ %
  \end{block}
  \item   \textbf{end} \\
\end{block}

  \end{block}
  \item   \textbf{end} \\
\end{block}

\begin{machine}{m0}
	\with{functions}
	\with{sets}
	\newset{\G} \newset{\Req}
	\newevent{req:push:left}{request\_push\_left}
	\newevent{req:push:right}{request\_push\_right}
	\newevent{req:pop:left}{request\_pop\_left}
	\newevent{req:pop:right}{request\_pop\_right}
	\newevent{resp:push:left}{respond\_push\_left}
	\newevent{resp:push:right}{respond\_push\_right}
	\newevent{resp:pop:left}{respond\_pop\_left}
	\newevent{resp:pop:right}{respond\_pop\_right}
	\[\variable{ppL,ppR : \set[\Req] } \]
	\[\variable{psL,psR : \Req \pfun \G }\]
	\[ \dummy{r : \Req} \]
	\[ \dummy{X : \G} \]
    \[ \param{req:push:left}{r : \Req} \]
    \[ \param{req:push:right}{r : \Req} \]
    \[ \param{req:push:left}{x : \G} \]
    \[ \param{req:push:right}{x : \G} \]
    \[ \param{req:pop:left}{r : \Req} \]
    \[ \param{req:pop:right}{r : \Req} \]
\subsection{Issuing requests}

\begin{align}
	&\evguard{req:push:left}{m0:grd0}{ \neg r \in \dom.psL } \\
	&\evbcmeq{req:push:left}{m0:act0}{psL}{ psL \1| r \fun x } \\
	&\evguard{req:push:right}{m0:grd0}{ \neg r \in \dom.psR } \\
	&\evbcmeq{req:push:right}{m0:act0}{psR}{ psR \1| r \fun x } \\
	&\evguard{req:pop:left}{m0:grd0}{ \neg r \in ppL } \\
	&\evbcmeq{req:pop:left}{m0:act0}{ppL}{ ppL \1\bunion \{r\} } \\
	&\evguard{req:pop:right}{m0:grd0}{ \neg r \in ppR } \\
	&\evbcmeq{req:pop:right}{m0:act0}{ppL}{ ppL \1\bunion \{r\} }
\end{align}

\subsection{Requirements}

\begin{align}
	\progress{m0:p0}
		{r \in \dom.psL }
		{ \neg r \in \dom.psL } \\
	\progress{m0:p1}
		{r \in \dom.psR }
		{ \neg r \in \dom.psR } \\
	\progress{m0:p2}
		{r \in ppL }
		{ \neg r \in ppL } \\
	\progress{m0:p3}
		{r \in ppR }
		{ \neg r \in ppR } \\
	\safety{m0:s0}
		{ psL.r = X }
		{ \neg r \in \dom.psL } \\
	\safety{m0:s1}
		{ psR.r = X }
		{ \neg r \in \dom.psR }
\end{align}

\subsection{Addressing the requirements}
\[ \indices{resp:pop:right}{r : \Req} \]
\[ \indices{resp:pop:left}{r : \Req} \]
\[ \indices{resp:push:right}{r : \Req} \]
\[ \indices{resp:push:left}{r : \Req} \]
\begin{align}
	\refine{m0:p0}{ensure}{resp:push:left}{ \index{r}{r' = r} }
	\cschedule{resp:push:left}{m0:sch0}{ r \in \dom.psL } \\
	\evbcmeq{resp:push:left}{m0:act0}{psL}{ \{r\} \domsub psL }
\end{align}
\removecoarse{resp:push:left}{default}
\begin{align}
	\refine{m0:p2}{ensure}{resp:pop:left}{ \index{r}{r' = r} }
	\cschedule{resp:pop:left}{m0:sch0}{ r \in ppL } \\
	\evbcmeq{resp:pop:left}{m0:act0}{ppL}{ ppL \setminus \{ r \} }
\end{align}
\removecoarse{resp:pop:left}{default}
\begin{align}
	\refine{m0:p1}{ensure}{resp:push:right}{ \index{r}{r' = r} }
	\cschedule{resp:push:right}{m0:sch0}{ r \in \dom.psR } \\
	\evbcmeq{resp:push:right}{m0:act0}{psR}{ \{r\} \domsub psR }
\end{align}
\removecoarse{resp:push:right}{default}
\begin{align}
	\refine{m0:p3}{ensure}{resp:pop:right}{ \index{r}{r' = r} }
	\cschedule{resp:pop:right}{m0:sch0}{ r \in ppR } \\
	\evbcmeq{resp:pop:right}{m0:act0}{ppR}{ ppR \setminus \{ r \} }
\end{align}
\removecoarse{resp:pop:right}{default}
\end{machine}
\section{First Refinement --- Sequence of Values}
\begin{block}
  \item   \textbf{machine} m1
  \item   \textbf{variables}
  \begin{block}
    \item   $in$
    \item   $loc$
  \end{block}
  \item   %!TEX root=../main8.tex
\textbf{invariants}
\begin{block}
\item[ \eqref{m1:inv0} ]{$qe \in \intervalR{p}{q} \tfun \G $} %
\item[ \eqref{m1:inv1} ]{$p \le q $} %
\end{block}

  \item   \textbf{progress}
\begin{block}
\item[ \eqref{m1:prog0} ]$r \in \dom.pshL \1\land pshL.r = x  \quad \mapsto\quad p < q \land qe.p = x \1\land \neg r \in \dom.pshL $ %
\item[ \eqref{m1:prog1} ]$r \in \dom.pshR \1\land pshR.r = x  \quad \mapsto\quad p < q \1\land qe.(q-1) = x \1\land \neg r \in \dom.pshR $ %
\item[ \eqref{m1:prog2} ]$r \in popR  \quad \mapsto\quad \neg r \in popR $ %
\item[ \eqref{m1:prog3} ]$r \in popL  \quad \mapsto\quad \neg r \in popL  $ %
\end{block}

  \item   \textbf{safety}
\begin{block}
\item[ \eqref{m1:saf0} ]$\neg t \in in  \textbf{\quad unless \quad} t \in in \land loc.t = ent $ %
\item[ \eqref{m1:saf1} ]$t \in in \land loc.t = ent  \textbf{\quad unless \quad} t \in in \land loc.t \in plf $ %
\item[ \eqref{m1:saf2} ]$t \in in \land loc.t \in plf  \textbf{\quad unless \quad} t \in in \land loc.t = ext $ %
\item[ \eqref{m1:saf3} ]$t \in in \land loc.t = ext  \textbf{\quad unless \quad} \neg t \in in $ %
\end{block}

  \item   %!TEX root=../puzzle.tex

  \item   \textbf{events}
  \begin{block}
    \item   \noindent \ref{m0:enter} [t] \textbf{event}
\begin{block}
  \item   \textbf{when}
  \begin{block}
  \item[ \eqref{m0:enterent:grd1} ]$\neg t \in in $ %
  \end{block}
  \item   \textbf{begin}
  \begin{block}
  \item[ \eqref{m0:entera1} ]$in \bcmsuch in' = in \bunion \{ t \} $ %
  \item[ \eqref{m0:entera3} ]$loc \bcmsuch loc' = loc \1| t \fun ent $ %
  \end{block}
  \item   \textbf{end} \\
\end{block}

    \item   \noindent \ref{m0:leave} [t] \textbf{event}
\begin{block}
  \item   \textbf{during}
  \begin{block}
  \item[ \eqref{m0:leavelv:c0} ]$t \in in $ %
  \item[ \eqref{m0:leavelv:c1} ]$loc.t = ext $ %
  \end{block}
  \item   \textbf{when}
  \begin{block}
  \item[ \eqref{m0:leavelv:grd0} ]$t \in in $ %
  \item[ \eqref{m0:leavelv:grd1} ]$loc.t = ext $ %
  \end{block}
  \item   \textbf{begin}
  \begin{block}
  \item[ \eqref{m0:leavelv:a0} ]$in \bcmsuch in' = in \setminus \{ t \} $ %
  \item[ \eqref{m0:leavelv:a2} ]$loc \bcmsuch loc' = \{ t \} \domsub loc $ %
  \end{block}
  \item   \textbf{end} \\
\end{block}

    \item   \noindent \ref{m1:movein} [t] \textbf{event}
\begin{block}
  \item   \textbf{during}
  \begin{block}
  \item[ \eqref{m1:moveinmi:c1} ]$t \in in $ %
  \item[ \eqref{m1:moveinmi:c2} ]$loc.t = ent $ %
  \end{block}
  \item   \textbf{any} b
  \item   \textbf{when}
  \begin{block}
  \item[ \eqref{m1:moveinmi:g1} ]$t \in in $ %
  \item[ \eqref{m1:moveinmi:grd0} ]$loc.t = ent $ %
  \item[ \eqref{m1:moveinmi:grd7} ]$b \in plf $ %
  \end{block}
  \item   \textbf{begin}
  \begin{block}
  \item[ \eqref{m1:moveinmi:a2} ]$loc \bcmsuch loc' = loc \1| t \fun b $ %
  \end{block}
  \item   \textbf{end} \\
\end{block}

    \item   \noindent \ref{m1:moveout} [t] \textbf{event}
\begin{block}
\item \textbf{during}
\begin{block}
\item[ \eqref{m1:moveoutc1} ]$t \in in  %
		\1\land loc.t \in plf $ %
\end{block}
\item \textbf{when}
\begin{block}
\item[ \eqref{m1:moveoutmo:g1} ]$t \in in $ %
\item[ \eqref{m1:moveoutmo:g2} ]$loc.t \in plf $ %
\end{block}
\item \textbf{begin}
\begin{block}
\item[ \eqref{m1:moveoutSKIP:in} ]$in' = in$ %
\item[ \eqref{m1:moveouta2} ]$loc' = loc \1 | t \fun ext $ %
\end{block}
\item \textbf{end} \\
\end{block}

  \end{block}
  \item   \textbf{end} \\
\end{block}

\begin{machine}{m1}
	\refines{m0}
	\with{intervals}
\splitevent{resp:pop:right}{resp:pop:right:empty,resp:pop:right}
\splitevent{resp:pop:left}{resp:pop:left:empty,resp:pop:left}
	\[ \variable{p,q : \Int} \]
	\[ \variable{qe : \Int \pfun \G} \]
	\[ \variable{emp : \Bool } \]
	\[ \variable{ res : \G } \]
\begin{align}
	\invariant{m1:inv0}{ qe \in \intervalR{p}{q} \tfun \G } \\
	\invariant{m1:inv1}{ p \le q } \\
	\initialization{m1:init0}{ qe = \emptyfun } \\
	\initialization{m1:init1}{ p = 0 \land q = 0 } \\
	\evbcmeq{resp:push:right}{m1:act0}{q}{q+1} \\
	\evguard{resp:push:right}{m1:grd0}{ r \in \dom.psR } \\
	\evbcmeq{resp:push:right}{m1:act1}{qe}{ qe \1| q \fun psR.r } \\
	\evbcmeq{resp:pop:right}{m1:act0}{q}{q-1} \\
	\evguard{resp:pop:right}{m1:grd0}{\neg p = q} \\
	\fschedule{resp:pop:right}{m1:sch0}{\neg p = q} \\
	\evbcmeq{resp:pop:right}{m1:act1}{qe}{ \{ q\0-1 \} \domsub qe } \\
	\evbcmeq{resp:pop:right}{m1:act2}{emp}{ \false } \\
	\evbcmeq{resp:pop:right}{m1:act3}{res}{ qe.(q\0-1) } \\
	\fschedule{resp:pop:right:empty}{m1:sch0}{ p = q } \\
	\evguard{resp:pop:right:empty}{m1:grd0}{ p = q } \\
	\evbcmeq{resp:pop:right:empty}{m1:act2}{emp}{ \true } \\
	\evbcmeq{resp:push:left}{m1:act0}{p}{p-1} \\
	\evguard{resp:push:left}{m1:grd0}{ r \in \dom.psL } \\
	\evbcmeq{resp:push:left}{m1:act1}{qe}{qe \1| p\0-1 \fun psL.r} \\
	\evbcmeq{resp:pop:left}{m1:act0}{p}{p+1} \\
	\evguard{resp:pop:left}{m1:grd0}{\neg p = q} \\
	\fschedule{resp:pop:left}{m1:sch0}{\neg p = q} \\
	\evbcmeq{resp:pop:left}{m1:act1}{qe}{ \{ p \} \domsub qe } \\
	\evbcmeq{resp:pop:left}{m1:act2}{emp}{ \false } \\
	\evbcmeq{resp:pop:left}{m1:act3}{res}{ qe.p } \\
	\fschedule{resp:pop:left:empty}{m1:sch0}{ p = q } \\
	\evguard{resp:pop:left:empty}{m1:grd0}{ p = q } \\
	\evbcmeq{resp:pop:left:empty}{m1:act2}{emp}{ \true } 
\end{align}
\end{machine}
\begin{machine}{m2}
	\refines{m1}
\end{machine}
\end{document}