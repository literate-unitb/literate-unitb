\documentclass{article}
\usepackage{geometry}
\usepackage{amsmath}
\usepackage{bsymb}
\usepackage{../unitb}
% \usepackage{eventb} 
\usepackage{calculational}

\begin{document}
	
\begin{machine}{m0}
	
\[
\variable{ b : \Bool }
\]
\begin{description}
\comment{b}{ indicates termination of the process }
\end{description}
\newevent{term}{terminate}

\begin{align}
\initialization{in0}{b = \false} \\
\progress{prog0}{\true}{b}
\end{align}

\begin{align}
\refine{prog0}{ensure}{term}{} \\
% \removecoarse{term}{default} % \weakento{term}{default}{sch0} \\
\cschedule{term}{sch0}{\true} 
\end{align}

\begin{align}
\evbcmeq{term}{act0}{b}{\true}
\end{align}
\begin{block}
  \item   \textbf{machine} m0
  \item   \textbf{variables}
  \begin{block}
    \item   $in$
  \end{block}
  \item   \textbf{progress}
\begin{block}
\item[ \eqref{m0:prog0} ]$t \in in  \quad \mapsto\quad \neg t \in in $ %
\end{block}

  \item   \textbf{transient}
\begin{block}
\item[ \eqref{m0:tr0} ]$t \in in  \qquad \text{(\ref{m0:leave}: [t := t' | t' = t])}$ %
\end{block}

  \item   \textbf{events}
  \begin{block}
    \item   \noindent \ref{m0:enter} [t] \textbf{event}
\begin{block}
\item \textbf{during}
\begin{block}
\item[ \eqref{m0:enterdefault} ]$\false$ %
\end{block}
\item \textbf{begin}
\begin{block}
\item[ \eqref{m0:entera1} ]$in' = in \bunion \{ t \} $ %
\end{block}
\item \textbf{end} \\
\end{block}

    \item   \noindent \ref{m0:leave} [t] \textbf{event}
\begin{block}
  \item   \textbf{during}
  \begin{block}
  \item[ \eqref{m0:leavelv:c0} ]$t \in in $ %
  \end{block}
  \item   \textbf{begin}
  \begin{block}
  \item[ \eqref{m0:leavelv:a0} ]$in \bcmsuch in' = in \setminus \{ t \} $ %
  \end{block}
  \item   \textbf{end} \\
\end{block}

  \end{block}
  \item   \textbf{end} \\
\end{block}


\end{machine}

\newcommand{\Pcs}{\text{P}}

\begin{machine}{m1}
	\refines{m0}

\with{sets}
\begin{align*}	
\newset{\Pcs} \\
\variable{ vs : \set[\Pcs] } \\
\dummy{V : \set[\Pcs]}
\end{align*}\begin{description}
\comment{vs}{set of visited processes}\end{description}
\begin{align}
	\invariant{inv0}{b \2\implies& vs = \Pcs} \\
	\evguard{term}{grd0}{ vs &= \Pcs } \\
	\cschedule{term}{sch1}{ vs &= \Pcs }
\end{align}\begin{description}
\comment{\ref{inv0}}{ termination is characterized by everyone 
	having visited }\end{description}
\replace{term}{sch0}{sch1}{}{prog1}{saf1}
\begin{align*}
	\safety{saf1}{vs = \Pcs}{\false}
\end{align*}
	
\begin{align*}
	&\progress{prog1}{\true}{vs = \Pcs} 
\refine{prog1}{induction}{prog2}
		{ \var{\Pcs \setminus vs}{down}{\emptyset} }
	&\progress{prog2}
		{ \Pcs \setminus vs = V }
		{ (\Pcs \setminus vs \subset V) \1\lor vs = \Pcs }
\refine{prog2}{psp}{prog3,saf2}{}
	&\progress{prog3}
		{ \Pcs \setminus vs = V \land \neg vs = \Pcs}
		{\neg \Pcs \setminus vs = V} \\
	&\safety{saf2}{\Pcs \setminus vs \subseteq V}{vs = \Pcs}
\refine{prog3}{ensure}{visit}{ [\index{p}{\neg p' \in vs}] }
\end{align*}

\begin{align*}
	\assumption{asm0}{ \finite.\Pcs }
\end{align*}

\newevent{visit}{visit}

\[ \indices{visit}{p : \Pcs} \]

\begin{align}
% 	\cschedule{visit}{sch0}{}
	\evbcmeq{visit}{act1}{vs}{ vs \bunion \{ p \} }
\end{align}

% \removecoarse{visit}{default} % \weakento{visit}{default}{}

% \input{}
\begin{block}
  \item   \textbf{machine} m1
  \item   \textbf{variables}
  \begin{block}
    \item   $in$
    \item   $loc$
  \end{block}
  \item   %!TEX root=../main8.tex
\textbf{invariants}
\begin{block}
\item[ \eqref{m1:inv0} ]{$qe \in \intervalR{p}{q} \tfun \G $} %
\item[ \eqref{m1:inv1} ]{$p \le q $} %
\end{block}

  \item   \textbf{progress}
\begin{block}
\item[ \eqref{m1:prog0} ]$r \in \dom.pshL \1\land pshL.r = x  \quad \mapsto\quad p < q \land qe.p = x \1\land \neg r \in \dom.pshL $ %
\item[ \eqref{m1:prog1} ]$r \in \dom.pshR \1\land pshR.r = x  \quad \mapsto\quad p < q \1\land qe.(q-1) = x \1\land \neg r \in \dom.pshR $ %
\item[ \eqref{m1:prog2} ]$r \in popR  \quad \mapsto\quad \neg r \in popR $ %
\item[ \eqref{m1:prog3} ]$r \in popL  \quad \mapsto\quad \neg r \in popL  $ %
\end{block}

  \item   \textbf{safety}
\begin{block}
\item[ \eqref{m1:saf0} ]$\neg t \in in  \textbf{\quad unless \quad} t \in in \land loc.t = ent $ %
\item[ \eqref{m1:saf1} ]$t \in in \land loc.t = ent  \textbf{\quad unless \quad} t \in in \land loc.t \in plf $ %
\item[ \eqref{m1:saf2} ]$t \in in \land loc.t \in plf  \textbf{\quad unless \quad} t \in in \land loc.t = ext $ %
\item[ \eqref{m1:saf3} ]$t \in in \land loc.t = ext  \textbf{\quad unless \quad} \neg t \in in $ %
\end{block}

  \item   %!TEX root=../puzzle.tex

  \item   \textbf{events}
  \begin{block}
    \item   \noindent \ref{m0:enter} [t] \textbf{event}
\begin{block}
  \item   \textbf{when}
  \begin{block}
  \item[ \eqref{m0:enterent:grd1} ]$\neg t \in in $ %
  \end{block}
  \item   \textbf{begin}
  \begin{block}
  \item[ \eqref{m0:entera1} ]$in \bcmsuch in' = in \bunion \{ t \} $ %
  \item[ \eqref{m0:entera3} ]$loc \bcmsuch loc' = loc \1| t \fun ent $ %
  \end{block}
  \item   \textbf{end} \\
\end{block}

    \item   \noindent \ref{m0:leave} [t] \textbf{event}
\begin{block}
  \item   \textbf{during}
  \begin{block}
  \item[ \eqref{m0:leavelv:c0} ]$t \in in $ %
  \item[ \eqref{m0:leavelv:c1} ]$loc.t = ext $ %
  \end{block}
  \item   \textbf{when}
  \begin{block}
  \item[ \eqref{m0:leavelv:grd0} ]$t \in in $ %
  \item[ \eqref{m0:leavelv:grd1} ]$loc.t = ext $ %
  \end{block}
  \item   \textbf{begin}
  \begin{block}
  \item[ \eqref{m0:leavelv:a0} ]$in \bcmsuch in' = in \setminus \{ t \} $ %
  \item[ \eqref{m0:leavelv:a2} ]$loc \bcmsuch loc' = \{ t \} \domsub loc $ %
  \end{block}
  \item   \textbf{end} \\
\end{block}

    \item   \noindent \ref{m1:movein} [t] \textbf{event}
\begin{block}
  \item   \textbf{during}
  \begin{block}
  \item[ \eqref{m1:moveinmi:c1} ]$t \in in $ %
  \item[ \eqref{m1:moveinmi:c2} ]$loc.t = ent $ %
  \end{block}
  \item   \textbf{any} b
  \item   \textbf{when}
  \begin{block}
  \item[ \eqref{m1:moveinmi:g1} ]$t \in in $ %
  \item[ \eqref{m1:moveinmi:grd0} ]$loc.t = ent $ %
  \item[ \eqref{m1:moveinmi:grd7} ]$b \in plf $ %
  \end{block}
  \item   \textbf{begin}
  \begin{block}
  \item[ \eqref{m1:moveinmi:a2} ]$loc \bcmsuch loc' = loc \1| t \fun b $ %
  \end{block}
  \item   \textbf{end} \\
\end{block}

    \item   \noindent \ref{m1:moveout} [t] \textbf{event}
\begin{block}
\item \textbf{during}
\begin{block}
\item[ \eqref{m1:moveoutc1} ]$t \in in  %
		\1\land loc.t \in plf $ %
\end{block}
\item \textbf{when}
\begin{block}
\item[ \eqref{m1:moveoutmo:g1} ]$t \in in $ %
\item[ \eqref{m1:moveoutmo:g2} ]$loc.t \in plf $ %
\end{block}
\item \textbf{begin}
\begin{block}
\item[ \eqref{m1:moveoutSKIP:in} ]$in' = in$ %
\item[ \eqref{m1:moveouta2} ]$loc' = loc \1 | t \fun ext $ %
\end{block}
\item \textbf{end} \\
\end{block}

  \end{block}
  \item   \textbf{end} \\
\end{block}


\end{machine}

\begin{machine}{m2}
\refines{m1}

\[ \variable{ts : \set [\Pcs] } \]\begin{description}
\comment{ts}{set of process detected to have visited} \end{description}
\begin{align}
	\cschedule{term}{sch2}{ ts = \Pcs }
\end{align}

\replace{term}{sch1}{sch2}{}{prog4}{saf3}

\begin{align}
	\safety{saf3}{ ts = \Pcs }{ \neg vs = \Pcs } \\
	\progress{prog4}{ \true }{ ts = \Pcs } \\
	\invariant{inv1}{ ts \subseteq vs } \\
	\initialization{in1}{ ts = \emptyset }
\end{align}

\[ \variable{ cs : \set [\Pcs] } \]\begin{description}
\comment{cs}{one-slot channel used to notify the counter of visitations} \end{description}
\begin{align*}
	\refine{prog4}{transitivity}{prog10,prog9}{}
& \progress{prog10}{ \true }{ cs = \emptyset } \\
& \progress{prog9}{ cs = \emptyset }{ ts = \Pcs }
	\refine{prog9}{induction}{prog5}{ \var{ts}{up}{\Pcs} }
& \progress{prog5}
	 	{ cs = \emptyset \land ts = V }
	 	{ (cs = \emptyset \land V \subset ts) \lor ts = \Pcs }
	\refine{prog5}{PSP}{prog8,saf8}{}
& \progress{prog8}
		{ cs = \emptyset \land ts = V \land \neg ts = \Pcs }
		{ cs = \emptyset \land \neg ts = V }
\\& \safety{saf8}{ V \subseteq ts }{ ts = \Pcs }
	\refine{prog8}{transitivity}{prog11,prog6,prog7}{}
& \progress{prog11}
		{ cs = \emptyset \land ts = V \land \neg ts = \Pcs }
		{ cs = \emptyset \land ts = V \land \neg ts = \Pcs 
			\land \neg ts = vs } \\
& \progress{prog6}
		{ cs = \emptyset \land ts = V \land \neg ts = \Pcs \land \neg ts = vs }
		{ ts = V \land \neg cs = \emptyset \land cs \subseteq \Pcs \setminus ts } \\
& \progress{prog7}
		{ ts = V \land \neg cs = \emptyset 
				 \land cs \subseteq \Pcs \setminus ts }
		{ cs = \emptyset \land \neg ts = V }
	\refine{prog6}{ensure}{flick}{ \index{p}{p' \in vs \setminus ts} }
	\refine{prog7}{ensure}{count}{} % \index{p}{ p' \in cs } }
	\refine{prog11}{ensure}{visit}{ \index{p}{ \neg p' \in vs} }
\end{align*}

\begin{align*}
	% \refine{prog10}{trading}{prog11}{}
% & \progress{prog11}{ \neg cs = \emptyset }{ cs = \emptyset }
	\refine{prog10}{ensure}{count}{} % \index{p}{ p' \in cs } }
\end{align*}

\newevent{flick}{flick}
\newevent{count}{count}

\[ \indices{flick}{p : \Pcs} \]

\begin{align}
	\cschedule{flick}{sch0}{ cs = \emptyset } \\
	\cschedule{flick}{sch1}{ \neg p \in ts } \\
	\evguard{flick}{grd0}{ \neg p \in ts } \\
	\evbcmeq{flick}{act0}{cs}{ \{ p \} }
\end{align}

% \removecoarse{flick}{default} % \weakento{flick}{default}{sch0,sch1,sch2}

%!TEX root=../puzzle.tex
\noindent \ref{term}  \textbf{event}
\begin{block}
  \item   \textbf{during}
  \begin{block}
  \item[ \eqref{termsch1} ]\sout{$vs = \Pcs$} %
  \end{block}
  \begin{block}
  \item[ \eqref{termsch2} ]{$ts = \Pcs $} %
  \end{block}
  \item   \textbf{when}
  \begin{block}
  \item[ \eqref{termgrd0} ]{$vs = \Pcs$} %
  \end{block}
  \item   \textbf{begin}
  \begin{block}
  \item[ \eqref{termact0} ]{$b \bcmeq \true$} %
  \end{block}
  \item   \textbf{end} \\
\end{block}

\noindent \ref{flick} [p] \textbf{event}
\begin{block}
  \item   \textbf{during}
  \begin{block}
  \item[ \eqref{flicksch0} ]$cs = \emptyset$ %
  \item[ \eqref{flicksch1} ]$\neg p \in ts $ %
  \item[ \eqref{flicksch2} ]$p \in vs$ %
  \end{block}
  \item   \textbf{when}
  \begin{block}
  \item[ \eqref{flickgrd0} ]$\neg p \in ts $ %
  \item[ \eqref{flickgrd1} ]$p \in vs$ %
  \end{block}
  \item   \textbf{begin}
  \begin{block}
  \item[ \eqref{flickact0} ]$cs \bcmeq \{ p \} $ %
  \end{block}
  \item   \textbf{end} \\
\end{block}


% \[ \indices{count}{p : \Pcs} \]

\begin{align}
	\invariant{inv2}{cs \subseteq vs} \\
	\initialization{in2}{ cs = \emptyset } \\
	\cschedule{count}{sch0}{ \neg cs = \emptyset } \\
	% \cschedule{count}{sch1}{ cs \subseteq vs } \\
	% \evguard{count}{grd0}{ cs \subseteq vs } \\
	\evbcmeq{count}{act0}{cs}{\emptyset} \\
	\evbcmeq{count}{act1}{ts}{ ts \bunion cs }
\end{align}

% \removecoarse{count}{default} % \weakento{count}{default}{sch0}

%!TEX root=../puzzle.tex
\noindent \ref{count}  \textbf{event}
\begin{block}
  \item   \textbf{during}
  \begin{block}
  \item[ (\ref{count}/default) ]\sout{$\false$} %
  \end{block}
  \begin{block}
  \item[ \eqref{countsch0} ]{$\neg cs = \emptyset $} %
  \end{block}
  \item   \textbf{begin}
  \begin{block}
  \item[ \eqref{countact0} ]{$cs \bcmeq \emptyset$} %
  \item[ \eqref{countact1} ]{$ts \bcmeq ts \bunion cs $} %
  \end{block}
  \item   \textbf{end} \\
\end{block}


\begin{align*}
	\evguard{flick}{grd1}{p \in vs} \\
	\cschedule{flick}{sch2}{p \in vs}
\end{align*}

%!TEX root=../main9.tex
\begin{block}
  \item   \textbf{machine} m2
  \item   \textbf{refines} m0
  \item   \textbf{variables}
  \begin{block}
    \item   $req$
    \item   $req0$
    \item   $reqA$
    \item   $reqB$
  \end{block}
  \item   %!TEX root=../main9.tex
\textbf{invariants}
\begin{block}
\item[ \eqref{m1:inv0} ]{$reqA \bunion reqB = req$} %
\item[ \eqref{m1:inv1} ]{$reqA \binter reqB = req$} %
\end{block}

  \item   \textbf{events}
  \begin{block}
    \item   %!TEX root=../main9.tex
\noindent \ref{handle}  \textbf{event}
\begin{block}
  \item   \textbf{during}
  \begin{block}
  \item[ \eqref{handlem0:sch0} ]{$\neg req = \emptyset$} %
  \end{block}
  \item   \textbf{any} r
  \item   \textbf{when}
  \begin{block}
  \item[ \eqref{handlegrd0} ]{$r \in req$} %
  \end{block}
  \item   \textbf{begin}
  \begin{block}
  \item[ \eqref{handleact0} ]{$req \bcmeq req \setminus \{ r \}$} %
  \item[ \eqref{handleact1} ]{$req0 \bcmeq req$} %
  \end{block}
  \item   \textbf{end} \\
\end{block}

    \item   %!TEX root=../main9.tex
\noindent \ref{req}  \textbf{event}
\begin{block}
  \item   \textbf{during}
  \begin{block}
  \item[ (\ref{req}/default) ]{$\false$} %
  \end{block}
  \item   \textbf{any} r
  \item   \textbf{when}
  \begin{block}
  \item[ \eqref{reqgrd0} ]{$\neg r \in req$} %
  \end{block}
  \item   \textbf{begin}
  \begin{block}
  \item[ \eqref{reqact0} ]{$req \bcmeq req \bunion \{ r \}$} %
  \item[ \eqref{reqact1} ]{$req0 \bcmeq req$} %
  \end{block}
  \item   \textbf{end} \\
\end{block}

  \end{block}
  \item   \textbf{end} \\
\end{block}


\end{machine}

\begin{machine}{m3}
	\refines{m2}
	\[ \variable{ c : \Int } \]
	\begin{description}
		\comment{c}{ one-bit channel used to communicate }
	\end{description}\removevar{cs}
	\begin{align}
		\invariant{m3:inv5}{ c \in \{0,1\} } \\
		\invariant{m3:inv0}{ c = \card.cs } \\
		\initialization{m3:in0}{ c = 0 } \\ 
		\evbcmeq{count}{m3:act0}{c}{0} \\
		\evbcmeq{flick}{m3:act0}{c}{1}
	\end{align}
	\[ \variable{ n : \Int } \]
	\begin{align}
		\invariant{m3:inv1}{ n = \qsum{p}{p \in ts}{1} } \\
		\invariant{m3:inv2}{ \finite.ts }\\
		\invariant{m3:inv3}{ \finite.cs } \\
		\initialization{m3:in1}{ n = 0 }  \\
		% \invariant{m3:inv4}{ \qsum{p}{p \in ts \bunion cs}{1} = n + c }\\
		\invariant{m3:inv6}{ ts \binter cs = \emptyset } \\
		\evbcmeq{count}{m3:act1}{n}{n+1} \\
		\evguard{count}{m3:grd0}{ c = 1 }
	\end{align}
	\begin{block}
  \item   \textbf{machine} m3
  \item   \textbf{variables}
  \begin{block}
    \item   $in$
    \item   $isgn$
    \item   $loc$
    \item   $osgn$
  \end{block}
  \item   %!TEX root=../main.tex
\textbf{invariants}
\begin{block}
\item[ \eqref{m3:inv0} ]{$\dom.delta = Pcs $} %
\item[ \eqref{m3:inv1} ]{$b \2\subseteq d \setminus \qset{p,q}{q \in delta.p}{q} $} %
\end{block}

  \item   \textbf{events}
  \begin{block}
    \item   \noindent \ref{m0:enter} [t] \textbf{event}
\begin{block}
  \item   \textbf{when}
  \begin{block}
  \item[ \eqref{m0:enterent:grd1} ]$\neg t \in in $ %
  \item[ \eqref{m0:enteret:g1} ]$\neg ent \in \ran.loc $ %
  \end{block}
  \item   \textbf{begin}
  \begin{block}
  \item[ \eqref{m0:entera1} ]$in \bcmsuch in' = in \bunion \{ t \} $ %
  \item[ \eqref{m0:entera3} ]$loc \bcmsuch loc' = loc \1| t \fun ent $ %
  \item[ \eqref{m0:enterm3:ent:act0} ]$isgn,osgn \bcmsuch isgn' = isgn$ %
  \item[ \eqref{m0:enterm3:ent:act1} ]$isgn,osgn \bcmsuch osgn' = osgn$ %
  \end{block}
  \item   \textbf{end} \\
\end{block}

    \item   \noindent \ref{m0:leave} [t] \textbf{event}
\begin{block}
  \item   \textbf{during}
  \begin{block}
  \item[ \eqref{m0:leavelv:c0} ]$t \in in $ %
  \item[ \eqref{m0:leavelv:c1} ]$loc.t = ext $ %
  \end{block}
  \item   \textbf{when}
  \begin{block}
  \item[ \eqref{m0:leavelv:grd0} ]$t \in in $ %
  \item[ \eqref{m0:leavelv:grd1} ]$loc.t = ext $ %
  \end{block}
  \item   \textbf{begin}
  \begin{block}
  \item[ \eqref{m0:leavelv:a0} ]$in \bcmsuch in' = in \setminus \{ t \} $ %
  \item[ \eqref{m0:leavelv:a2} ]$loc \bcmsuch loc' = \{ t \} \domsub loc $ %
  \item[ \eqref{m0:leavem3:ext:act0} ]$isgn,osgn \bcmsuch isgn' = isgn$ %
  \item[ \eqref{m0:leavem3:ext:act1} ]$isgn,osgn \bcmsuch osgn' = osgn$ %
  \end{block}
  \item   \textbf{end} \\
\end{block}

    \item   \noindent \ref{m1:movein} [t] \textbf{event}
\begin{block}
  \item   \textbf{during}
  \begin{block}
  \item[ \eqref{m1:moveinmi:c0} ]$\neg ~ plf \subseteq \ran.loc $ %
  \item[ \eqref{m1:moveinmi:c1} ]$t \in in $ %
  \item[ \eqref{m1:moveinmi:c2} ]$loc.t = ent $ %
  \end{block}
  \item   \textbf{any} b
  \item   \textbf{when}
  \begin{block}
  \item[ \eqref{m1:moveinmi:g0} ]$\neg b \in \ran.loc  	$ %
  \item[ \eqref{m1:moveinmi:g1} ]$t \in in $ %
  \item[ \eqref{m1:moveinmi:grd0} ]$loc.t = ent $ %
  \item[ \eqref{m1:moveinmi:grd7} ]$b \in plf $ %
  \end{block}
  \item   \textbf{begin}
  \begin{block}
  \item[ \eqref{m1:moveinm3:mi:act0} ]$isgn,osgn \bcmsuch isgn' = isgn$ %
  \item[ \eqref{m1:moveinm3:mi:act1} ]$isgn,osgn \bcmsuch osgn' = osgn$ %
  \item[ \eqref{m1:moveinmi:a2} ]$loc \bcmsuch loc' = loc \1| t \fun b $ %
  \end{block}
  \item   \textbf{end} \\
\end{block}

    \item   \noindent \ref{m1:moveout} [t] \textbf{event}
\begin{block}
\item \textbf{during}
\begin{block}
\item[ \eqref{m1:moveoutc1} ]$t \in in  %
		\1\land loc.t \in plf $ %
\item[ \eqref{m1:moveoutm3:mo:sch0} ]$loc.t \in osgn $ %
\end{block}
\item \textbf{when}
\begin{block}
\item[ \eqref{m1:moveoutm3:mo:grd0} ]$loc.t \in osgn $ %
\item[ \eqref{m1:moveoutmo:g1} ]$t \in in $ %
\item[ \eqref{m1:moveoutmo:g2} ]$loc.t \in plf $ %
\end{block}
\item \textbf{begin}
\begin{block}
\item[ \eqref{m1:moveoutSKIP:in} ]$in' = in$ %
\item[ \eqref{m1:moveouta2} ]$loc' = loc \1 | t \fun ext $ %
\item[ \eqref{m1:moveoutm3:mo:act0} ]$isgn' = isgn$ %
\item[ \eqref{m1:moveoutm3:mo:act1} ]$osgn'  \2 = osgn  %
	\setminus \{ loc.t \} $ %
\end{block}
\item \textbf{end} \\
\end{block}

  \end{block}
  \item   \textbf{end} \\
\end{block}

\end{machine}

\begin{machine}{m4}
	\refines{m3}
	\invariant{m4:inv0}{cs \subseteq cs}
\end{machine}

\end{document}
