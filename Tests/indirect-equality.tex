\documentclass[12pt]{amsart}
\usepackage{geometry} % see geometry.pdf on how to lay out the page. There's lots.
\usepackage{bsymb}
\usepackage{unitb}
\usepackage{calculation}
\usepackage{ulem}
\usepackage{hyperref}
\normalem
\geometry{a4paper} % or letter or a5paper or ... etc
% \geometry{landscape} % rotated page geometry

% See the ``Article customise'' template for some common customisations

\title{}
\author{}
\date{} % delete this line to display the current date

%%% BEGIN DOCUMENT
\setcounter{tocdepth}{4}
\begin{document}

\maketitle
\tableofcontents

\newcommand{\G}{\text{G}}
\renewcommand{\H}{\text{H}}

\section{Initial model}
\begin{machine}{m0}

\newset{G}{\G}
\newset{H}{\H}

\with{sets}
\with{functions}

\begin{align*}
\dummy{x, x_0, x_1 : \set[\G]} \\
\dummy{y,y_0,y_1, z: \set[\H]}
\end{align*}
\begin{align*}
\constant{f : \set [\G] \pfun \set [\H]} \\
\constant{g : \set [\H] \pfun \set [\G]} \\
\end{align*}
\begin{align*}
\assumption{asm0}{ \qforall{x,y}{}{ f.x \subseteq y \2\equiv x \subseteq g.y } } \\
\invariant{inv0}{ \qforall{x_0,x_1}{}{ f.(x_0 \bunion x_1) \1 = f.x_0 \bunion f.x_1 } }
\end{align*}

\newenvironment{indirect:equality}[3]{}{}

\begin{proof}{INIT/INV/\ref{inv0}}
	\begin{free:var}{x_0}{x_0}
	\begin{free:var}{x_1}{x_1}
	\begin{indirect:equality}{right}{\subseteq}{z}
%	\easy
	\begin{calculation}
		f.(x_0 \bunion x_1) \1\subseteq z
	\hint{=}{ \eqref{asm0} }
		x_0 \bunion x_1 \1\subseteq g.z
	\hint{=}{ $\bunion$ and $\subseteq$ }
		x_0 \subseteq g.z \2\land x_1 \subseteq g.z
	\hint{=}{ \eqref{asm0} }
		f.x_0 \subseteq z \2\land f.x_1 \subseteq z
	\hint{=}{ $\bunion$ and $\subseteq$ }
		f.x_0 \bunion f.x_1 \1\subseteq z
	\end{calculation}
	\end{indirect:equality}
	\end{free:var}
	\end{free:var}
\end{proof}

\begin{align*}
\invariant{inv1}{ \qforall{y_0,y_1}{}{ y_0 = y_1 \2\equiv \qforall{z}{}{ y_0 \subseteq z \1\equiv y_1 \subseteq z } } }
\end{align*}

\begin{proof}{INIT/INV/\ref{inv1}}
	\begin{free:var}{y_0}{y_0}
	\begin{free:var}{y_1}{y_1}
	\begin{by:parts} \\
		\begin{part:a}{y_0 = y_1 \2\implies \qforall{z}{}{ y_0 \subseteq z \1\equiv y_1 \subseteq z }}
		\easy 
%		banana
		\end{part:a}

		\begin{part:a}{ \qforall{z}{}{ y_0 \subseteq z \1\equiv y_1 \subseteq z } \2\implies y_0 = y_1 }
		\begin{align}
		\assume{hyp0}{\qforall{z}{}{ y_0 \subseteq z \1\equiv y_1 \subseteq z }} \\
		\goal{ y_0 = y_1 }
		\end{align}
		\begin{calculation}
			y_0 = y_1
		\hint{=}{ antisymmetry of $\subseteq$ }
			y_0 \subseteq y_1 \1\land y_1 \subseteq y_0
		\hint{=}{ \eqref{hyp0} }
%		\hint{=}{  }
			y_1 \subseteq y_1 \1\land y_0 \subseteq y_0
		\hint{=}{ reflexivity of $\subseteq$ }
			\true
		\end{calculation}
		\end{part:a}
	\end{by:parts}
	\end{free:var}
	\end{free:var}
\end{proof}

\end{machine}

\begin{context}{ctx0}

\newset{G}{\G}

\dummy{x,y,z: \G}
\with{functions}

\operator{\le}{ le : Pair [ \G, \G ] \pfun \Bool }

\axiom{axm0}{ \qforall{x,y}{}{ (x \le y) \1\land (y \le x) \2\equiv x = y } }
\theorem{thm0}{ \qforall{x,y}{}{ x = y \2\equiv \qforall{z}{}{ z \le x \1\equiv z \le y } } }

\end{context}

\end{document}