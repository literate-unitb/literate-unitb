
\documentclass[12pt]{amsart}
\usepackage{geometry} % see geometry.pdf on how to lay out the page. There's lots.
\usepackage{calculation}
\usepackage{unitb}
\usepackage{amsmath}
\geometry{a4paper} % or letter or a5paper or ... etc
% \geometry{landscape} % rotated page geometry

% See the ``Article customise'' template for come common customisations

\title{A Sample Unit-B Toolkit proof}
\author{Simon Hudon}
\date{} % delete this line to display the current date

\newtheorem{theorem}{Theorem}
\newtheorem{lemma}{Lemma}

%%% BEGIN DOCUMENT
\begin{document}

\maketitle
%\tableofcontents

%\section{}

%\begin{theorem} 
%Any conjunctive function is monotonic
%\begin{assumption}
%f.(x \land y) ~ \equiv~ f.x \land f.y
%\end{assumption}
%\end{theorem}
%\begin{dummy}
\begin{calculation}
	f.x \le f.y
\hint{=}{ $\le$ to $=$ and $\uparrow$ }
	f.x \uparrow f.y \,=\, f.y
\hint{=}{ $f$ over $\uparrow$ }
	f.(x \uparrow y) \,=\, f.y
\hint{\follows}{ Leibniz }
	x \uparrow y \,=\, y
\hint{=}{ $\le$ from $=$ and $\uparrow$ }
	x \le y
\end{calculation}

\begin{calculation}
	f.x
\hint{\le}{ $x \le x\uparrow y$ }
	f.x \uparrow f.y
\hint{=}{ $f$ over $\uparrow$ }
	f.(x \uparrow y)
\hint{=}{ $x \le y$ cast as $x\uparrow y = y$ }
	f.y
\end{calculation}

%\subsection{}

\end{document}