\documentclass[12pt]{amsart}
\usepackage[margin=0.5in]{geometry} 
    % see geometry.pdf on how to lay out the page. There's lots.
\usepackage{../bsymb}
\usepackage{../unitb}
\usepackage{../calculation}
\usepackage{ulem}
\usepackage{hyperref}
\normalem
\geometry{a4paper} % or letter or a5paper or ... etc
% \geometry{landscape} % rotated page geometry

% See the ``Article customise'' template for some common
% customisations

\title{}
\author{}
\date{} % delete this line to display the current date

%%% BEGIN DOCUMENT
\setcounter{tocdepth}{4}
\begin{document}

\maketitle
% \tableofcontents
\begin{block}
  \item   \textbf{machine} m0
  \item   \textbf{variables}
  \begin{block}
    \item   $in$
  \end{block}
  \item   \textbf{progress}
\begin{block}
\item[ \eqref{m0:prog0} ]$t \in in  \quad \mapsto\quad \neg t \in in $ %
\end{block}

  \item   \textbf{transient}
\begin{block}
\item[ \eqref{m0:tr0} ]$t \in in  \qquad \text{(\ref{m0:leave}: [t := t' | t' = t])}$ %
\end{block}

  \item   \textbf{events}
  \begin{block}
    \item   \noindent \ref{m0:enter} [t] \textbf{event}
\begin{block}
\item \textbf{during}
\begin{block}
\item[ \eqref{m0:enterdefault} ]$\false$ %
\end{block}
\item \textbf{begin}
\begin{block}
\item[ \eqref{m0:entera1} ]$in' = in \bunion \{ t \} $ %
\end{block}
\item \textbf{end} \\
\end{block}

    \item   \noindent \ref{m0:leave} [t] \textbf{event}
\begin{block}
  \item   \textbf{during}
  \begin{block}
  \item[ \eqref{m0:leavelv:c0} ]$t \in in $ %
  \end{block}
  \item   \textbf{begin}
  \begin{block}
  \item[ \eqref{m0:leavelv:a0} ]$in \bcmsuch in' = in \setminus \{ t \} $ %
  \end{block}
  \item   \textbf{end} \\
\end{block}

  \end{block}
  \item   \textbf{end} \\
\end{block}

\begin{machine}{m0}

\newset{FS} \newset{AST}
    \with{functions}
    \with{sets}
    \with{intervals}

\noindent \textbf{Constant:}
    \[\constant{parse : FS \pfun AST}\]

\noindent \textbf{Variables:}
    \[\variable{in : \Int \pfun FS};~\variable{v : \Int} \]
    \newevent{input}{INPUT}
    \begin{align*}
        \invariant{m0:inv0}{ &in \in \intervalL{0}{v} \tfun FS } \\
        \initialization{m0:in0}{ &in = \emptyfun } \\
        \initialization{m0:in1}{ &v = 0 } \\
        \invariant{m0:inv1}{ &0 \le v } \\
    \end{align*}
    \begin{itemize}
    \comment{v}{Latest version of the source files}
    \comment{in}{Sequence of versions of the source files}
    \end{itemize}
    \[ \param{input}{file : FS} \]
    \begin{align}
        \evbcmeq{input}{m0:act0}{in}{ &in \2| (v\0+1 \fun file) } \\
        \evbcmeq{input}{m0:act1}{v}{&v+1} \\
        \dummy{V : \Int} \\
        \safety{m0:saf0}{V \le v}{\false}
    \end{align}
        
    \end{machine}
    \begin{block}
  \item   \textbf{machine} m1
  \item   \textbf{variables}
  \begin{block}
    \item   $in$
    \item   $loc$
  \end{block}
  \item   %!TEX root=../main8.tex
\textbf{invariants}
\begin{block}
\item[ \eqref{m1:inv0} ]{$qe \in \intervalR{p}{q} \tfun \G $} %
\item[ \eqref{m1:inv1} ]{$p \le q $} %
\end{block}

  \item   \textbf{progress}
\begin{block}
\item[ \eqref{m1:prog0} ]$r \in \dom.pshL \1\land pshL.r = x  \quad \mapsto\quad p < q \land qe.p = x \1\land \neg r \in \dom.pshL $ %
\item[ \eqref{m1:prog1} ]$r \in \dom.pshR \1\land pshR.r = x  \quad \mapsto\quad p < q \1\land qe.(q-1) = x \1\land \neg r \in \dom.pshR $ %
\item[ \eqref{m1:prog2} ]$r \in popR  \quad \mapsto\quad \neg r \in popR $ %
\item[ \eqref{m1:prog3} ]$r \in popL  \quad \mapsto\quad \neg r \in popL  $ %
\end{block}

  \item   \textbf{safety}
\begin{block}
\item[ \eqref{m1:saf0} ]$\neg t \in in  \textbf{\quad unless \quad} t \in in \land loc.t = ent $ %
\item[ \eqref{m1:saf1} ]$t \in in \land loc.t = ent  \textbf{\quad unless \quad} t \in in \land loc.t \in plf $ %
\item[ \eqref{m1:saf2} ]$t \in in \land loc.t \in plf  \textbf{\quad unless \quad} t \in in \land loc.t = ext $ %
\item[ \eqref{m1:saf3} ]$t \in in \land loc.t = ext  \textbf{\quad unless \quad} \neg t \in in $ %
\end{block}

  \item   %!TEX root=../puzzle.tex

  \item   \textbf{events}
  \begin{block}
    \item   \noindent \ref{m0:enter} [t] \textbf{event}
\begin{block}
  \item   \textbf{when}
  \begin{block}
  \item[ \eqref{m0:enterent:grd1} ]$\neg t \in in $ %
  \end{block}
  \item   \textbf{begin}
  \begin{block}
  \item[ \eqref{m0:entera1} ]$in \bcmsuch in' = in \bunion \{ t \} $ %
  \item[ \eqref{m0:entera3} ]$loc \bcmsuch loc' = loc \1| t \fun ent $ %
  \end{block}
  \item   \textbf{end} \\
\end{block}

    \item   \noindent \ref{m0:leave} [t] \textbf{event}
\begin{block}
  \item   \textbf{during}
  \begin{block}
  \item[ \eqref{m0:leavelv:c0} ]$t \in in $ %
  \item[ \eqref{m0:leavelv:c1} ]$loc.t = ext $ %
  \end{block}
  \item   \textbf{when}
  \begin{block}
  \item[ \eqref{m0:leavelv:grd0} ]$t \in in $ %
  \item[ \eqref{m0:leavelv:grd1} ]$loc.t = ext $ %
  \end{block}
  \item   \textbf{begin}
  \begin{block}
  \item[ \eqref{m0:leavelv:a0} ]$in \bcmsuch in' = in \setminus \{ t \} $ %
  \item[ \eqref{m0:leavelv:a2} ]$loc \bcmsuch loc' = \{ t \} \domsub loc $ %
  \end{block}
  \item   \textbf{end} \\
\end{block}

    \item   \noindent \ref{m1:movein} [t] \textbf{event}
\begin{block}
  \item   \textbf{during}
  \begin{block}
  \item[ \eqref{m1:moveinmi:c1} ]$t \in in $ %
  \item[ \eqref{m1:moveinmi:c2} ]$loc.t = ent $ %
  \end{block}
  \item   \textbf{any} b
  \item   \textbf{when}
  \begin{block}
  \item[ \eqref{m1:moveinmi:g1} ]$t \in in $ %
  \item[ \eqref{m1:moveinmi:grd0} ]$loc.t = ent $ %
  \item[ \eqref{m1:moveinmi:grd7} ]$b \in plf $ %
  \end{block}
  \item   \textbf{begin}
  \begin{block}
  \item[ \eqref{m1:moveinmi:a2} ]$loc \bcmsuch loc' = loc \1| t \fun b $ %
  \end{block}
  \item   \textbf{end} \\
\end{block}

    \item   \noindent \ref{m1:moveout} [t] \textbf{event}
\begin{block}
\item \textbf{during}
\begin{block}
\item[ \eqref{m1:moveoutc1} ]$t \in in  %
		\1\land loc.t \in plf $ %
\end{block}
\item \textbf{when}
\begin{block}
\item[ \eqref{m1:moveoutmo:g1} ]$t \in in $ %
\item[ \eqref{m1:moveoutmo:g2} ]$loc.t \in plf $ %
\end{block}
\item \textbf{begin}
\begin{block}
\item[ \eqref{m1:moveoutSKIP:in} ]$in' = in$ %
\item[ \eqref{m1:moveouta2} ]$loc' = loc \1 | t \fun ext $ %
\end{block}
\item \textbf{end} \\
\end{block}

  \end{block}
  \item   \textbf{end} \\
\end{block}

    \begin{machine}{m1}
    \refines{m0}
    \[ \variable{c : \Int} \]
    \begin{align}
        \invariant{m1:inv0}{ &\between{0}{c}{v} } \\
        \initialization{m1:in0}{ &c = 0 } 
    \end{align}
    \begin{itemize}
    \comment{c}{Current version of the internal AST}
    \end{itemize}
    \begin{align}
        \invariant{m1:inv1}{c > 0 \land in.c \in \dom.parse 
            \2\implies ast = parse.(in.c) }
    \end{align}
    \[ \variable{ast : AST} \]
    \begin{align}
        \invariant{m1:inv2}{ err &\2\equiv c > 0 
            \1\land \neg in.c \in \dom.parse } \\
        \initialization{m1:in1}{ err &= \false }
    \end{align}
    \[ \variable{ err : \Bool } \]
\noindent \textbf{variables}
\begin{itemize}
    \comment{err}{ Is the input syntactically correct? } 
    \comment{ast}{ Internal syntax tree }
\end{itemize}
\[ \dummy{C : \Int} \]
\noindent \textbf{requirement}
\begin{align}
    \progress{m1:prog0}{c = C}{c > C \1\lor c = v}
\end{align}
\newevent{parse}{PARSE}
\newevent{fail}{FAIL}
\newevent{choose}{READ}
\[ \param{choose}{vv : \Int} ; \quad \variable{ nx : \Int } \]
\begin{align}
    \cschedule{parse}{m1:sch0}{&0 < nx} \\
    \cschedule{parse}{m1:sch1}{&in.nx \in \dom.parse} \\
    \evguard{parse}{m1:grd0}{&0 < nx} \\
    \evguard{parse}{m1:grd1}{&in.nx \in \dom.parse} \\
    \invariant{m1:inv3}{ &\between{c}{nx}{v} } \\
    \initialization{m1:in2}{ &nx = 0 } \\
    \evbcmeq{parse}{m1:act0}{ast}{ parse.(in.nx) } \\
    \evbcmeq{parse}{m1:act1}{c}{nx} \\
    \evbcmeq{parse}{m1:act2}{err}{\false}
\end{align}
% \removecoarse{parse}{default}
% trading
\begin{align*}
\refine{m1:prog0}{trading}{m1:prog1}{}
& \progress{m1:prog1}{c = C \land \neg c = v}{C < c}
\refine{m1:prog1}{transitivity}{m1:prog2,m1:prog3}{}
& \progress{m1:prog2}
    {c = C \land \neg c = v}
    {\betweenR{0}{C}{nx}} \\
& \progress{m1:prog3}{\betweenR{0}{C}{nx}}{C < c} 
\refine{m1:prog3}{disjunction}{m1:prog4,m1:prog5}{}
& \progress{m1:prog4}
    {\betweenR{0}{C}{nx} \land in.nx \in \dom.parse}
    {C < c} \\
& \progress{m1:prog5}
    {\betweenR{0}{C}{nx} \land \neg in.nx \in \dom.parse}
    {C < c}
\refine{m1:prog2}{ensure}{choose}{}
\refine{m1:prog4}{ensure}{parse}{}
\refine{m1:prog5}{ensure}{fail}{}
\end{align*}
\begin{align}
    \cschedule{fail}{m1:sch0}{0 < nx} \\
    \cschedule{fail}{m1:sch1}{\neg in.nx \in \dom.parse} \\
    \evguard{fail}{m1:grd0}{0 < nx} \\
    \evguard{fail}{m1:grd1}{\neg in.nx \in \dom.parse} \\
    \evbcmeq{fail}{m1:act0}{c}{nx} \\
    \evbcmeq{fail}{m1:act1}{err}{\true}
\end{align}
    % \removecoarse{fail}{default}
\begin{align}
    \cschedule{choose}{m1:sch0}{c < v} \\
    \evguard{choose}{m1:grd0}{\betweenL{c}{vv}{v}} \\
    \cschedule{choose}{m1:sch1}{c = nx} \\
    \evguard{choose}{m1:grd1}{c = nx} \\
    \evbcmeq{choose}{m1:act0}{nx}{vv}
\end{align}
    % \removecoarse{choose}{default}
\end{machine}
%!TEX root=../main9.tex
\begin{block}
  \item   \textbf{machine} m2
  \item   \textbf{refines} m0
  \item   \textbf{variables}
  \begin{block}
    \item   $req$
    \item   $req0$
    \item   $reqA$
    \item   $reqB$
  \end{block}
  \item   %!TEX root=../main9.tex
\textbf{invariants}
\begin{block}
\item[ \eqref{m1:inv0} ]{$reqA \bunion reqB = req$} %
\item[ \eqref{m1:inv1} ]{$reqA \binter reqB = req$} %
\end{block}

  \item   \textbf{events}
  \begin{block}
    \item   %!TEX root=../main9.tex
\noindent \ref{handle}  \textbf{event}
\begin{block}
  \item   \textbf{during}
  \begin{block}
  \item[ \eqref{handlem0:sch0} ]{$\neg req = \emptyset$} %
  \end{block}
  \item   \textbf{any} r
  \item   \textbf{when}
  \begin{block}
  \item[ \eqref{handlegrd0} ]{$r \in req$} %
  \end{block}
  \item   \textbf{begin}
  \begin{block}
  \item[ \eqref{handleact0} ]{$req \bcmeq req \setminus \{ r \}$} %
  \item[ \eqref{handleact1} ]{$req0 \bcmeq req$} %
  \end{block}
  \item   \textbf{end} \\
\end{block}

    \item   %!TEX root=../main9.tex
\noindent \ref{req}  \textbf{event}
\begin{block}
  \item   \textbf{during}
  \begin{block}
  \item[ (\ref{req}/default) ]{$\false$} %
  \end{block}
  \item   \textbf{any} r
  \item   \textbf{when}
  \begin{block}
  \item[ \eqref{reqgrd0} ]{$\neg r \in req$} %
  \end{block}
  \item   \textbf{begin}
  \begin{block}
  \item[ \eqref{reqact0} ]{$req \bcmeq req \bunion \{ r \}$} %
  \item[ \eqref{reqact1} ]{$req0 \bcmeq req$} %
  \end{block}
  \item   \textbf{end} \\
\end{block}

  \end{block}
  \item   \textbf{end} \\
\end{block}

\begin{machine}{m2}
    \refines{m1}

    \newset{Name}
    \newset{Data}
    \[\constant{data : FS \pfun Name \pfun Data}\]
    \[\constant{dep : Data \pfun \set [Name] }\]
    \[\constant{depG : FS \pfun \set [Pair[Name,Name]] }\]
    \begin{align}
        \assumption{asm0}{ \dom.data = FS } \\
        \assumption{asm1}{ \qforall{f}{}{f \in \dom.depG \3\equiv \qforall{m}{m \in \dom.(data.f)}{data.f.m \in \dom.dep}} } \\
        \assumption{asm2}{ \qforall{m}{m \implies \true}{m \lor \false} }
    \end{align}
\end{machine}
\end{document}