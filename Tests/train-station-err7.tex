
\documentclass[12pt]{amsart}
\usepackage{geometry} % see geometry.pdf on how to lay out the page. There's lots.
\usepackage{bsymb}
\usepackage{unitb}
\usepackage{calculation}
\usepackage{ulem}
\normalem
\geometry{a4paper} % or letter or a5paper or ... etc
% \geometry{landscape} % rotated page geometry

% See the ``Article customise'' template for some common customisations

\title{}
\author{}
\date{} % delete this line to display the current date

%%% BEGIN DOCUMENT
\begin{document}

\maketitle
\tableofcontents

%\section{}
%\subsection{}

\begin{machine}{train0}

\newset{TRAIN}{\TRAIN}
\newset{BLK}{\BLK}
%\newset{LOC}{\LOC}
\newset{LOC}{\LOC}
%\end{set}

%
%\hide{
	\begin{variable}
		in : \set[\TRAIN]
	\end{variable}
%}
%

\newevent{enter}

\newevent{leave}

\begin{align*}
\transient{leave}{0}{tr0}
{	t \in in	}
\end{align*}

\begin{dummy}
	t,t_0,t_1,t_2,t_3 : \TRAIN
\end{dummy}

\begin{dummy}
	p,q : \BLK
\end{dummy}


\begin{variable}
	loc : \TRAIN \pfun \BLK
\end{variable}

\begin{align*}
\invariant{inv2}
%	loc \in in \tfun \BLK
{	\dom . loc = in	}
\end{align*}

\begin{align*}
\evassignment{enter}{a1}
{	in' = in \bunion \{ t \}	}
\end{align*}

\begin{proof}{enter/INV/inv2}
	\begin{calculation}
		in'
	\hint{=}{ \ref{a1} }
		in \bunion \{ t \}
	\hint{=}{ \ref{inv2} }
		\dom.loc \bunion \{ t \}
	\hint{=}{ function calculus }
		\dom.loc \bunion \dom.(t \tfun ent)
	\hint{=}{ $\dom$ over $|$ }
		\dom.(loc ~  | ~  t \tfun ent)
	\hint{=}{ \ref{a2} }
		\dom. ( loc' )
	\end{calculation}
\end{proof}

\begin{align*}
\evassignment{enter}{a2}
{	loc' = loc | (t \tfun ent)	}
\end{align*}

\begin{proof}{leave/INV/inv2}
	\begin{calculation}
		in'
	\hint{=}{ \ref{a0} }
		in \setminus \{ t \}
	\hint{=}{ \ref{inv2} }
		\dom.loc \setminus \{ t \}
	\hint{=}{ domain of domain subtraction }
		\dom.(\{ t \} \domsub loc)
	\hint{=}{ \ref{a3} } 
		\dom. loc' 
	\end{calculation}
\end{proof}

\begin{align*}
\evassignment{leave}{a3}
{	loc' = \{ t \} \domsub loc 	}
\end{align*}

\begin{align*}
\cschedule{leave}{c0}
{	t \in in	}
\end{align*}

\begin{align*}
\initialization{in0}
{	in = \emptyset 	}
\end{align*}
\begin{align*}
\initialization{in1}
{	loc = \emptyfun	}
\end{align*}

\begin{use:set}{\TRAIN} \end{use:set}
\begin{use:set}{\LOC} \end{use:set}
\begin{use:set}{\BLK} \end{use:set}
\begin{use:fun}{\TRAIN}{\BLK} \end{use:fun}
%\begin{use:fun}{\TRAIN}{\LOC} \end{use:fun}

\begin{indices}{leave}
	t: \TRAIN
\end{indices}

\begin{indices}{enter}
	t: \TRAIN
\end{indices}

\begin{align*}
\evassignment{leave}{a0}
%	in' = in | (t \rightarrow \false)
{	in' = in \setminus \{ t \}	}
%	in'.t = \false 
\end{align*}

\begin{constant}
	ent,ext : \BLK
\end{constant}
\begin{constant}
	PLF : \set [\BLK]
\end{constant}

\begin{align*}
\assumption{axm0}
{	\BLK = \{ ent, ext \} \bunion PLF	} \\
\safety{s0}{ \neg t \in in }{ t \in in \land loc.t = ent } \\
\safety{s1}{ t \in in \land loc.t = ent }{ t \in in \land loc.t \in PLF }
\end{align*}

%
\begin{align*}
\constraint{co0}
{	\neg t \in in \land t \in in' \implies  loc'.t = ent } % \lor \neg t \in in'
\end{align*}
%
\begin{proof}{leave/CO/co0}
	\begin{free:var}{t}{t_0}
	
	\noindent Assuming
	\begin{align}
	\assume{hyp0}{ t_0 \in in \land \neg  t_0 \in in' }
	\end{align}
	we now have to prove $\goal{ loc'.t_0 = ent }$.
	
	\begin{calculation}
		loc'.t_0 = ent
	\hint{\follows}{ }
		\false
	\hint{=}{ \eqref{hyp0} }
		\true
%		\neg t_0 \in in \land t_0 \in in'
%	\hint{=}{ \ref{grd0} }
%		\neg t_0 = t \land \neg t_0 \in in \land t_0 \in in' 
%	\hint{=}{ \ref{a0} }
%		\neg t_0 = t \land \neg t_0 \in in \land t_0 \in in \bunion \{ t \} 
%	\hint{=}{ contradiction }
%		\false
%	\hint{\implies}{ $\false \implies p$ }
%		loc'.t_0 = ent
	\end{calculation}
	\end{free:var}
\end{proof}
%
\begin{align*}
\constraint{co1}
{	 t \in in \land loc.t = ent  \land \neg loc.t \in PLF 
\implies t \in in' \land (loc'.t \in PLF \lor loc'.t = ent)	}
\end{align*}

\discharge{s0}{co0}
\discharge{s1}{co1}

% test the error message for
%\begin{guard}{co0}
%
%\end{guard}

\begin{align*}
\evguard{leave}{grd0}
{	loc.t = ext \land t \in in	}
\end{align*}

\begin{align*}
\evguard{enter}{grd1}
{	\neg t \in in	}
\end{align*}

\begin{proof}{enter/CO/co0}
	% assume 
	% forall
	% case
	% split
	% TODO: add a case distinction that isn't complete
	\begin{free:var}{t}{t_0}

	\begin{by:cases}

	\begin{case}{h0}{t_0 = t}
	\begin{calculation}
		\neg t_0 \in in \land t_0 \in in' \implies  loc'.t_0 = ent 
	\hint{=}{ \eqref{h0} and \ref{a1}  }
		\neg t \in in \implies  loc'.t = ent 
	\hint{=}{ \ref{a2} }
		\true
	\end{calculation}

	\end{case}

	\begin{case}{h1}{ \neg t_0 = t }
%	\easy
	\begin{calculation}
		\neg t_0 \in in \land t_0 \in in' 
	\hint{=}{ \ref{a1}  }
		\neg t_0 \in in \land  t_0 \in in \bunion \{ t \}
	\hint{=}{ \eqref{h1} }
		\neg t_0 \in in \land  t_0 \in in
	\hint{=}{ contradiction }
		\false
	\hint{\implies}{ pred. calc. }
		loc'.t_0 = ent 
	\end{calculation}

	\end{case}

	\end{by:cases}
	\end{free:var}
\end{proof}

\begin{proof}{leave/CO/co1}
	\begin{free:var}{t}{t_0}
	\begin{calculation}
		t_0 \in in' \land (loc'.t_0 \in PLF \lor loc'.t_0 = ent)
	\hint{=}{  \ref{a0}  and \ref{a3} } % step 1
		t_0 \in in \setminus \{ t \} 
		\land ( (\{ t \} \domsub loc).t_0 \in PLF \lor (\{ t \} \domsub loc).t_0 = ent)
	\hint{=}{ }	% step 2
		( t_0 \in in \setminus \{ t \} 
		\land  ( \{ t \} \domsub loc).t_0 \in PLF )
			   \lor ( t_0 \in (in \setminus \{ t \} )
		\land (\{ t \} \domsub loc).t_0 = ent)
	\hint{=}{ \ref{inv2} }	% step 3
			t_0 \in in \setminus \{ t \} 
		\land ( loc.t_0 \in PLF \lor  loc.t_0 = ent)
	\hint{=}{ } % step 4
		t_0 \in in \land \neg t_0  = t 
		\land ( loc.t_0 \in PLF \lor  loc.t_0 = ent)
	\hint{=}{  } % step 5
		t_0 \in in \land \neg t_0  = t \land \neg loc.t_0 = ext
	\hint{=}{  \ref{grd0} } % step 6
	 	t_0 \in in \land \neg loc.t_0 = ext \land loc.t = ext
	\hint{\follows}{  \ref{grd0} } % step 7
%		\true
	 	t_0 \in in \land loc.t_0 = ent 
	\hint{=}{   } % step 8
	 	t_0 \in in \land loc.t_0 = ent  \land \neg loc.t_0 \in PLF 
	\end{calculation}
	\end{free:var}
\end{proof}

\begin{align*}
\invariant{inv1}
{	\qforall{t}{t \in in}{loc.t \in \BLK}	}
\end{align*}

\begin{align*}\assumption{asm2}
{	\neg ent = ext \land \neg ent \in PLF \land \neg ext \in PLF	}
\end{align*}

\begin{align*}\assumption{asm3}
{	\qforall{p}{}{ \neg p = ext \equiv p \in \{ent\} \bunion PLF }	}
\end{align*}

\begin{align*}\assumption{asm4}
{	\qforall{p}{}{ \neg p = ent \equiv p \in \{ext\} \bunion PLF }	}
\end{align*}

\begin{align*}\assumption{asm5}
{	\qforall{p}{}{ p = ent \lor p = ext \equiv \neg p \in PLF }	}
\end{align*}

\weakento{leave}{0}{default}{c0}

\end{machine}

\end{document}