
\documentclass[12pt]{amsart}
\usepackage{geometry} % see geometry.pdf on how to lay out the page. There's lots.
\geometry{a4paper} % or letter or a5paper or ... etc
% \geometry{landscape} % rotated page geometry
\usepackage{unitb}

% See the ``Article customise'' template for come common customisations

\title{A Small Machine}
\author{Simon Hudon}
\date{} % delete this line to display the current date

%%% BEGIN DOCUMENT
\begin{document}

\maketitle

As an example, we look

\begin{machine}{m0}

\newevent{inc}

\begin{align*}
\invariant{inv0}
{	x = 2 \cdot y	}
\end{align*}

\begin{align*}
\evassignment{inc}{a0}
{	x' = x + 2	}
\end{align*}

\begin{align*}
\initialization{a3}
{	x = 0	}
\end{align*}

%\begin{invariant}{inv1}
%	x
%\end{invariant}

\end{machine}
This is outside a machine's scope

\begin{machine}{m0}

\begin{align*}
\initialization{a2}
{	y = 0	}
\end{align*}

\begin{align*}
\evassignment{inc}{a1}
{	y' = y + 1	}
\end{align*}

\begin{align*}
\variable{	x,y: \Int}
\end{align*}

\begin{align*}
\fschedule{inc}{f0}
{	x = y	}
\end{align*}

\begin{align*}
\transientB{inc}{0}{tr0}{ \lt{prog0} }
{	x = y	}
\end{align*}

\begin{align*}
\invariant{inv1}
{	x \le 10	}
\\ \progress{prog0}{ \true }{ x = y }
\end{align*}

\replacefine{inc}{0}{}{}{f0}{prog0}

\end{machine}
\end{document}