\documentclass[12pt]{amsart}
\usepackage{geometry} % see geometry.pdf on how to lay out the page. There's lots.
\usepackage{bsymb}
\usepackage{unitb}
\usepackage{calculational}
\usepackage{ulem}
\usepackage{hyperref}
\normalem
\geometry{a4paper} % or letter or a5paper or ... etc
% \geometry{landscape} % rotated page geometry

% See the ``Article customise'' template for some common customisations

\title{}
\author{}
\date{} % delete this line to display the current date

%%% BEGIN DOCUMENT
\setcounter{tocdepth}{4}
\begin{document}

\maketitle
\tableofcontents

\newcommand{\G}{\text{G}}
\renewcommand{\H}{\text{H}}

\section{Initial model}
\begin{machine}{m0}

\newset{\G}
\newset{\H}

\with{sets}
\with{functions}

\begin{align*}
\dummy{x, x_0, x_1 : \set[\G]} \\
\dummy{y,y_0,y_1, z: \set[\H]}
\end{align*}
\begin{align*}
\constant{f : \set [\G] \pfun \set [\H]} \\
\constant{g : \set [\H] \pfun \set [\G]} \\
\end{align*}
\begin{align*}
\assumption{asm0}{ \qforall{x,y}{}{ f.x \subseteq y \2\equiv x \subseteq g.y } } \\
\invariant{inv0}{ \qforall{x_0,x_1}{}{ f.(x_0 \bunion x_1) \1 = f.x_0 \bunion f.x_1 } }
\end{align*}

\begin{proof}{INIT/INV/\ref{inv0}}
	\begin{free:var}{x_0}{x_0}
	\begin{free:var}{x_1}{x_1}
	\begin{indirect:equality}{right}{\subseteq}{z}
%	\easy
	\begin{calculation}
		f.(x_0 \bunion x_1) \1\subseteq z
	\hint{=}{ \eqref{asm0} }
		x_0 \bunion x_1 \1\subseteq g.z
	\hint{=}{ $\bunion$ and $\subseteq$ }
		x_0 \subseteq g.z \2\land x_1 \subseteq g.z
	\hint{=}{ \eqref{asm0} }
		f.x_0 \subseteq z \2\land f.x_1 \subseteq z
	\hint{=}{ $\bunion$ and $\subseteq$ }
		f.x_0 \bunion f.x_1 \1\subseteq z
	\end{calculation}
	\end{indirect:equality}
	\end{free:var}
	\end{free:var}
\end{proof}

\begin{align*}
\invariant{inv1}{ \qforall{y_0,y_1}{}{ y_0 = y_1 \2\equiv \qforall{z}{}{ y_0 \subseteq z \1\equiv y_1 \subseteq z } } }
\end{align*}

\begin{proof}{INIT/INV/\ref{inv1}}
	\begin{free:var}{y_0}{y_0}
	\begin{free:var}{y_1}{y_1}
	\begin{by:parts} \\
		\begin{part:a}{y_0 = y_1 \2\implies \qforall{z}{}{ y_0 \subseteq z \1\equiv y_1 \subseteq z }}
		\easy 
%		banana
		\end{part:a}

		\begin{part:a}{ \qforall{z}{}{ y_0 \subseteq z \1\equiv y_1 \subseteq z } \2\implies y_0 = y_1 }
		\begin{align}
		\assume{hyp0}{\qforall{z}{}{ y_0 \subseteq z \1\equiv y_1 \subseteq z }} \\
		\goal{ y_0 = y_1 }
		\end{align}
		\begin{calculation}
			y_0 = y_1
		\hint{=}{ antisymmetry of $\subseteq$ }
			y_0 \subseteq y_1 \1\land y_1 \subseteq y_0
		\hint{=}{ \eqref{hyp0} }
%		\hint{=}{  }
			y_1 \subseteq y_1 \1\land y_0 \subseteq y_0
		\hint{=}{ reflexivity of $\subseteq$ }
			\true
		\end{calculation}
		\end{part:a}
	\end{by:parts}
	\end{free:var}
	\end{free:var}
\end{proof}

\end{machine}

\begin{context}{ctx0}

\newset{\G}

\[ \dummy{x,y,y_0,y_1,z: \G} \]
\with{functions}

\operator{\le}{ le : Pair [ \G, \G ] \pfun \Bool }

\begin{align}
\axiom{axm0}{ 
	\qforall{x,y}{}{ x \le y \1\land y \le x \2\equiv x = y } } \\
\axiom{axm1}{ 
	\qforall{x,y,z}{}{ x \le y \1\land y \le z \2\implies x \le z } } \\
\theorem{thm0}{ 
	\qforall{x,y}{}{ x = y \2\equiv \qforall{z}{}{ z \le x \1\equiv z \le y } } }
\\
\theorem{thm1}{ 
	\qforall{x,y}{}{ x \le y \2\equiv \qforall{z}{}{ z \le x \1\implies z \le y } } }
\end{align}

\begin{proof}{THM/\ref{thm0}}
	\begin{free:var}{x}{x}
	\begin{free:var}{y}{y}
	\begin{by:parts} \\
		\begin{part:a}{x = y \2\implies \qforall{z}{}{ x \le z \1\equiv y \le z }}
		\easy 
%		It's easy: check this out
%		banana
		\end{part:a}

		\begin{part:a}{ \qforall{z}{}{ x \le z \1\equiv y \le z } \2\implies x = y }
		\begin{align}
		\assume{hyp0}{\qforall{z}{}{ x \le z \1\equiv y \le z }} \\
		\goal{ x = y }
		\end{align}
		\begin{calculation}
			x = y
		\hint{=}{ antisymmetry of $\le$ \eqref{axm0} }
			x \le y \1\land y \le x
		\hint{=}{ \eqref{hyp0} }
%		\hint{=}{  }
			y \le y \1\land x \le x
		\hint{=}{ reflexivity of $\le$ \eqref{axm0} }
			\true
		\end{calculation}
		\end{part:a}
	\end{by:parts}
	\end{free:var}
	\end{free:var}
\end{proof}

\begin{proof}{THM/\ref{thm1}}
	\begin{free:var}{x}{x}
	\begin{free:var}{y}{y}
	\begin{by:parts} \\
	\begin{part:a}{ \qforall{z}{}{x \le y \2\implies (z \le x \1\implies z \le y) } }
	\begin{free:var}{z}{z}
		\begin{calculation}
			z \le x \1\implies z \le y
		\hint{\follows}{ \eqref{axm1} }
			x \le y
		\end{calculation}
	\end{free:var}
	\end{part:a}
	\begin{part:a}{x \le y \2\follows \qforall{z}{}{ z \le x \1\implies z \le y } }
		\begin{calculation}
			\qforall{z}{}{ z \le x \1\implies z \le y }
		\hint{\implies}{ instantiation with $z := x$ }
			x \le x \1\implies x \le y
		\hint{=}{ \eqref{axm0} }
			x \le y
		\end{calculation}
	\end{part:a}
	\end{by:parts}
	\end{free:var}
	\end{free:var}
\end{proof}
\end{context}

\begin{context}{ctx1}

\newset{\G}

\[ \dummy{x,y,y_0,y_1,z,k: \G} \]
\with{functions}

\operator{\le}{ le : Pair [ \G, \G ] \pfun \Bool }
\operator{\uparrow}{ up : Pair [ \G, \G ] \pfun \G }
\precedence{[[\uparrow],[\le]]}
\begin{align}
\axiom{ctx1:axm0}{ 
	\qforall{x,y}{}{ x \le y \1\land y \le x \2\equiv x = y } } 
\\ \axiom{ctx1:axm1}{ 
	\qforall{x,y,z}{}{ x \le z \1\land y \le z \2\equiv 
		x \uparrow y \le z } }
\\ \axiom{ctx1:axm2}{ 
	\qforall{x,y}{}{ x = y \2\equiv \qforall{z}{}{ z \le x \1\equiv z \le y } } }
\\ \axiom{ctx1:axm3}{ 
	\qforall{x,y}{}{ x = y \2\equiv \qforall{z}{}{ x \le z \1\equiv y \le z } } }
\\ \theorem{ctx1:thm3}{ \qforall{x,y}{}{ y \le x \uparrow y } }
\end{align}

\begin{proof}{THM/\ref{ctx1:thm3}}
	\begin{free:var}{x}{x}
	\begin{free:var}{y}{y}
%\begin{indirect:equality}{right}{\le}{z}
\begin{calculation}
		y \le x \uparrow y
	\hint{\follows}{ pred. calc. }
		x \le x \uparrow y
	\2\land	y \le x \uparrow y
	\hint{=}{ \eqref{ctx1:axm1} }
		x \uparrow y \1\le x \uparrow y
	\hint{=}{ \eqref{ctx1:axm0} }
		\true
\end{calculation}
%\end{indirect:equality}
	\end{free:var}
	\end{free:var}
\end{proof}

\begin{align}
\theorem{ctx1:thm4}{ \qforall{x,y,z}{}{ (x \uparrow y) \uparrow z \1= x \uparrow (y \uparrow z) } }
\end{align}

\begin{proof}{THM/\ref{ctx1:thm4}}
	\begin{free:var}{x}{x}
	\begin{free:var}{y}{y}
	\begin{free:var}{z}{z}
\begin{indirect:equality}{right}{\le}{k}
\begin{calculation}
		(x \uparrow y) \uparrow z \1\le k
	\hint{=}{ \eqinst{ctx1:axm1}{ \subst{x}{x \uparrow y}, \subst{y}{z} and \subst{z}{k} } }
		(x \uparrow y) \1\le k \2\land z  \1\le k
	\hint{=}{ \eqref{ctx1:axm1} }
		x \le k \2\land y \le k \2\land z  \le k
	\hint{=}{ \eqref{ctx1:axm1} }
		x \1\le k \2\land y \uparrow z \1\le k
	\hint{=}{ \eqinst{ctx1:axm1}{ \subst{x}{x \uparrow y}, 
			\subst{y}{z} and \subst{z}{k} } }
		x \uparrow (y \uparrow z) \1\le k
\end{calculation}
\end{indirect:equality}
	\end{free:var}
	\end{free:var}
	\end{free:var}
\end{proof}

\begin{align}
\theorem{ctx1:thm5}{ \qforall{x,y}{}{ x \uparrow y = y \2\equiv x \le y } }
\end{align}

\begin{proof}{THM/\ref{ctx1:thm5}}
	\begin{free:var}{x}{x}
	\begin{free:var}{y}{y}
%\begin{indirect:equality}{right}{\le}{k}
\begin{calculation}
		x \uparrow y = y
	\hint{=}{ \inst{ctx1:axm0}{ \subst{x}{x \uparrow y} } }
		x \uparrow y \le y \2\land y \le x \uparrow y
	\hint{=}{ \eqref{ctx1:thm3} }
		x \uparrow y \le y
	\hint{=}{ \inst{ctx1:axm1}{ \subst{z}{y} } }
		x  \le y \2\land y \le y
	\hint{=}{ \eqinst{ctx1:axm0}{ \subst{x}{y}, \subst{y}{y}} }
		x  \le y
\end{calculation}
%\end{indirect:equality}
	\end{free:var}
	\end{free:var}
\end{proof}

\begin{align}
\theorem{ctx1:thm6}{ \qforall{x,y,z}{x\le y \land y \le z}{ x \le z } }
\end{align}

\begin{proof}{THM/\ref{ctx1:thm6}}
	\begin{free:var}{x}{x}
	\begin{free:var}{y}{y}
	\begin{free:var}{z}{z}
%\begin{indirect:equality}{right}{\le}{k}
\begin{calculation}
		y \le z \1\implies x \le z
	\hint{=}{ pred. calc. }
		x \le z \1\land y \le z \2\equiv y \le z
	\hint{=}{ \eqinst{ctx1:axm1}{ \subst{x}{x}, \subst{y}{y} and \subst{z}{z}  } }
		x \uparrow y \le z \2\equiv y \le z
	\hint{\follows}{ Leibniz }
		x \uparrow y = y
	\hint{=}{ \eqref{ctx1:thm5} }
		x  \le y
\end{calculation}
%\end{indirect:equality}
	\end{free:var}
	\end{free:var}
	\end{free:var}
\end{proof}

\[ \dummy{ f : \G \pfun \G } \]
%
\begin{align}
\theorem{ctx1:thm7}{ \qforall{f}{\qforall{x,y}{}{ f.(x \uparrow y) = f.x \uparrow f.y } }{ \qforall{x,y}{ x \le y }{ f.x \le f.y } } }
\end{align}

\begin{proof}{THM/\ref{ctx1:thm7}}
	\begin{free:var}{f}{f}
	\begin{align}
	\assume{hyp0}{ \qforall{x,y}{}{ 
		f.(x \uparrow y) \1= f.x \uparrow f.y } } 
	\hide{ \goal{ \qforall{x,y}{ x \le y }{ f.x \le f.y } } }
	\end{align}
	\begin{free:var}{x}{x}
	\begin{free:var}{y}{y}
%	\begin{align}
%	\assume{hyp1}{ x\le y } 
%	\end{align}
%\begin{indirect:equality}{right}{\le}{k}
\begin{calculation}
		f.x \le f.y
	\hint{=}{ \eqref{ctx1:thm5} }
		f.x \uparrow f.y \1= f.y
	\hint{=}{ \eqref{hyp0} }
		f.(x \uparrow y) \1 = f.y
	\hint{\follows}{ Leibniz }
		x \uparrow y \1= y
	\hint{=}{ \eqref{ctx1:thm5} }
		x  \le y
\end{calculation}
%\end{indirect:equality}
	\end{free:var}
	\end{free:var}
	\end{free:var}
\end{proof}

\begin{align}
\theorem{ctx1:thm8}{ \qforall{x,y}{}{ x\le y \2\equiv 
	\qforall{z}{}{x \le z \1\follows y \le z } } }
\end{align}

\begin{proof}{THM/\ref{ctx1:thm8}}
	\begin{free:var}{x}{x}
	\begin{free:var}{y}{y} \\
\begin{by:parts}
\begin{part:a}{ x\le y \2\implies 
		\qforall{z}{}{x \le z \1\follows y \le z } }
%	\begin{align}
%	\assume{hyp5}{ x\le y }
%	\end{align}
%	\hide{ \goal{ \qforall{z}{}{x \le z \1\follows y \le z } } }
%\begin{free:var}{z}{z}
\begin{calculation}
		\qforall{z}{}{y \le z \1\implies x \le z}
	\hint{\follows}{ \eqref{ctx1:thm6} }
		x  \le y
%	\hint{=}{ \eqref{hyp5} }
%		\true
\end{calculation}
%\end{free:var}
\end{part:a}
\begin{part:a}{ x \le y \2\follows
		\qforall{z}{}{x \le z \1\follows y \le z } }
	\begin{align}
	\assume{hyp6}{ \qforall{z}{}{x \le z \1\follows y \le z } }
	\end{align}
	\hide{ \goal{ x \le y } }
\begin{calculation}
		x \le y
	\hint{\follows}{ \eqref{hyp6} }
		y  \le y
	\hint{=}{ \eqref{ctx1:axm0} }
		\true
\end{calculation}
\end{part:a}
\end{by:parts}
	\end{free:var}
	\end{free:var}
\end{proof}

\begin{align}
\theorem{ctx1:thm9}{ \qforall{x,y,z}{ }{ z \1\le x \uparrow y \2\follows z \le x \1\lor z\le y  } }
\end{align}

\begin{proof}{THM/\ref{ctx1:thm9}}
	\begin{free:var}{z}{z}
	\begin{free:var}{x}{x}
	\begin{free:var}{y}{y}
	\begin{align}
%	\assume{hyp2}{  } \\
	\assume{hyp1}{ z \le x \1\lor z\le y } 
	\end{align}
	\hide{ \goal{ z \le x \uparrow y } }
%	\using{ \eqref{ctx1:thm10} }
\begin{by:symmetry}{hyp1}{x,y}
\begin{indirect:inequality}{right}{\le}{k}
\begin{calculation}
		x \uparrow y \1\le k
	\hint{=}{ \eqref{ctx1:axm1} }
		x \le k \1\land y \le k
	\hint{\implies}{ weakening }
		x \le k
	\hint{\implies}{ \eqref{hyp1} with \eqref{ctx1:thm6} }
		z \le k
\end{calculation}
\end{indirect:inequality}
\end{by:symmetry}
	\end{free:var}
	\end{free:var}
	\end{free:var}
\end{proof}

\begin{align}
\theorem{ctx1:thm10}{ \qforall{x,y}{}{ x \uparrow y \1= y \uparrow x } }
\end{align}

\end{context}

\begin{context}{ctx2}

\with{functions}
\with{arithmetic}
\begin{align*}
\constant{swap : Pair [Pair [ \Int, \Int ], \Int ] \pfun \Int } \\
\dummy{ f : \Int \pfun \Int } \\
\dummy{ i,j,k : \Int } 
\end{align*}
\begin{align*}
\axiom{axm0}{ \qforall{i,j,k}{}{ swap.((i \mapsto j) \mapsto k) = swap.((j \mapsto i) \mapsto k) } } \\
\axiom{axm1}{ \qforall{i,j}{}{ swap.((i \mapsto j) \mapsto i) = j } } \\
\axiom{axm2}{ \qforall{i,j}{}{ swap.((i \mapsto j) \mapsto j) = i } } \\
\axiom{axm3}{ \qforall{i,j,k}{\neg k = i \1\land \neg k = j}{ swap.((i \mapsto j) \mapsto k) = k } } \\
\theorem{thm4}{ \qforall{i,j,k}{i \le k \1\land j \le k}{  ~ ~ swap.((i \mapsto j) \mapsto k) \le k } } \\
\end{align*}

\begin{proof}{THM/\ref{thm4}}
\begin{free:var}{i}{i}
\begin{free:var}{j}{j}
\begin{free:var}{k}{k}
	\begin{align}
	\assume{hyp0}{i \le k} \\
	\assume{hyp1}{j \le k} \\
	\goal{ swap.((i \mapsto j) \mapsto k) \le k }
	\end{align}
	\noindent
	\begin{by:cases} \\
	\begin{case}{hyp2}{\neg i = k \land \neg j = k}
		\easy
	\end{case} \\
	\begin{case}{hyp3}{i = k \lor j = k}
		By symmetry on $i$ and $j$.
		\begin{by:symmetry}{hyp3}{i,j}
		\easy
%		\begin{calculation}
%			swap.((i \mapsto j) \mapsto k) \le k
%		\hint{=}{ \eqref{hyp3} }
%			swap.((i \mapsto j) \mapsto i) \le k
%		\hint{=}{ \eqref{axm1} }
%			j \le k
%		\hint{=}{ \eqref{hyp0} }
%			\true
%		\end{calculation}
		\end{by:symmetry} \\
	\noindent \end{case} \\
	\end{by:cases}
\end{free:var}
\end{free:var}
\end{free:var}
\end{proof}

\end{context}

\end{document}