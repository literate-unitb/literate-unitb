\documentclass[12pt]{amsart}
\usepackage[margin=0.5in]{geometry} 
    % see geometry.pdf on how to lay out the page. There's lots.
\usepackage{../bsymb}
\usepackage{../unitb}
\usepackage{../calculation}
\usepackage{ulem}
\usepackage{hyperref}
\normalem
\geometry{a4paper} % or letter or a5paper or ... etc
% \geometry{landscape} % rotated page geometry

% See the ``Article customise'' template for some common
% customisations

\title{}
\author{}
\date{} % delete this line to display the current date

%%% BEGIN DOCUMENT
\newcommand{\lookup}[2]{#1[#2]}
\setcounter{tocdepth}{4}
\begin{document}
	\begin{block}
  \item   \textbf{machine} m0
  \item   \textbf{variables}
  \begin{block}
    \item   $in$
  \end{block}
  \item   \textbf{progress}
\begin{block}
\item[ \eqref{m0:prog0} ]$t \in in  \quad \mapsto\quad \neg t \in in $ %
\end{block}

  \item   \textbf{transient}
\begin{block}
\item[ \eqref{m0:tr0} ]$t \in in  \qquad \text{(\ref{m0:leave}: [t := t' | t' = t])}$ %
\end{block}

  \item   \textbf{events}
  \begin{block}
    \item   \noindent \ref{m0:enter} [t] \textbf{event}
\begin{block}
\item \textbf{during}
\begin{block}
\item[ \eqref{m0:enterdefault} ]$\false$ %
\end{block}
\item \textbf{begin}
\begin{block}
\item[ \eqref{m0:entera1} ]$in' = in \bunion \{ t \} $ %
\end{block}
\item \textbf{end} \\
\end{block}

    \item   \noindent \ref{m0:leave} [t] \textbf{event}
\begin{block}
  \item   \textbf{during}
  \begin{block}
  \item[ \eqref{m0:leavelv:c0} ]$t \in in $ %
  \end{block}
  \item   \textbf{begin}
  \begin{block}
  \item[ \eqref{m0:leavelv:a0} ]$in \bcmsuch in' = in \setminus \{ t \} $ %
  \end{block}
  \item   \textbf{end} \\
\end{block}

  \end{block}
  \item   \textbf{end} \\
\end{block}

\begin{machine}{m0}
	
	\with{sets}
	\newset{Node}
	\[ \constant{ r : Node } \]
	\[ \variable{ free,live: \set[Node] } \]
\begin{align}
	\invariant{m0:inv0}{r \in live} \\
	\initialization{m0:in0}{ live = \{ r\} }
\end{align}
	\newevent{alloc}{ALLOCATE}
	\newevent{free}{FREE}
\begin{align}
	\evbcmeq{alloc}{m0:act0}{live}{ live \bunion \{ p \} } \\
	\evbcmeq{free}{m0:act0}{free}{ free \bunion \{ p \} } \\
	\invariant{m0:inv1}{ Node = live \bunion free } \\
	\initialization{m0:in1}{ free = Node \setminus \{r\} } \\
	\invariant{m0:inv2}{ live \binter free = \emptyset } \\
	\evbcmeq{alloc}{m0:act1}{free}{ free \setminus \{ p \} } \\
	\evbcmeq{free}{m0:act1}{live}{ live \setminus \{ p \} } \\
	\evguard{free}{m0:grd0}{ \neg p = r }
\end{align}
	\[ \param{alloc}{p : Node} \]
	\[ \indices{free}{p : Node} \]
\end{machine}

	\begin{block}
  \item   \textbf{machine} m1
  \item   \textbf{variables}
  \begin{block}
    \item   $in$
    \item   $loc$
  \end{block}
  \item   %!TEX root=../main8.tex
\textbf{invariants}
\begin{block}
\item[ \eqref{m1:inv0} ]{$qe \in \intervalR{p}{q} \tfun \G $} %
\item[ \eqref{m1:inv1} ]{$p \le q $} %
\end{block}

  \item   \textbf{progress}
\begin{block}
\item[ \eqref{m1:prog0} ]$r \in \dom.pshL \1\land pshL.r = x  \quad \mapsto\quad p < q \land qe.p = x \1\land \neg r \in \dom.pshL $ %
\item[ \eqref{m1:prog1} ]$r \in \dom.pshR \1\land pshR.r = x  \quad \mapsto\quad p < q \1\land qe.(q-1) = x \1\land \neg r \in \dom.pshR $ %
\item[ \eqref{m1:prog2} ]$r \in popR  \quad \mapsto\quad \neg r \in popR $ %
\item[ \eqref{m1:prog3} ]$r \in popL  \quad \mapsto\quad \neg r \in popL  $ %
\end{block}

  \item   \textbf{safety}
\begin{block}
\item[ \eqref{m1:saf0} ]$\neg t \in in  \textbf{\quad unless \quad} t \in in \land loc.t = ent $ %
\item[ \eqref{m1:saf1} ]$t \in in \land loc.t = ent  \textbf{\quad unless \quad} t \in in \land loc.t \in plf $ %
\item[ \eqref{m1:saf2} ]$t \in in \land loc.t \in plf  \textbf{\quad unless \quad} t \in in \land loc.t = ext $ %
\item[ \eqref{m1:saf3} ]$t \in in \land loc.t = ext  \textbf{\quad unless \quad} \neg t \in in $ %
\end{block}

  \item   %!TEX root=../puzzle.tex

  \item   \textbf{events}
  \begin{block}
    \item   \noindent \ref{m0:enter} [t] \textbf{event}
\begin{block}
  \item   \textbf{when}
  \begin{block}
  \item[ \eqref{m0:enterent:grd1} ]$\neg t \in in $ %
  \end{block}
  \item   \textbf{begin}
  \begin{block}
  \item[ \eqref{m0:entera1} ]$in \bcmsuch in' = in \bunion \{ t \} $ %
  \item[ \eqref{m0:entera3} ]$loc \bcmsuch loc' = loc \1| t \fun ent $ %
  \end{block}
  \item   \textbf{end} \\
\end{block}

    \item   \noindent \ref{m0:leave} [t] \textbf{event}
\begin{block}
  \item   \textbf{during}
  \begin{block}
  \item[ \eqref{m0:leavelv:c0} ]$t \in in $ %
  \item[ \eqref{m0:leavelv:c1} ]$loc.t = ext $ %
  \end{block}
  \item   \textbf{when}
  \begin{block}
  \item[ \eqref{m0:leavelv:grd0} ]$t \in in $ %
  \item[ \eqref{m0:leavelv:grd1} ]$loc.t = ext $ %
  \end{block}
  \item   \textbf{begin}
  \begin{block}
  \item[ \eqref{m0:leavelv:a0} ]$in \bcmsuch in' = in \setminus \{ t \} $ %
  \item[ \eqref{m0:leavelv:a2} ]$loc \bcmsuch loc' = \{ t \} \domsub loc $ %
  \end{block}
  \item   \textbf{end} \\
\end{block}

    \item   \noindent \ref{m1:movein} [t] \textbf{event}
\begin{block}
  \item   \textbf{during}
  \begin{block}
  \item[ \eqref{m1:moveinmi:c1} ]$t \in in $ %
  \item[ \eqref{m1:moveinmi:c2} ]$loc.t = ent $ %
  \end{block}
  \item   \textbf{any} b
  \item   \textbf{when}
  \begin{block}
  \item[ \eqref{m1:moveinmi:g1} ]$t \in in $ %
  \item[ \eqref{m1:moveinmi:grd0} ]$loc.t = ent $ %
  \item[ \eqref{m1:moveinmi:grd7} ]$b \in plf $ %
  \end{block}
  \item   \textbf{begin}
  \begin{block}
  \item[ \eqref{m1:moveinmi:a2} ]$loc \bcmsuch loc' = loc \1| t \fun b $ %
  \end{block}
  \item   \textbf{end} \\
\end{block}

    \item   \noindent \ref{m1:moveout} [t] \textbf{event}
\begin{block}
\item \textbf{during}
\begin{block}
\item[ \eqref{m1:moveoutc1} ]$t \in in  %
		\1\land loc.t \in plf $ %
\end{block}
\item \textbf{when}
\begin{block}
\item[ \eqref{m1:moveoutmo:g1} ]$t \in in $ %
\item[ \eqref{m1:moveoutmo:g2} ]$loc.t \in plf $ %
\end{block}
\item \textbf{begin}
\begin{block}
\item[ \eqref{m1:moveoutSKIP:in} ]$in' = in$ %
\item[ \eqref{m1:moveouta2} ]$loc' = loc \1 | t \fun ext $ %
\end{block}
\item \textbf{end} \\
\end{block}

  \end{block}
  \item   \textbf{end} \\
\end{block}

\begin{machine}{m1}
	\refines{m0}

	\with{relations}
	\with{functions}
	\[ \variable{ptr: \set[Pair[Node,Node]]} \]
	% \[ \constant{star: \set[Pair[Node,Node]]\pfun \set[Pair[Node,Node]] } \]
\begin{align*}
	% & \assumption{m1:asm0}{ \qforall{rel}{}{rel \in \dom.star} } \\
	% & \assumption{m1:asm1}{ \qforall{r,x}{}
		% {(x\mapsto x) \in star.r} } \\
	& \progress{m1:prog0}{ p \in live \land \neg r \mapsto p \in \star ptr }{ p \in free }
\refine{m1:prog0}{ensure}{free}{ $\index{p}{p' = p }$ }
\end{align*}
	\[ \dummy{ p : Node } \]
	\removecoarse{free}{default}
\begin{align}
	\cschedule{free}{m1:sch0}{p \in live} \\
	\cschedule{free}{m1:sch1}{ \neg r \mapsto p \in \star ptr}
\end{align}
	\newevent{add}{ADD\_EDGE}
	\newevent{delete}{DELETE\_EDGE}
\begin{align}
	\evguard{add}{m1:grd0}{ r \mapsto p \in \star ptr } \\
	\evguard{add}{m1:grd1}{ r \mapsto q \in \star ptr } \\
	\evguard{delete}{m1:grd0}{ r \mapsto p \in \star ptr } \\
	\evbcmeq{add}{m1:act0}{ptr}{ ptr \bunion \{ p \mapsto q \} } \\
	\evbcmeq{delete}{m1:act0}{ptr}{ ptr \setminus \{ p \mapsto q \} } 
	% \assumption{m1:asm2}{ \qforall{r_0,r_1}{}
	% 	{ \star{(r_0 \setminus r_1)} \subseteq \star{r_0} } }
	% relational theory with relational image
	% \assumption{m1:asm3}{ \qforall{p,q,r,S}{}
		% { star.(r_0 \setminus r_1) \subseteq star.r_0 } }
\end{align}
	\[ \param{add}{p,q : Node} \]
	\[ \param{delete}{p,q : Node} \]
\begin{proof}{\ref{m1:prog0}/REF/ensure/m1/\ref{add}/SAF}
\begin{align}
	\define{rr}{ \{ r \mapsto r \} } \\
	\define{E}{ \{ p \mapsto q \} } \\
	\assert{lmm0}{ rr;\star (ptr \0\bunion E) \2\subseteq rr ; \star ptr }
\end{align}
\begin{free:var}{p}{ref}
\begin{align*}
	\assert{lmm1}{ r \mapsto ref \in \star ptr'
		\implies r \mapsto ref \in \star ptr }
\end{align*}
\easy
\begin{subproof}{lmm1}
\begin{calculation}
	r \mapsto ref \in \star ptr'
\hint{=}{ \eqref{m1:act0} }
	r \mapsto ref \in \star (ptr \bunion E)
\hint{=}{ identity }
	r \mapsto ref \in \star (ptr \bunion E)
\hint{=}{ aiming for $rr$ }
	r \mapsto ref \in \{r\} \rdomres (\star (ptr \bunion E))
\hint{=}{ }
	r \mapsto ref \in \star ptr
\end{calculation}
\end{subproof}
\end{free:var}
\begin{subproof}{lmm0}
	% * over \/ 
	% unroll *
	% ; over \/
\begin{calculation}
	rr;\star (ptr \0\bunion E)
\hint{=}{ }
	rr; \star (\star ptr ; E) ; \star ptr
\hint{=}{ }
	rr; (\star (\star ptr ; E) ; \star ptr) ; E ; \star ptr 
	\2\bunion rr ; \id ; \star ptr
\hint{\subseteq}{ }
	rr; \all ; E ; \star ptr 
	\2\bunion rr ; \id ; \star ptr
\hint{=}{ }
	rr ; \star ptr
\end{calculation}	
\end{subproof}
\end{proof}

\end{machine}

	%!TEX root=../main9.tex
\begin{block}
  \item   \textbf{machine} m2
  \item   \textbf{refines} m0
  \item   \textbf{variables}
  \begin{block}
    \item   $req$
    \item   $req0$
    \item   $reqA$
    \item   $reqB$
  \end{block}
  \item   %!TEX root=../main9.tex
\textbf{invariants}
\begin{block}
\item[ \eqref{m1:inv0} ]{$reqA \bunion reqB = req$} %
\item[ \eqref{m1:inv1} ]{$reqA \binter reqB = req$} %
\end{block}

  \item   \textbf{events}
  \begin{block}
    \item   %!TEX root=../main9.tex
\noindent \ref{handle}  \textbf{event}
\begin{block}
  \item   \textbf{during}
  \begin{block}
  \item[ \eqref{handlem0:sch0} ]{$\neg req = \emptyset$} %
  \end{block}
  \item   \textbf{any} r
  \item   \textbf{when}
  \begin{block}
  \item[ \eqref{handlegrd0} ]{$r \in req$} %
  \end{block}
  \item   \textbf{begin}
  \begin{block}
  \item[ \eqref{handleact0} ]{$req \bcmeq req \setminus \{ r \}$} %
  \item[ \eqref{handleact1} ]{$req0 \bcmeq req$} %
  \end{block}
  \item   \textbf{end} \\
\end{block}

    \item   %!TEX root=../main9.tex
\noindent \ref{req}  \textbf{event}
\begin{block}
  \item   \textbf{during}
  \begin{block}
  \item[ (\ref{req}/default) ]{$\false$} %
  \end{block}
  \item   \textbf{any} r
  \item   \textbf{when}
  \begin{block}
  \item[ \eqref{reqgrd0} ]{$\neg r \in req$} %
  \end{block}
  \item   \textbf{begin}
  \begin{block}
  \item[ \eqref{reqact0} ]{$req \bcmeq req \bunion \{ r \}$} %
  \item[ \eqref{reqact1} ]{$req0 \bcmeq req$} %
  \end{block}
  \item   \textbf{end} \\
\end{block}

  \end{block}
  \item   \textbf{end} \\
\end{block}


\begin{machine}{m2}
	\refines{m1}

	\[ \variable{reach : \set[Node]} \]
\begin{align}
	\invariant{m2:inv0}{ r \in reach } \\
	\initialization{m2:in0}{ reach = \{r\} }
	% \invariant{m1}
\end{align}
	\removecoarse{free}{m1:sch1}
\begin{align}
	\cschedule{free}{m2:sch0}
		{ \lookup{(\star ptr)}{~reach~} = reach } \\
	\cschedule{free}{m2:sch1}{ \neg p \in reach }
\end{align}

\end{machine}
\end{document}