
\documentclass[12pt]{amsart}
\usepackage{geometry} % see geometry.pdf on how to lay out the page. There's lots.
\usepackage{bsymb}
\usepackage{unitb}
\usepackage{calculation}
\usepackage{ulem}
\usepackage{hyperref}
\normalem
\geometry{a4paper} % or letter or a5paper or ... etc
% \geometry{landscape} % rotated page geometry

% See the ``Article customise'' template for some common customisations

\title{}
\author{}
\date{} % delete this line to display the current date

%%% BEGIN DOCUMENT
\setcounter{tocdepth}{4}
\begin{document}

\maketitle
\tableofcontents

\newcommand{\Train}{\text{TRAIN}}
\newcommand{\Blk}{\text{BLK}}

\section{Initial model}
\begin{machine}{m0}

\begin{align*}
	\false \tag{default} \label{default}
\end{align*}

\newset{\Train}

\with{sets}

\begin{align*}
\variable{	in : \set [ \Train ]}
\end{align*}

\begin{align*}
\dummy{	t : \Train}
\end{align*}

\newevent{m0:enter}{enter} 
\newevent{m0:leave}{leave}

\begin{align*}
\indices{m0:leave}{	t : \Train}
\end{align*}
\begin{align*}
\indices{m0:enter}{	t : \Train}
\end{align*}

\weakento{m0:leave}{default}{lv:c0}

\begin{align*}
\cschedule{m0:leave}{lv:c0}
	{	t &\in in } \\ 
\evassignment{m0:leave}{lv:a0}
	{	in' &= in \setminus \{ t \} } \\
\evassignment{m0:enter}{a1}
	{	in' &= in \bunion \{ t \} }
\end{align*}

\begin{align*}
&\progress{m0:prog0}
	{	t \in in }{ \neg t \in in }
\refine{m0:prog0}{discharge}{m0:tr0}{}
&\transientB{m0:leave}{m0:tr0}{ \index{t}{t} }
	{	t \in in }
\end{align*}

\end{machine}

\section{First refinement}
\begin{machine}{m1}



\refines{m0}

\newset{\Blk}

% \with{sets}
\with{functions}

\begin{align*}
\variable{	loc : \Train \pfun \Blk}
\end{align*}

\begin{align*}
\\ \initialization{in1}
	{ in = \emptyset }
\end{align*}

\begin{align*}
\invariant{inv0}
	{	\dom.loc = in }
\end{align*}

\begin{align*}
\constant{	ent,ext : \Blk} ; \quad
\constant{	plf : \set [ \Blk ]}
\end{align*}

\subsection{New requirements}
\begin{align*}
\safety{m1:saf0}
	{ \neg t \in in& }{ t \in in \land loc.t = ent }
\\ \safety{m1:saf1}
	{ t \in in \land loc.t = ent& }{ t \in in \land loc.t \in plf }
\\ \safety{m1:saf2}
	{ t \in in \land loc.t \in plf& }{ t \in in \land loc.t = ext }
\\ \safety{m1:saf3}
	{ t \in in \land loc.t = ext& }{ \neg t \in in }
\end{align*}

\subsection{Proofs}

\subsubsection{Invariant \ref{inv0}}

\begin{align*}
\evassignment{m0:leave}{lv:a2}
	{ loc' &= \{ t \} \domsub loc }
\\ \evassignment{m0:enter}{a3}
	{ loc' &= loc \1| t \fun ent }
\\ \initialization{in0}
	{ loc \, &= \emptyfun }
\end{align*}

\subsubsection{Safety \ref{m1:saf1}, \ref{m1:saf2}}

This takes care of \eqref{m1:saf2}:

\begin{align*}
\evguard{m0:leave}{lv:grd1}
	{ loc.t = ext }
\\ \evguard{m0:enter}{ent:grd1}
	{ \neg t \in in }
\end{align*}

This takes care of \eqref{m1:saf1}

\begin{align*}
\evguard{m0:leave}{lv:grd0}
	{ t \in in }
\\ \assumption{asm0}
	{ \neg ext \in plf \1\land \neg ext = ent }
\end{align*}

\subsubsection{Side conditions}

In order to take care of the schedulability of \ref{m0:leave}, we need to strengthen the coarse schedule by adding \ref{c1}. The side conditions for schedule replacement requires us to prove \ref{m1:prog0} and \ref{m1:saf3} in order to prove refinement.
\replace{m0:leave}{lv:c0}{lv:c0,lv:c1}{}{m1:prog0}{m1:saf3}

\begin{align*}
\cschedule{m0:leave}{lv:c1}
	{ loc.t = ext }
\\ \progress{m1:prog0}
	{ t \in in }{ t \in in \land loc.t = ext }
\end{align*}

The first step in implementing \eqref{m1:prog0} is to break it down in a few cases. The property says ``a train inside the station eventually reaches the exit''. Now, we're going to break the predicate ``the train is inside the station'' into ``the train is at the entrance'', ``the train is at a platform'' and ``the train is at the exit'', as described by the following assumption:
\hide{
	\begin{align*}
	\dummy{		b : \Blk}
	\end{align*}
}
\begin{align*}
\assumption{asm1}
{	\qforall{b}{}{ b \in \Blk \2\equiv b \in plf \1\lor b = ent \1\lor b = ext }	}
\end{align*}

Which allows us to appy \emph{disjunction} to \eqref{m1:prog0}.

\begin{align*}
& { t \in in }\3\mapsto{ t \in in \land loc.t = ext }
\refine{m1:prog0}{disjunction}{m1:prog1,m1:prog2,m1:prog3}{}
& \progress{m1:prog2}
	{ t \in in \land loc.t = ext }{ t \in in \land loc.t = ext }
\\ & \progress{m1:prog1}
	{ t \in in \land loc.t = ent }{ t \in in \land loc.t = ext }
\\ & \progress{m1:prog3}
	{ t \in in \land loc.t \in plf }{ t \in in \land loc.t = ext }
\end{align*} 
%
\hide{ \refine{m1:prog2}{implication}{}{} } %
%
\eqref{m1:prog2} is true by implication. This leaves us \eqref{m1:prog1} and \eqref{m1:prog3}. \eqref{m1:prog1} says that a train at the entrance reaches the exit. However, \eqref{m1:saf1} says that a train can only leave the entrance through a platform:

\begin{align*}
	& { t \in in \land loc.t = ent }\3\mapsto{ t \in in \land loc.t = ext } \tag{\ref{m1:prog1}}
\refine{m1:prog1}{transitivity}{m1:prog4,m1:prog3}{}
& \progress{m1:prog4}
	{ t \in in \land loc.t = ent }{ t \in in \land loc.t \in plf } 
\\ & { t \in in \land loc.t \in plf }\3\mapsto{ t \in in \land loc.t = ext } \tag{\ref{m1:prog3}}
\end{align*}

We introduce new events to satisfy \eqref{m1:prog3} and \eqref{m1:prog4}
\newevent{m1:movein}{move\_in} 
\newevent{m1:moveout}{move\_out} 
\hide{ 
	\refine{m1:prog3}{discharge}{m1:tr0,m1:saf2}{}
	\refine{m1:prog4}{discharge}{m1:tr1,m1:saf1}{}
}
\begin{align*} 
& \transientB{m1:movein}{m1:tr1}{ \index{t}{t} }
	{ t \in in \1\land loc.t = ent }
\\ & \transientB{m1:moveout}{m1:tr0}{ \index{t}{t} }
	{ t \in in \1\land loc.t \in plf }
\\ & \evguard{m1:movein}{mi:grd7}
	{ b \in plf }
\end{align*}
\subsubsection{New events} 


\begin{align*}
\indices{m1:moveout}{	t : \Train}
\end{align*}

We adjust \ref{m1:moveout} to satisfy \ref{m1:tr0}.

\weakento{m1:moveout}{default}{c1}{}

\begin{align*}
\evassignment{m1:moveout}{a2}{ loc' = loc \1 | t \fun ext }
\\ \assumption{asm2}
	{ \qexists{b}{}{b \in plf} }
\\ \cschedule{m1:moveout}{c1}{ t \in in \land loc.t \in plf }
\\ \evguard{m1:moveout}{mo:g1}{ t \in in }
\\ \evguard{m1:moveout}{mo:g2}{ loc.t \in plf }
\end{align*}

We adjust \ref{m1:movein} to satisfy \ref{m1:tr1}.

\weakento{m1:movein}{default}{mi:c1,mi:c2}{}

\begin{align*}
\indices{m1:movein}{	t : \Train} ; \quad
\param{m1:movein}{ b : \Blk }
\end{align*}

\begin{align*}
\evassignment{m1:movein}{mi:a2}
	{ loc' &= loc \1| t \fun b }
\\ \assumption{asm3}
	{ \neg ent &\in plf }
\\ \cschedule{m1:movein}{mi:c1}{ t &\in in } 
\\ \cschedule{m1:movein}{mi:c2}{ loc.t &= ent }
\\ \evguard{m1:movein}{mi:g1}{ t &\in in }
\\ \evguard{m1:movein}{mi:grd0}
	{ loc.t &= ent } % m1:saf3
\end{align*}
%
\end{machine}

\section{Second refinement}

\begin{machine}{m2}

\refines{m1} 

\begin{align*}
\dummy{	t_0,t_1 : \Train}
\end{align*}

\subsection{New Requirement}
\begin{align*}
\invariant{m2:inv0}
	{	\qforall{t_0,t_1}{t_0 \in in \land t_1 \in in \land loc.t_0 = loc.t_1}{t_0 = t_1}	}
\end{align*}
%
\subsection{Design}
%
\replace{m1:movein}{}{mi:c0}{}{m2:prog0}{m2:saf0}
\begin{align*}
\progress{m2:prog0}{\true}{\neg plf \subseteq \ran.loc  }
\\ \safetyB{m2:saf0}{m1:movein}{ \neg ~ plf \subseteq \ran.loc }{\false} % \2{\textbf{except}} \text{\ref{m1:movein}}
\end{align*}

%The above property, \eqref{m2:saf0}, should not discharge automatically.

\begin{align*} 
\evguard{m1:movein}{mi:g0}
	{	 \neg b \in \ran.loc  	}
\\ \cschedule{m1:movein}{mi:c0}
	{	\neg ~ plf \subseteq \ran.loc 	}
%\\ \evguard{m1:movein}{mi:g3}
%	{ \qforall{t}{t \in in}{ \neg loc.t = b} }
\\ \evguard{m0:enter}{et:g1}
	{	\neg ent \in \ran.loc } 
\\ \evguard{m1:moveout}{mo:g3}
	{	\neg ext \in \ran.loc 	}
\end{align*}
%

\begin{proof}{\ref{m0:leave}/INV/\ref{m2:inv0}}
\begin{free:var}{t_0}{t_0}
\begin{free:var}{t_1}{t_1}
	\begin{align}
	\assume{hyp0}{\neg t_0 = t_1}
	\hide{
		\\ \assume{hyp1}{ t_0 \in in' }
		\\ \assume{hyp2}{ t_1 \in in' }}
	\\ \assert{hyp5}{ t_0 \in in }
	\\ \assert{hyp6}{ t_1 \in in }
	\\ \assert{hyp3}{ \neg t_0 = t } % \label{hyp7}
	\\ \assert{hyp4}{ \neg t_1 = t } % \label{hyp8}
	\\ \goal{\neg (loc'.t_0 = loc'.t_1)} \notag
	\end{align}
	\hide{
		\assert{hyp7}{ t_0 \in \dom.loc \setminus \{ t \} }
		\assert{hyp8}{ t_1 \in \dom.loc \setminus \{ t \} }
		}
	\begin{calculation}
		loc'.t_0 = loc'.t_1
	\hint{=}{ \ref{lv:a2} \ref{m2:inv0} }
		(\{ t \} \domsub loc).t_0 = (\{ t \} \domsub loc).t_1 
	\hint{=}{ \hide{ \eqref{hyp7} and \eqref{hyp8} } 
			\igeqref{hyp3} and \igeqref{hyp4} }
		loc.t_0 = loc.t_1 
	\hint{=}{ \eqref{hyp5}, \eqref{hyp6}, 
			\eqref{hyp0} with \ref{m2:inv0} 
			} 
		\false
	\end{calculation}
	\hide
	{	\begin{subproof}{hyp3} \easy \end{subproof}
		\begin{subproof}{hyp4} \easy \end{subproof}
		\begin{subproof}{hyp5} \easy \end{subproof}
		\begin{subproof}{hyp6} \easy \end{subproof}
		}
	\hide{
		\begin{subproof}{hyp7} 
		\begin{calculation}
			t_0 \in \dom.loc \setminus \{ t \}
		\hint{=}{ \ref{inv0} }
			t_0 \in in \setminus \{ t \} 
		\hint{=}{ \ref{lv:a0} }
			t_0 \in in'
		\hint{=}{ \eqref{hyp1} }
			\true
		\end{calculation}
		\end{subproof}
		\begin{subproof}{hyp8} 
		\begin{calculation}
			t_1 \in \dom.loc \setminus \{ t \}
		\hint{=}{ \ref{inv0} }
			t_1 \in in \setminus \{ t \} 
		\hint{=}{ \ref{lv:a0} }
			t_1 \in in'
		\hint{=}{ \ref{hyp2} }
			\true
		\end{calculation}
		\end{subproof}
		}
\end{free:var}
\end{free:var}
\end{proof}

\replacefine{m1:moveout}{}{mo:f0}{m2:prog1} 

\begin{align*}
\fschedule{m1:moveout}{mo:f0}{ \neg ext \in \ran.loc } 
\\ \progress{m2:prog1}{\true}{ \neg ext \in \ran.loc }
\end{align*}

\begin{align*}
	& \true \2\mapsto \neg ext \in \ran. loc 
\refine{m2:prog1}{trading}{m2:prog2}{}
	& \progress{m2:prog2}{ext \in \ran. loc }{ \neg ext \in \ran. loc } 
\refine{m2:prog2}{disjunction}{m2:prog3}{}
	& \progress{m2:prog3}{ t \in in \land loc.t = ext }{\neg ext \in \ran. loc  } 
%\\	& \progress{m2:prog3b}{ t \in in \land loc.t = ext }{ t \in in \land \neg loc.t = ext  } 
\end{align*}
%
\hide{
	\refine{m2:prog3}{discharge}{m2:tr0,m2:saf1}{}
	}
%
We implement \eqref{m2:prog3} with the following:
%	
\begin{align*}
	\transientB{m0:leave}{m2:tr0}{ \index{t}{t} }
		{ t \in in \1\land loc.t = ext }
\\	\safety{m2:saf1}{  t \in in \land loc.t = ext }{ \neg ext \in \ran. loc } 
\end{align*}

%\begin{proof}{m1:movein/SCH/m2/0/REF/delay/prog/rhs}
%
%\end{proof}

\begin{align*}
	& {\true} \1\mapsto { \neg ~ plf \subseteq \ran.loc } 
%	\tag{\ref{m2:prog0}}
\refine{m2:prog0}{trading}{m2:prog4}{}
%	& {\qforall{b}{b \in plf}{ \qexists{t}{t \in in}{ loc.t = b}} \!\!} \3\mapsto
%		{\!\! \qexists{b}{b \in plf}{ \qforall{t}{t \in in}{ \neg loc.t = b}} }  
%	\notag \\
	& \progress{m2:prog4}{plf \subseteq \ran.loc \!\!}
		{\!\! \neg ~ plf \subseteq \ran.loc }
\refine{m2:prog4}{monotonicity}{m2:prog5}{}
%	& {\qexists{b,t}{b \in plf \land t \in in}{ loc.t = b}} \3\mapsto
%		{\qexists{b}{b \in plf}{ \qforall{t}{t \in in}{ \neg loc.t = b}} } \notag \\
	& \progress{m2:prog5}{\qexists{b}{b \in plf}{ b \in \ran.loc }}
		{ \qexists{b}{}{ b \in plf \1\land \neg b \in \ran.loc } }
\refine{m2:prog5}{disjunction}{m2:prog6}{}
	& \progress{m2:prog6}{ \qexists{t}{t \in in}{ b \in plf \1\land  loc.t = b } }
		{ b \in plf \land \neg b \in \ran.loc }
\refine{m2:prog6}{disjunction}{m2:prog7}{}
	& \progress{m2:prog7}{ t \in in \land b \in plf \1\land  loc.t = b }
		{ b \in plf \land \neg b \in \ran.loc }
\end{align*}

\hide{
	\refine{m2:prog7}{discharge}{m2:tr1,m2:saf2}{}
	}

\begin{align*}
	\transientB{m1:moveout}{m2:tr1}{ \lt{m2:prog1} \index{t}{t} }
		{ b \in plf \land t \in in \land loc.t = b }
\\ 	\safety{m2:saf2}{ t \in in \land b \in plf \1\land  loc.t = b }
		{ b \in plf \1\land \qforall{t}{ t \in in }{ \neg loc.t = b } }
\end{align*}

In order to prove \ref{m1:moveout}, we need a need progress property, to make sure that the fine schedule becomes true infinitely often. We can use the same that we used to introduce the fine schedule \eqref{m2:prog1}.

\subsection{Summary of the events} 

\begin{block}

\item[{}] Here are the events
\item[{}]
\noindent \ref{m0:enter} [t] \textbf{event}
\begin{block}
  \item   \textbf{when}
  \begin{block}
  \item[ \eqref{m0:enterent:grd1} ]$\neg t \in in $ %
  \item[ \eqref{m0:enteret:g1} ]$\neg ent \in \ran.loc $ %
  \end{block}
  \item   \textbf{begin}
  \begin{block}
  \item[ \eqref{m0:entera1} ]$in \bcmsuch in' = in \bunion \{ t \} $ %
  \item[ \eqref{m0:entera3} ]$loc \bcmsuch loc' = loc \1| t \fun ent $ %
  \end{block}
  \item   \textbf{end} \\
\end{block}

\item[{}]
\noindent \ref{m1:movein} [t] \textbf{event}
\begin{block}
\item \textbf{during}
\begin{block}
\item[ \eqref{mi:c0} ]$\!\! \neg ~ plf \subseteq \ran.loc $ %
\item[ \eqref{mi:c1} ]$t \in in $ %
\item[ \eqref{mi:c2} ]$loc.t = ent $ %
\end{block}
\item \textbf{any} b
\item \textbf{when}
\begin{block}
\item[ \eqref{mi:g0} ]$\neg b \in \ran.loc  	$ %
\item[ \eqref{mi:g1} ]$t \in in $ %
\item[ \eqref{mi:grd0} ]$loc.t = ent $ %
\item[ \eqref{mi:grd7} ]$b \in plf $ %
\end{block}
\item \textbf{begin}
\begin{block}
\item[ \eqref{SKIP:in} ]$in' = in$ %
\item[ \eqref{mi:a2} ]$loc' = loc \1| t \fun b $ %
\end{block}
\item \textbf{end} \\
\end{block}

\item[{}]
\noindent \ref{m1:moveout} [t] \textbf{event}
\begin{block}
  \item   \textbf{during}
  \begin{block}
  \item[ \eqref{m1:moveoutc1} ]$t \in in  %
  		\1\land loc.t \in plf $ %
  \end{block}
  \item   \textbf{upon}
  \begin{block}
  \item[ \eqref{m1:moveoutmo:f0} ]$\neg ext \in \ran. loc $ %
  \end{block}
  \item   \textbf{when}
  \begin{block}
  \item[ \eqref{m1:moveoutmo:g1} ]$t \in in $ %
  \item[ \eqref{m1:moveoutmo:g2} ]$loc.t \in plf $ %
  \item[ \eqref{m1:moveoutmo:g3} ]$\neg ext \in \ran. loc $ %
  \end{block}
  \item   \textbf{begin}
  \begin{block}
  \item[ \eqref{m1:moveouta2} ]$loc \bcmsuch loc' = loc \1 | t \fun ext $ %
  \end{block}
  \item   \textbf{end} \\
\end{block}

\item[{}]
\noindent \ref{m0:leave} [t] \textbf{event}
\begin{block}
\item \textbf{during}
\begin{block}
\item[ \eqref{lv:c0} ]$t \in in $ %
\item[ \eqref{lv:c1} ]$loc.t = ext $ %
\end{block}
\item \textbf{when}
\begin{block}
\item[ \eqref{lv:grd0} ]$t \in in $ %
\item[ \eqref{lv:grd1} ]$loc.t = ext $ %
\end{block}
\item \textbf{begin}
\begin{block}
\item[ \eqref{lv:a0} ]$in' = in \setminus \{ t \} $ %
\item[ \eqref{lv:a2} ]$loc' = \{ t \} \domsub loc $ %
\end{block}
\item \textbf{end} \\
\end{block}

\end{block}

\end{machine}

\section{Third refinement}

\begin{machine}{m3}

\refines{m2}

\[	\variable{isgn: \Bool} ; ~
	\variable{osgn: \set [\Blk] } 
\]

% \with{sets}
% \with{sets}
% \with{functions}


\begin{align*}
	\dummy{ b_0, b_1, p, p_0, p_1, p_2 : \Blk }
\\	\invariant{m3:inv0}{ osgn \subseteq plf }
\\	\invariant{m3:inv1}{ \qforall{p_0,p_1}{ p_0 \in osgn \land p_1 \in osgn }{ p_0 = p_1 } }
\\	\evassignment{m0:enter}{m3:ent:act0}{isgn' = isgn}
\\	\evassignment{m0:enter}{m3:ent:act1}{osgn' = osgn}
\\	\evassignment{m1:movein}{m3:mi:act0}{isgn' = isgn}
\\	\evassignment{m1:movein}{m3:mi:act1}{osgn' = osgn}
\\	\evassignment{m0:leave}{m3:ext:act0}{isgn' = isgn}
\\	\evassignment{m0:leave}{m3:ext:act1}{osgn' = osgn}
\\	\evassignment{m1:moveout}{m3:mo:act0}{isgn' = isgn}
\\	\evassignment{m1:moveout}{m3:mo:act1}{osgn' = osgn}
\\	\initialization{m3:init0}{ osgn = \emptyset }
\\	\initialization{m3:init1}{ isgn = \false }
\end{align*}

\begin{align*}
	\evguard{m1:moveout}{m3:mo:grd0}{ loc.t \in osgn } \\
	\fschedule{m1:moveout}{m3:mo:sch0}{ loc.t \in osgn }
\end{align*}

\replacefine{m1:moveout}{c1}{m3:mo:sch0}{m3:prog0}{}

\begin{align*}
\progress{m3:prog0}{\neg \, ext \in \ran.loc 
		\1\land t \in in 
		\1\land loc.t \in plf }{ loc.t \in osgn }
\end{align*}

\begin{align*} % if we remove inv, it doesn't prove
\invariant{m3:inv2}{ loc.t \in osgn \2\implies \neg \, ext \in \ran.loc } 
\end{align*}

\noindent \ref{m1:moveout} [t] \textbf{event}
\begin{block}
\item \textbf{during}
\begin{block}
\item[ \eqref{m1:moveoutc1} ]$t \in in  %
		\1\land loc.t \in plf $ %
\item[ \eqref{m1:moveoutm3:mo:sch0} ]$loc.t \in osgn $ %
\end{block}
\item \textbf{when}
\begin{block}
\item[ \eqref{m1:moveoutm3:mo:grd0} ]$loc.t \in osgn $ %
\item[ \eqref{m1:moveoutmo:g1} ]$t \in in $ %
\item[ \eqref{m1:moveoutmo:g2} ]$loc.t \in plf $ %
\end{block}
\item \textbf{begin}
\begin{block}
\item[ \eqref{m1:moveoutSKIP:in} ]$in' = in$ %
\item[ \eqref{m1:moveouta2} ]$loc' = loc \1 | t \fun ext $ %
\item[ \eqref{m1:moveoutm3:mo:act0} ]$isgn' = isgn$ %
\item[ \eqref{m1:moveoutm3:mo:act1} ]$osgn'  \2 = osgn  %
	\setminus \{ loc.t \} $ %
\end{block}
\item \textbf{end} \\
\end{block}


\end{machine}

\end{document}