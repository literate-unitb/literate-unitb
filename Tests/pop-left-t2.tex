
\documentclass[12pt]{amsart}
\usepackage[margin=0.5in]{geometry} 
  % see geometry.pdf on how to lay out the page. There's lots.
\usepackage{bsymb}
\usepackage{calculation}
\usepackage{ulem}
\usepackage{hyperref}
\usepackage{unitb}

\newcommand{\REQ}{\text{REQ}}
\newcommand{\OBJ}{\mathcal{O}bj}
\newcommand{\Addr}{\mathcal{P}tr}
\newcommand{\link}{\textit{LINK}}
\newcommand{\trash}{\textit{TRASH}}
\newcommand{\shared}{\textit{SHARED}}
\newcommand{\State}{\mathcal{S}tate}
\newcommand{\cEmpty}{\text{``empty''}}
\newcommand{\cNonEmpty}{\text{``non-empty''}}
\newcommand{\cInit}{\text{``init''}}
\newcommand{\cPopped}{\text{``popped''}}
\newcommand{\cBot}{\text{``idle''}}

\begin{document}

\tableofcontents

\section{m0}
\begin{block}
  \item   \textbf{machine} m0
  \item   \textbf{variables}
  \begin{block}
    \item   $in$
  \end{block}
  \item   \textbf{progress}
\begin{block}
\item[ \eqref{m0:prog0} ]$t \in in  \quad \mapsto\quad \neg t \in in $ %
\end{block}

  \item   \textbf{transient}
\begin{block}
\item[ \eqref{m0:tr0} ]$t \in in  \qquad \text{(\ref{m0:leave}: [t := t' | t' = t])}$ %
\end{block}

  \item   \textbf{events}
  \begin{block}
    \item   \noindent \ref{m0:enter} [t] \textbf{event}
\begin{block}
\item \textbf{during}
\begin{block}
\item[ \eqref{m0:enterdefault} ]$\false$ %
\end{block}
\item \textbf{begin}
\begin{block}
\item[ \eqref{m0:entera1} ]$in' = in \bunion \{ t \} $ %
\end{block}
\item \textbf{end} \\
\end{block}

    \item   \noindent \ref{m0:leave} [t] \textbf{event}
\begin{block}
  \item   \textbf{during}
  \begin{block}
  \item[ \eqref{m0:leavelv:c0} ]$t \in in $ %
  \end{block}
  \item   \textbf{begin}
  \begin{block}
  \item[ \eqref{m0:leavelv:a0} ]$in \bcmsuch in' = in \setminus \{ t \} $ %
  \end{block}
  \item   \textbf{end} \\
\end{block}

  \end{block}
  \item   \textbf{end} \\
\end{block}

\begin{machine}{m0}
    \with{functions}
    \with{sets}
    \[ \newset{\Addr_0} \]
    \[ \constant{ dummy : \Addr_0} \]
    \[ \constant{ \bot : \OBJ_0} \]
    \[ \definition
        {\Addr}{ \Addr_0 \setminus \{ dummy \} } \]
    \[ \definition
        {\OBJ}{ \OBJ_0 \setminus \{ \bot \} } \]
    \[ \definition
        {\Node}{ [ 'item : \OBJ_0 \setminus \{ \bot \} ] } \]
    \[\newset{\OBJ_0}\]

shared:
    \[ \variable{ LH, RH : \Addr_0 } \]
    \[ \variable{ free : \set [\Addr_0] } \]
    \[ \variable{ ver : \Int } \]
    \[ \variable{ result : \OBJ_0 } \]
    \[ \variable{ \link, \trash : \Addr_0 \pfun 
        \{ 'item : \OBJ_0 
         , 'left : \Addr_0
         , 'right : \Addr_0  \} } \]

private:
    \[ \variable{ popL,remL : \Bool } \]
    \[ \variable{ resL : \OBJ_0 } \]

\begin{align*}
  \initialization{m0:init0}{ LH = dummy } \\
  \initialization{m0:init1}{ RH = dummy } \\
  \initialization{m0:init2}{ free = \Addr } \\
  \initialization{m0:init3}{ ver = 0 } \\
  \initialization{m0:init4}{ result = \bot } \\
  \initialization{m0:init5}{ \link = \emptyfun } \\
  \initialization{m0:init6}{ \trash = \emptyfun } \\
  \initialization{m0:init7}{ popL = \false } \\
  \initialization{m0:init8}{ remL = \false } \\
  \initialization{m0:init9}{ resL = \bot } \\
\end{align*}

internal:
\newevent{add:popL}{req\_popL}
\newevent{hdl:popL:empty}{hdl\_popL\_empty}
\newevent{hdl:popL:one}{hdl\_popL\_one}
\newevent{hdl:popL:more}{hdl\_popL\_more}
\newevent{returnL}{returnL}

\begin{align*}
    \evguard{add:popL}{m0:grd0}{\neg popL} \\
    \evbcmeq{add:popL}{m0:act0}{popL}{\true} \\
    \cschedule{returnL}{m0:sch0}{remL} \\
    \evbcmeq{returnL}{m0:act0}{result}{resL} \\
    \evbcmeq{returnL}{m0:act1}{remL}{\false} \\    
\end{align*}

\begin{align*}
  \theorem{m0:thm0:ASM}{ LH \in \dom.\link \bunion \{ dummy \} } \\
  \theorem{m0:thm1:ASM}{ RH \in \dom.\link \bunion \{ dummy \} } \\
  \theorem{m0:thm2:ASM}{ \neg LH \in \dom.\trash  } \\
  \theorem{m0:thm3:ASM}{ \neg RH \in \dom.\trash  } \\
  \invariant{m0:inv0}{ free \subseteq \Addr }
\end{align*}

\[ \indices{hdl:popL:empty}{ v : \Int } \]
\[ \indices{hdl:popL:one}{ v : \Int } \]
\[ \indices{hdl:popL:more}{ v : \Int } \]

\begin{align*}
  \cschedule{hdl:popL:empty}{m0:sch0}{v = ver} \\
  \cschedule{hdl:popL:one}{m0:sch0}{v = ver} \\
  \cschedule{hdl:popL:more}{m0:sch0}{v = ver} \\
  \cschedule{hdl:popL:empty}{m0:sch1}{popL} \\
  \cschedule{hdl:popL:one}{m0:sch1}{popL} \\
  \cschedule{hdl:popL:more}{m0:sch1}{popL} \\
  \cschedule{hdl:popL:empty}{m0:sch2}{LH = dummy} \\
  \cschedule{hdl:popL:one}{m0:sch2}{\neg LH = dummy} \\
  \cschedule{hdl:popL:more}{m0:sch2}{\neg LH = dummy} \\
  \cschedule{hdl:popL:one}{m0:sch3}{LH = RH} \\
  \cschedule{hdl:popL:more}{m0:sch3}{\neg LH = RH} \\
  \evbcmeq{hdl:popL:empty}{m0:act0}
    {ver}{ver + 1} \\
  \evbcmeq{hdl:popL:one}{m0:act0}
    {ver}{ver + 1} \\
  \evbcmeq{hdl:popL:more}{m0:act0}
    {ver}{ver + 1} \\
  \evbcmeq{hdl:popL:empty}{m0:act1}
    {popL}{\false} \\
  \evbcmeq{hdl:popL:one}{m0:act1}
    {popL}{\false} \\
  \evbcmeq{hdl:popL:more}{m0:act1}
    {popL}{\false} \\
  \evbcmeq{hdl:popL:empty}{m0:act2}
    {remL}{\true} \\
  \evbcmeq{hdl:popL:one}{m0:act2}
    {remL}{\true} \\
  \evbcmeq{hdl:popL:more}{m0:act2}
    {remL}{\true} \\
  \evbcmeq{hdl:popL:empty}{m0:act3}
    {resL}{\bot} \\
  \evbcmeq{hdl:popL:one}{m0:act3}
    {resL}{\link.LH.'item} \\
  \evbcmeq{hdl:popL:more}{m0:act3}
    {resL}{\link.LH.'item} \\
  \evbcmeq{hdl:popL:one}{m0:act4}
    {\trash}{\trash \1| LH \fun \link.LH} \\
  \evbcmeq{hdl:popL:more}{m0:act4}
    {\trash}{\trash \1| LH \fun \link.LH} \\
  \evbcmeq{hdl:popL:one}{m0:act5}
    {\link}{ \{LH\} \domsub \link } \\
  \evbcmeq{hdl:popL:more}{m0:act5}
    {\link}{ \{LH\} \domsub \link } \\
  \evbcmeq{hdl:popL:one}{m0:act6}
    {LH}{ dummy } \\
  \evbcmeq{hdl:popL:one}{m0:act7}
    {RH}{ dummy } \\
  \evbcmeq{hdl:popL:more}{m0:act6}
    {LH}{ \link.LH.'right } \\
\end{align*}

external:
\newevent{ext:pushL:empty}{EXT\_pushL\_empty}
\newevent{ext:pushL:non:empty}{EXT\_pushL\_non\_empty}
\newevent{ext:pushR:empty}{EXT\_pushR\_empty}
\newevent{ext:pushR:non:empty}{EXT\_pushR\_non\_empty}
\newevent{ext:popL:empty}{EXT\_popL\_empty}
\newevent{ext:popL:one}{EXT\_popL\_one}
\newevent{ext:popL:more}{EXT\_popL\_more}
\newevent{ext:popR:empty}{EXT\_popR\_empty}
\newevent{ext:popR:one}{EXT\_popR\_one}
\newevent{ext:popR:more}{EXT\_popR\_more}
\newevent{ext:return}{EXT\_return}

\paragraph{pushL --- empty}

\begin{align*}
  \param{ext:pushL:empty}{x : \OBJ_0 } \\
  \param{ext:pushL:empty}{n : \Addr_0 } \\
\end{align*}

\begin{align*}
  \evguard{ext:pushL:empty}{m0:grd0}{LH = dummy } \\
  \evguard{ext:pushL:empty}{m0:grd1}{x \in \OBJ } \\
  \evguard{ext:pushL:empty}{m0:grd2}{n \in free } \\
  \evbcmeq{ext:pushL:empty}{m0:act0}
      {ver}{ver + 1} \\
  \evbcmeq{ext:pushL:empty}{m0:act1}
      {free}{free \setminus \{ n \} } \\
  \evbcmeq{ext:pushL:empty}{m0:act2}
      {\link}{\link \1| n \fun 
          \left[ \begin{array}{l}
            'item := x, \\
            'left := dummy, \\
            'right := dummy
          \end{array} \right] } \\
  \evbcmeq{ext:pushL:empty}{m0:act3}
      {LH}{n} \\
  \evbcmeq{ext:pushL:empty}{m0:act4}
      {RH}{n} \\
\end{align*}

\paragraph{pushL --- non-empty}


\begin{align*}
  \param{ext:pushL:non:empty}{x : \OBJ_0 } \\
  \param{ext:pushL:non:empty}{n : \Addr_0 } \\
\end{align*}

\begin{align*}
  \evguard{ext:pushL:non:empty}
    {m0:grd0}{ \neg LH = dummy } \\
  \evguard{ext:pushL:non:empty}
    {m0:grd1}{x \in \OBJ } \\
  \evguard{ext:pushL:non:empty}
    {m0:grd2}{n \in free } \\
  \evbcmeq{ext:pushL:non:empty}{m0:act0}
      {ver}{ver + 1} \\
  \evbcmeq{ext:pushL:non:empty}{m0:act1}
      {free}{free \setminus \{ n \} } \\
  \evbcmeq{ext:pushL:non:empty}{m0:act2}
      {\link}{\link \1| n \fun 
          \left[ \begin{array}{l}
            'item := x, \\
            'left := dummy, \\
            'right := LH
          \end{array} \right] } \\
  \evbcmeq{ext:pushL:non:empty}{m0:act3}
      {LH}{n} \\
  \evbcmeq{ext:pushL:non:empty}{m0:act4}
      {\link}{ \link \1| LH \fun 
        (\link.LH) [ 'left := n ] } \\
\end{align*}

\paragraph{pushR --- empty}


\begin{align*}
  \param{ext:pushR:empty}{x : \OBJ_0 } \\
  \param{ext:pushR:empty}{n : \Addr_0 } \\
\end{align*}

\begin{align*}
  \evguard{ext:pushR:empty}
      {m0:grd0}{RH = dummy } \\
  \evguard{ext:pushR:empty}
      {m0:grd1}{x \in \OBJ } \\
  \evguard{ext:pushR:empty}
      {m0:grd2}{n \in free } \\
  \evbcmeq{ext:pushR:empty}{m0:act0}
      {ver}{ver + 1} \\
  \evbcmeq{ext:pushR:empty}{m0:act1}
      {free}{free \setminus \{ n \} } \\
  \evbcmeq{ext:pushR:empty}{m0:act2}
      {\link}{\link \1| n \fun 
          \left[ \begin{array}{l}
            'item := x, \\
            'left := dummy,
            'right := dummy
          \end{array} \right] } \\
  \evbcmeq{ext:pushR:empty}{m0:act3}
      {LH}{n} \\
  \evbcmeq{ext:pushR:empty}{m0:act4}
      {RH}{n} \\
\end{align*}

\paragraph{pushR --- non-empty}

\begin{align*}
  \param{ext:pushR:non:empty}{x : \OBJ_0 } \\
  \param{ext:pushR:non:empty}{n : \Addr_0 } \\
\end{align*}

\begin{align*}
  \evguard{ext:pushR:non:empty}
    {m0:grd0}{ \neg RH = dummy } \\
  \evguard{ext:pushR:non:empty}
    {m0:grd1}{x \in \OBJ } \\
  \evguard{ext:pushR:non:empty}
    {m0:grd2}{n \in free } \\
  \evbcmeq{ext:pushR:non:empty}{m0:act0}
      {ver}{ver + 1} \\
  \evbcmeq{ext:pushR:non:empty}{m0:act1}
      {free}{free \setminus \{ n \} } \\
  \evbcmeq{ext:pushR:non:empty}{m0:act2}
      {\link}{\link \1| n \fun 
          \left[ \begin{array}{l}
            'item := x, \\
            'left := RH,
            'right := dummy
          \end{array} \right] } \\
  \evbcmeq{ext:pushR:non:empty}{m0:act3}
      {RH}{n} \\
  \evbcmeq{ext:pushR:non:empty}{m0:act4}
      {\link}{ \link \1| RH \fun 
        (\link.RH) [ 'right := n ] } \\
\end{align*}

\paragraph{popL --- empty}

\begin{align*}
  \evguard{ext:popL:empty}
      {m0:grd0}{LH = dummy} \\
  \evbcmeq{ext:popL:empty}{m0:act0}
      {ver}{ver + 1}
\end{align*}

\paragraph{popL --- one}

\begin{align*}
  \evguard{ext:popL:one}
      {m0:grd0}{\neg LH = dummy} \\
  \evguard{ext:popL:one}
      {m0:grd1}{LH = RH} \\
  \evbcmeq{ext:popL:one}{m0:act0}
      {ver}{ver + 1} \\
  \evbcmeq{ext:popL:one}{m0:act1}
      {\link}{ \{LH\} \domsub \link } \\
  \evbcmeq{ext:popL:one}{m0:act2}
      {LH}{ dummy } \\
  \evbcmeq{ext:popL:one}{m0:act3}
      {RH}{ dummy } \\
  \evbcmeq{ext:popL:one}{m0:act4}
      {\trash}{ \trash \1| LH \fun \link.LH } \\
\end{align*}

\paragraph{popL --- more}

\begin{align*}
  \evguard{ext:popL:more}
      {m0:grd0}{\neg LH = dummy} \\
  \evguard{ext:popL:more}
      {m0:grd1}{\neg LH = RH} \\
  \evbcmeq{ext:popL:more}{m0:act0}
      {ver}{ver + 1} \\
  \evbcmeq{ext:popL:more}{m0:act1}
      {\link}{ \{LH\} \domsub \link } \\
  \evbcmeq{ext:popL:more}{m0:act2}
      {LH}{ \link.LH.'right } \\
  \evbcmeq{ext:popL:more}{m0:act3}
      {\trash}{ \trash \1| LH \fun \link.LH } \\
\end{align*}

\paragraph{popR --- empty}

\begin{align*}
  \evguard{ext:popR:empty}
      {m0:grd0}{RH = dummy} \\
  \evbcmeq{ext:popR:empty}{m0:act0}
      {ver}{ver + 1}
\end{align*}

\paragraph{popR --- one}

\begin{align*}
  \evguard{ext:popR:one}
      {m0:grd0}{\neg RH = dummy} \\
  \evguard{ext:popR:one}
      {m0:grd1}{LH = RH} \\
  \evbcmeq{ext:popR:one}{m0:act0}
      {ver}{ver + 1} \\
  \evbcmeq{ext:popR:one}{m0:act1}
      {\link}{ \{RH\} \domsub \link } \\
  \evbcmeq{ext:popR:one}{m0:act2}
      {LH}{ dummy } \\
  \evbcmeq{ext:popR:one}{m0:act3}
      {RH}{ dummy } \\
  \evbcmeq{ext:popR:one}{m0:act4}
      {\trash}{ \trash \1| RH \fun \link.RH } \\
\end{align*}

\paragraph{popR --- more}

\begin{align*}
  \evguard{ext:popR:more}
      {m0:grd0}{\neg RH = dummy} \\
  \evguard{ext:popR:more}
      {m0:grd1}{\neg LH = RH} \\
  \evbcmeq{ext:popR:more}{m0:act0}
      {ver}{ver + 1} \\
  \evbcmeq{ext:popR:more}{m0:act1}
      {\link}{ \{RH\} \domsub \link } \\
  \evbcmeq{ext:popR:more}{m0:act2}
      {RH}{ \link.RH.'left } \\
  \evbcmeq{ext:popR:more}{m0:act3}
      {\trash}{ \trash \1| RH \fun \link.RH } \\
\end{align*}

\paragraph{return}

\end{machine}
\section{m1}

\begin{block}
  \item   \textbf{machine} m1
  \item   \textbf{variables}
  \begin{block}
    \item   $in$
    \item   $loc$
  \end{block}
  \item   %!TEX root=../main8.tex
\textbf{invariants}
\begin{block}
\item[ \eqref{m1:inv0} ]{$qe \in \intervalR{p}{q} \tfun \G $} %
\item[ \eqref{m1:inv1} ]{$p \le q $} %
\end{block}

  \item   \textbf{progress}
\begin{block}
\item[ \eqref{m1:prog0} ]$r \in \dom.pshL \1\land pshL.r = x  \quad \mapsto\quad p < q \land qe.p = x \1\land \neg r \in \dom.pshL $ %
\item[ \eqref{m1:prog1} ]$r \in \dom.pshR \1\land pshR.r = x  \quad \mapsto\quad p < q \1\land qe.(q-1) = x \1\land \neg r \in \dom.pshR $ %
\item[ \eqref{m1:prog2} ]$r \in popR  \quad \mapsto\quad \neg r \in popR $ %
\item[ \eqref{m1:prog3} ]$r \in popL  \quad \mapsto\quad \neg r \in popL  $ %
\end{block}

  \item   \textbf{safety}
\begin{block}
\item[ \eqref{m1:saf0} ]$\neg t \in in  \textbf{\quad unless \quad} t \in in \land loc.t = ent $ %
\item[ \eqref{m1:saf1} ]$t \in in \land loc.t = ent  \textbf{\quad unless \quad} t \in in \land loc.t \in plf $ %
\item[ \eqref{m1:saf2} ]$t \in in \land loc.t \in plf  \textbf{\quad unless \quad} t \in in \land loc.t = ext $ %
\item[ \eqref{m1:saf3} ]$t \in in \land loc.t = ext  \textbf{\quad unless \quad} \neg t \in in $ %
\end{block}

  \item   %!TEX root=../puzzle.tex

  \item   \textbf{events}
  \begin{block}
    \item   \noindent \ref{m0:enter} [t] \textbf{event}
\begin{block}
  \item   \textbf{when}
  \begin{block}
  \item[ \eqref{m0:enterent:grd1} ]$\neg t \in in $ %
  \end{block}
  \item   \textbf{begin}
  \begin{block}
  \item[ \eqref{m0:entera1} ]$in \bcmsuch in' = in \bunion \{ t \} $ %
  \item[ \eqref{m0:entera3} ]$loc \bcmsuch loc' = loc \1| t \fun ent $ %
  \end{block}
  \item   \textbf{end} \\
\end{block}

    \item   \noindent \ref{m0:leave} [t] \textbf{event}
\begin{block}
  \item   \textbf{during}
  \begin{block}
  \item[ \eqref{m0:leavelv:c0} ]$t \in in $ %
  \item[ \eqref{m0:leavelv:c1} ]$loc.t = ext $ %
  \end{block}
  \item   \textbf{when}
  \begin{block}
  \item[ \eqref{m0:leavelv:grd0} ]$t \in in $ %
  \item[ \eqref{m0:leavelv:grd1} ]$loc.t = ext $ %
  \end{block}
  \item   \textbf{begin}
  \begin{block}
  \item[ \eqref{m0:leavelv:a0} ]$in \bcmsuch in' = in \setminus \{ t \} $ %
  \item[ \eqref{m0:leavelv:a2} ]$loc \bcmsuch loc' = \{ t \} \domsub loc $ %
  \end{block}
  \item   \textbf{end} \\
\end{block}

    \item   \noindent \ref{m1:movein} [t] \textbf{event}
\begin{block}
  \item   \textbf{during}
  \begin{block}
  \item[ \eqref{m1:moveinmi:c1} ]$t \in in $ %
  \item[ \eqref{m1:moveinmi:c2} ]$loc.t = ent $ %
  \end{block}
  \item   \textbf{any} b
  \item   \textbf{when}
  \begin{block}
  \item[ \eqref{m1:moveinmi:g1} ]$t \in in $ %
  \item[ \eqref{m1:moveinmi:grd0} ]$loc.t = ent $ %
  \item[ \eqref{m1:moveinmi:grd7} ]$b \in plf $ %
  \end{block}
  \item   \textbf{begin}
  \begin{block}
  \item[ \eqref{m1:moveinmi:a2} ]$loc \bcmsuch loc' = loc \1| t \fun b $ %
  \end{block}
  \item   \textbf{end} \\
\end{block}

    \item   \noindent \ref{m1:moveout} [t] \textbf{event}
\begin{block}
\item \textbf{during}
\begin{block}
\item[ \eqref{m1:moveoutc1} ]$t \in in  %
		\1\land loc.t \in plf $ %
\end{block}
\item \textbf{when}
\begin{block}
\item[ \eqref{m1:moveoutmo:g1} ]$t \in in $ %
\item[ \eqref{m1:moveoutmo:g2} ]$loc.t \in plf $ %
\end{block}
\item \textbf{begin}
\begin{block}
\item[ \eqref{m1:moveoutSKIP:in} ]$in' = in$ %
\item[ \eqref{m1:moveouta2} ]$loc' = loc \1 | t \fun ext $ %
\end{block}
\item \textbf{end} \\
\end{block}

  \end{block}
  \item   \textbf{end} \\
\end{block}


\begin{machine}{m1}
  \refines{m0}
  \[ \newset{\State} \]
  \[ \constant{\cInit : \State} \]
  \[ \constant{\cBot : \State} \]
  \[ \constant{\cPopped : \State} \]
  \[ \constant{\cEmpty : \State} \]
  \[ \constant{\cNonEmpty : \State} \]
  \[ \assumption{m0:asm0}{ \neg \cInit = \cBot } \]
  \[ \assumption{m0:asm1}{ \neg \cInit = \cEmpty } \]
  \[ \assumption{m0:asm2}{ \neg \cInit = \cNonEmpty } \]
  \[ \assumption{m0:asm3}{ \neg \cInit = \cPopped } \]
  \[ \assumption{m0:asm4}{ \neg \cBot = \cEmpty } \]
  \[ \assumption{m0:asm5}{ \neg \cBot = \cNonEmpty } \]
  \[ \assumption{m0:asm6}{ \neg \cBot = \cPopped } \]
  \[ \assumption{m0:asm7}{ \neg \cEmpty = \cNonEmpty } \]
  \[ \assumption{m0:asm8}{ \neg \cEmpty = \cPopped } \]
  \[ \assumption{m0:asm9}{ \neg \cNonEmpty = \cPopped } \]
  \[ \variable{lh : \Addr_0} \]
  \[ \variable{rh : \Addr_0} \]
  \[ \variable{state : \State} \]
  \begin{align*}
    \initialization{m1:init0}
      { lh = dummy } \\
    \initialization{m1:init1}
      { rh = dummy } \\
    \initialization{m1:init2}
      { state = \cBot } \\
  \end{align*}

\newevent{read:LH}{read\_LH}

\begin{align*}
  \cschedule{read:LH}{m1:sch0}
    {state \in \{ \cInit, \cNonEmpty \} } \\
  \cschedule{read:LH}{m1:sch1}
    { \neg LH = dummy } \\
  \evbcmeq{read:LH}{m1:act0}
    {lh}{LH} \\
  \evbcmeq{read:LH}{m1:act1}
    {state}{ \ifelse{LH = dummy}{\cEmpty}{\cNonEmpty} } \\
  \evbcmeq{read:LH}{m1:act2}
    {ver}{ \ifelse{LH = dummy}{ver + 1}{ver} } \\
\end{align*}

\subsection{Invariants}

\begin{align*}
  \invariant{m1:inv0}
    {state = \cBot & \3\equiv \neg popL \land \neg remL} \\
  % \invariant{m1:inv1}
  %   {state = \cInit & \3\implies popL} \\
  % \invariant{m1:inv2}
  %   {state = \cNonEmpty & \3\implies popL \land \neg lh = dummy} \\
  \invariant{m1:inv3}
    {popL & \3\equiv state \in \{\cInit,\cNonEmpty\}} \\
  \invariant{m1:inv4}
    { state = \cPopped & \3\implies \left( 
      \begin{array}{ll}
         & remL \\
         \land & lh \in \dom.\trash \\
         \land & \trash.lh.'item = resL
       \end{array} \right) } \\
  \invariant{m1:inv5}
    { remL &\3\equiv state \in \{ \cPopped, \cEmpty \} } \\
  \invariant{m1:inv7}
    { \qforall{p}{p \in \dom.\link}{\link.p.'item \in \OBJ} } \\
  \invariant{m1:inv6}
    { state = \cEmpty & \3\equiv remL \land resL = \bot } \\
\end{align*}

\subsection{State}

\removevar{popL,remL,resL}

\removeguard{add:popL}{m0:grd0}
\removeact{add:popL}{m0:act0}
\begin{align*}
  \evguard{add:popL}{m1:grd0}
    { state = \cBot } \\
  \evbcmeq{add:popL}{m1:act0}
    {state}{ \cInit } \\
\end{align*}

\removeact{hdl:popL:empty}{m0:act1}
\removeact{hdl:popL:empty}{m0:act2}
\removeact{hdl:popL:empty}{m0:act3}
\removecoarse{hdl:popL:empty}{m0:sch1}
\begin{align*}
  &\cschedule{hdl:popL:empty}{m1:sch0}
    { state \in \{ \cInit, \cNonEmpty \} } \\
  &\evbcmeq{hdl:popL:empty}{m1:act0}
    { state }{ \ifelse{ LH = dummy }{\cEmpty}{\cNonEmpty} } \\
\end{align*}

\removeact{hdl:popL:more}{m0:act1}
\removeact{hdl:popL:more}{m0:act2}
\removeact{hdl:popL:more}{m0:act3}
\removecoarse{hdl:popL:more}{m0:sch0}{popL}
\removecoarse{hdl:popL:more}{m0:sch1}{popL}
\begin{align*}
  &\cschedule{hdl:popL:more}{m1:sch0}
    { state = \cNonEmpty } \\
  &\cschedule{hdl:popL:more}{m1:sch1}
    { lh = LH } \\
  &\evbcmeq{hdl:popL:more}{m1:act0}
    { state }{ \cPopped } \\
  % &\evbcmeq{hdl:popL:more}{m1:act0}
  %   { remL }{ \true } \\
\end{align*}

\replace{hdl:popL:more}{m1:sch0,m1:sch1}{m1:prog0}
\replace{hdl:popL:one}{m1:sch0,m1:sch1}{m1:prog0}
\begin{align*}
  \progress{m1:prog0}
    {v = ver \1\land \neg LH = dummy 
        \1\land state \in \{ \cInit, \cNonEmpty \} }
    { (LH = lh \1\land \neg LH = dummy \1\land state = \cNonEmpty)
      \1\lor \neg v = ver}
  \refine{m1:prog0}{ensure}{read:LH}{}
  % \progress{m1:prog0}
  %   {v = ver \1\land \neg LH = dummy 
  %       \1\land state \in \{ \cInit, \cNonEmpty \} }
  %   { (LH = lh \1\land \neg LH = dummy \1\land state = \cNonEmpty)
  %     \1\lor \neg v = ver}
  % \refine{m1:prog1}{ensure}{read:LH}{}
\end{align*}
\[ \dummy{v : \Int} \]

\removeact{hdl:popL:one}{m0:act1}
\removeact{hdl:popL:one}{m0:act2}
\removeact{hdl:popL:one}{m0:act3}
\removecoarse{hdl:popL:one}{m0:sch0}
\removecoarse{hdl:popL:one}{m0:sch1}
\removeind{hdl:popL:one}{v}
\begin{align*}
  &\cschedule{hdl:popL:one}{m1:sch0}
    { state = \cNonEmpty } \\
  &\cschedule{hdl:popL:one}{m1:sch1}
    { lh = LH } \\
  &\evbcmeq{hdl:popL:one}{m1:act0}
    { state }{ \cPopped } \\
\end{align*}

\[ \splitevent{returnL}{returnL:empty,returnL:non:empty} \]
% \begin{align*}
    \refiningevent{returnL}{returnL:empty}{returnL\_empty} \\
    \refiningevent{returnL}{returnL:non:empty}{returnL\_non\_empty}  \\
% \end{align*}
\hide{
\removeact{returnL:empty}{m0:act0}
\removeact{returnL:empty}{m0:act1}
\removecoarse{returnL:empty}{m0:sch0}
\removeact{returnL:non:empty}{m0:act0}
\removeact{returnL:non:empty}{m0:act1}
\removecoarse{returnL:non:empty}{m0:sch0}}
% \replace{returnL}{m0:sch0}{m1:sch}{}{}
\begin{align*}
  \cschedule{returnL:empty}{m1:sch0}{state = \cEmpty} \\
  \evbcmeq{returnL:empty}{m1:act0}{state}{\cBot} \\
  \evbcmeq{returnL:empty}{m1:act1}{result}{\bot} \\
  \cschedule{returnL:non:empty}{m1:sch0}{state = \cPopped} \\
  \evbcmeq{returnL:non:empty}{m1:act0}{state}{\cBot} \\
  \evbcmeq{returnL:non:empty}{m1:act1}{result}{\trash.lh.'item} \\
\end{align*}

\removeinit{m0:init7}
\removeinit{m0:init8}
\removeinit{m0:init9}
\begin{align*}
  \initwitness{resL}{resL = \bot} \\
  \initwitness{remL}{remL = \false}\\
  \initwitness{popL}{popL = \false}
\end{align*}

\subsection{Delete ver}

\end{machine}

\end{document}
